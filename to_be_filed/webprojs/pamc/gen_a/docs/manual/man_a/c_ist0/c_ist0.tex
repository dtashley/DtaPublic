%$Header: /home/dashley/cvsrep/e3ft_gpl01/e3ft_gpl01/webprojs/pamc/gen_a/docs/manual/man_a/c_ist0/c_ist0.tex,v 1.9 2009/11/04 16:50:19 dashley Exp $

\chapter{Installation of \emph{\productbasename{}-\productversion{}}}

\label{cist0}

\beginchapterquote{``A distributed system is one in which the failure of
                     a computer you didn't even know existed can render 
                     your own computer unusable.''}
                  {Les Lamport, as quoted in newsgroup post by Richard Heylen}


%%%%%%%%%%%%%%%%%%%%%%%%%%%%%%%%%%%%%%%%%%%%%%%%%%%%%%%%%%%%%%%%%%%%%%%%%%%%%%%
%%%%%%%%%%%%%%%%%%%%%%%%%%%%%%%%%%%%%%%%%%%%%%%%%%%%%%%%%%%%%%%%%%%%%%%%%%%%%%%
%%%%%%%%%%%%%%%%%%%%%%%%%%%%%%%%%%%%%%%%%%%%%%%%%%%%%%%%%%%%%%%%%%%%%%%%%%%%%%%
\section{Introduction}
%Section tag:  INT0
\label{cist0:sint0}

This chapter provides instructions for installing
\emph{\productbasename{}-\productversion{}}.


%%%%%%%%%%%%%%%%%%%%%%%%%%%%%%%%%%%%%%%%%%%%%%%%%%%%%%%%%%%%%%%%%%%%%%%%%%%%%%%
%%%%%%%%%%%%%%%%%%%%%%%%%%%%%%%%%%%%%%%%%%%%%%%%%%%%%%%%%%%%%%%%%%%%%%%%%%%%%%%
%%%%%%%%%%%%%%%%%%%%%%%%%%%%%%%%%%%%%%%%%%%%%%%%%%%%%%%%%%%%%%%%%%%%%%%%%%%%%%%
\section{System Requirements}
%Section tag:  srq0
\label{cist0:ssrq0}

\index{system requirements}In order to install 
\emph{\productbasename{}-\productversion{}}, the
server must meet the following requirements:

\begin{itemize}
\item Virtually any\footnote{\emph{Any} because  
      \emph{\productbasename{}-\productversion{}} is a very ordinary
      database application and does not make use of any special
      features of the operating system or \emph{MySQL}.}
      version of a *nix (\emph{Linux}, \emph{FreeBSD}, 
      \emph{Solaris}, etc.).
\item \index{apache@\emph{apache}}\emph{apache}, any modern version.
\item \index{PHP@\emph{PHP}}\emph{PHP}, version 4.X or above.
\item \index{MySQL@\emph{MySQL}}\emph{MySQL}, version 4.X or above.
\item Any sane processor and processor speed.
\item Any sane amount of RAM.
\item Adequate system permissions to inject e-mail from \emph{PHP} via \emph{PHP}'s
      \index{mail()@\emph{mail($\cdot{}$)}}\emph{mail($\cdot{}$)} function.
\item Adequate system permissions to set up a directory, with 
      read/write/create permissions
      for the UID/GID of the \emph{apache} server, to contain the file repository.
      The file repository must not be directly in the logical web space served
      directly by \emph{apache}.
\item Adequate system permissions to set up a \emph{cron} job that runs
      at least once every several minutes and runs under the same UID/GID as the
      \emph{Apache} server.\footnote{Because this \emph{cron} job performs
      some CPU-intensive tasks (such as verifying file signatures of files in the
      file repository), it would violate
      the terms of most shared hosting services.  A dedicated server is
      almost certainly required; and if not that then a server that is not
      too heavily loaded.}
\item Adequate system permissions to set up a location for the PHP library
      that is accessible to the \emph{apache} UID/GID but not in the
      logical web space served directly by \emph{apache}.
\end{itemize}


%%%%%%%%%%%%%%%%%%%%%%%%%%%%%%%%%%%%%%%%%%%%%%%%%%%%%%%%%%%%%%%%%%%%%%%%%%%%%%%
%%%%%%%%%%%%%%%%%%%%%%%%%%%%%%%%%%%%%%%%%%%%%%%%%%%%%%%%%%%%%%%%%%%%%%%%%%%%%%%
%%%%%%%%%%%%%%%%%%%%%%%%%%%%%%%%%%%%%%%%%%%%%%%%%%%%%%%%%%%%%%%%%%%%%%%%%%%%%%%
\section{Installation Checklist}
%Section tag:  ICK0
\label{cist0:sick0}

This section provides an enumerated overview of the steps required to
install \emph{\productbasename{}-\productversion{}}.  The steps are explained
in detail in the indicated sections.

\begin{enumerate}
\item Selection of unpack directory, web root directory, 
      PHP library directory, and file repository directory  
      (\S{}\ref{cist0:sdse0}).
\item Unpacking of \emph{\productbasename{}-\productversion{}} 
      \emph{tar.gz} file  (\S{}\ref{cist0:sutz0}).
\item Customization of \emph{PHP} include path (\S{}\ref{cist0:scpi0}).
\item Creation of site hash key (\S{}\ref{cist0:scsh0}).
\item Creation of \emph{MySQL} database (\S{}\ref{cist0:scmd0}).
\item Setup of \emph{apache} to serve web content (\S{}\ref{cist0:ssap0}).
\item Copying of web content files (\S{}\ref{cist0:swcf0}).
\item Copying of \emph{PHP} library files (\S{}\ref{cist0:scph0}).
\item Initialization of database (\S{}\ref{cist0:sdiz0}).
\item Initial testing (\S{}\ref{cist0:sits0}).
\end{enumerate}


%%%%%%%%%%%%%%%%%%%%%%%%%%%%%%%%%%%%%%%%%%%%%%%%%%%%%%%%%%%%%%%%%%%%%%%%%%%%%%%
%%%%%%%%%%%%%%%%%%%%%%%%%%%%%%%%%%%%%%%%%%%%%%%%%%%%%%%%%%%%%%%%%%%%%%%%%%%%%%%
%%%%%%%%%%%%%%%%%%%%%%%%%%%%%%%%%%%%%%%%%%%%%%%%%%%%%%%%%%%%%%%%%%%%%%%%%%%%%%%
\section{Directory Selection}
%Section tag:  dse0
\label{cist0:sdse0}

TBD.


%%%%%%%%%%%%%%%%%%%%%%%%%%%%%%%%%%%%%%%%%%%%%%%%%%%%%%%%%%%%%%%%%%%%%%%%%%%%%%%
%%%%%%%%%%%%%%%%%%%%%%%%%%%%%%%%%%%%%%%%%%%%%%%%%%%%%%%%%%%%%%%%%%%%%%%%%%%%%%%
%%%%%%%%%%%%%%%%%%%%%%%%%%%%%%%%%%%%%%%%%%%%%%%%%%%%%%%%%%%%%%%%%%%%%%%%%%%%%%%
\section{Unpacking of \emph{tar.gz} File}
%Section tag:  utz0
\label{cist0:sutz0}

TBD.


%%%%%%%%%%%%%%%%%%%%%%%%%%%%%%%%%%%%%%%%%%%%%%%%%%%%%%%%%%%%%%%%%%%%%%%%%%%%%%%
%%%%%%%%%%%%%%%%%%%%%%%%%%%%%%%%%%%%%%%%%%%%%%%%%%%%%%%%%%%%%%%%%%%%%%%%%%%%%%%
%%%%%%%%%%%%%%%%%%%%%%%%%%%%%%%%%%%%%%%%%%%%%%%%%%%%%%%%%%%%%%%%%%%%%%%%%%%%%%%
\section{Customizaton of \emph{PHP} Include Path}
%Section tag:  cpi0
\label{cist0:scpi0}

TBD.


%%%%%%%%%%%%%%%%%%%%%%%%%%%%%%%%%%%%%%%%%%%%%%%%%%%%%%%%%%%%%%%%%%%%%%%%%%%%%%%
%%%%%%%%%%%%%%%%%%%%%%%%%%%%%%%%%%%%%%%%%%%%%%%%%%%%%%%%%%%%%%%%%%%%%%%%%%%%%%%
%%%%%%%%%%%%%%%%%%%%%%%%%%%%%%%%%%%%%%%%%%%%%%%%%%%%%%%%%%%%%%%%%%%%%%%%%%%%%%%
\section{Creation of Site Hash Key}
%Section tag:  csh0
\label{cist0:scsh0}

The site hash key is best created using the
\index{hashkeygen@\emph{hashkeygen}} program.  The specific steps are:

\begin{enumerate}
\item Change directory to the directory containing the script using
      the command ``\texttt{cd cd sw/standalone}'' or similar.
\item Ensure that the file \emph{hashkeygen.php} has the
      ``\texttt{x}'' bit set.  The command ``\texttt{chmod +x hashkeygen.php}''
      will accomplish this in most circumstances.
\item Run the program using the command ``\texttt{./hashkeygen.php}''.
\end{enumerate}
 
\begin{figure}
\begin{footnotesize}
\begin{verbatim}
[dashley@pamc standalone]$ ./hashkeygen.php
The key char set size is 89.
To maintain a purely random distribution, the maximum value of a
random character that can be used is 177.
Target key length is 204 characters.
Open of "/dev/random" was successful.  Will now generate hash key.  This may
take up to several minutes, as the device may block.  Each character from
"/dev/random" that can be used is denoted with a ".", and each character
that cannot be used is denoted with a "/".
.../../../.././/../././....///..../............/././../.....
/////.//./////.....//./././.././/.../.../.../...//.../././/.
/...../..../....../......../.............//...../../..././..
///./....//......//.//....////../..../......../....//.../../
/./././/..//./......./......///..././/......../..../.......
Key generation complete.
\end{verbatim}
\end{footnotesize}
\caption{Typical Output of \emph{hashkeygen}}
\label{fig:cist0:scsh0:00}
\end{figure}

\begin{figure}
\begin{scriptsize}
\begin{verbatim}
<?php
//hashkey.inc -- Definition of hash key for PAMC.
//--------------------------------------------------------------------------------
//This file is automatically generated by the hashkeygen.php program.  Because
//this is a data file that should, for security reasons, be different for each
//deployment of the system, it is not kept under version control.  However, the
//hashkeygen.php program that generated this file has this version control
//information associated with it:
//$Source: /home/dashley/cvsrep/e3ft_gpl01/e3ft_gpl01/webprojs/pamc/gen_a/docs/manual/man_a/c_ist0/c_ist0.tex,v $
//$Revision: 1.9 $
//$Date: 2009/11/04 16:50:19 $
//$Author: dashley $
//$State: Exp $
//--------------------------------------------------------------------------------
$config_hard["hash"]["key"] =   "z)Jckkr?}6UC+GN8A{#VL{&DEdH=[Neu-X u4OONN+<7i)@t"
                              . "BZ_0LoD]8.@aYBrr[D6c(RV(vg3JdDIe^gW1?I2}5-[Imj5h"
                              . ">f{X]19R()i/)&;S1&A3^Wj_-Xjr!Vv(5VR]{ h9bFeWMXD "
                              . "+3@6W+/ _I *4yZ7umMa[o)!!J 43,OJmJBDpaRkzdr.;a2x"
                              . "%tXn&9a!QXa|";
//--------------------------------------------------------------------------------
?>
\end{verbatim}
\end{scriptsize}
\caption{Typical Hash Key Generated by \emph{hashkeygen}}
\label{fig:cist0:scsh0:01}
\end{figure}

Sample typical output of the \emph{hashkeygen} program is shown in
Fig. \ref{fig:cist0:scsh0:00}.  A typical key
generated is shown in 
Fig. \ref{fig:cist0:scsh0:01}.

Note that the \emph{hashkeygen} program writes its output to the file\\\\
``\texttt{../phplib/hash/hashkey.inc}''.\\\\  Later in the installation,
this file will be copied to the final location for the \emph{PHP} library.

It is naturally important that each deployment of
\emph{\productbasename{}-\productversion{}} have a hash key that is
unknown to a potential attacker.  Although the \emph{hashkeygen} program is 
the most effective way to generate a random hash key, the key can also
be created or edited manually (although this is not recommended).


%%%%%%%%%%%%%%%%%%%%%%%%%%%%%%%%%%%%%%%%%%%%%%%%%%%%%%%%%%%%%%%%%%%%%%%%%%%%%%%
%%%%%%%%%%%%%%%%%%%%%%%%%%%%%%%%%%%%%%%%%%%%%%%%%%%%%%%%%%%%%%%%%%%%%%%%%%%%%%%
%%%%%%%%%%%%%%%%%%%%%%%%%%%%%%%%%%%%%%%%%%%%%%%%%%%%%%%%%%%%%%%%%%%%%%%%%%%%%%%
\section{Creation of \emph{MySQL} Database}
%Section tag:  cmd0
\label{cist0:scmd0}

Setup of 
\index{MySQL@\emph{MySQL}!Setup for \productbasename{}-\productversion{}@Setup for \emph{\productbasename{}-\productversion{}}}%
\emph{MySQL} involves obtaining a database name,
userid, and password.  (This is the only information
required to set up \emph{\productbasename{}}---creation of
database tables is handled by a script.)

The steps to set up \emph{MySQL} depend on how the software 
is hosted.

\begin{itemize}
\item If the software is hosted by a hosting company, the 
      \emph{MySQL} database name, userid, and password will probably
      be assigned by the hosting company.
\item If the software is hosted on an owned or dedicated server,
      the setup must be performed by the individual
      installing \emph{\productbasename{}}.
\end{itemize}

If the software is hosted on an owned or dedicated server,
the following steps should be used to set up \emph{MySQL}:

\begin{enumerate}
\item Choose a database name, userid, and password
      for use with \emph{MySQL}.  In subsequent description, these
      are denoted \emph{dbname}, \emph{userid},
      and \emph{password}.
\item Log into \emph{MySQL} as the root user.\footnote{Note that the
      \emph{root} password for \emph{MySQL} is not the same
      thing as the \emph{root} user password for \emph{Linux}.}
      The command to do this is:

      \texttt{mysql --user=root -p}
\item Create the database.  The \emph{MySQL} command to do this is:

      \texttt{create database \emph{dbname};}
\item Grant the user \emph{userid} all privileges on database
      \emph{dbname} using password \emph{password} when connecting
      from \emph{localhost}.\footnote{The normal arrangement is that the
      \emph{MySQL} daemon runs on the same server as \emph{Apache}, hence
      the connection from \emph{localhost}.}  The command to do this is:

      \texttt{grant all on \emph{dbname}.* to \emph{userid}@localhost\\identified by '\emph{password}';}
\item Log out of \emph{MySQL} (Control-D).
\item Test the permissions created by running

      \texttt{mysql --user=\emph{userid} -p}

      and entering the \emph{password} chosen.  Issue the command:

      \texttt{use \emph{dbname};}

      to verify permission to access \emph{dbname}.
\item Log out of \emph{MySQL} (Control-D).
\end{enumerate}


%%%%%%%%%%%%%%%%%%%%%%%%%%%%%%%%%%%%%%%%%%%%%%%%%%%%%%%%%%%%%%%%%%%%%%%%%%%%%%%
%%%%%%%%%%%%%%%%%%%%%%%%%%%%%%%%%%%%%%%%%%%%%%%%%%%%%%%%%%%%%%%%%%%%%%%%%%%%%%%
%%%%%%%%%%%%%%%%%%%%%%%%%%%%%%%%%%%%%%%%%%%%%%%%%%%%%%%%%%%%%%%%%%%%%%%%%%%%%%%
\section{Creation of Network Interface Aliases (Some Installations Only)}
%Section tag:  cna0
\label{cist0:scna0}

For greater security, \emph{\productbasename{}-\productversion{}} may be 
served via \emph{https} rather than \emph{http}.  Because
each domain served by \emph{https} must have its own IP address, in some
installations additional IP addresses will need to be bound to the same
network interface.

The procedure for assigning additional IP addresses to a network
interface involves creating an additional file in the
\texttt{/etc/sysconfig/network-scripts} directory.
The most common scenario is to create a file with a \emph{:0} suffix.
The files below illustrate adding the IP address 208.81.180.179 to
an interface already bound to the IP address 208.81.180.178.

\begin{small}
\begin{verbatim}
[dashley@pamc ~]$ cat /etc/sysconfig/network-scripts/ifcfg-eth0
# Broadcom Corporation NetXtreme BCM5722 Gigabit Ethernet PCI Express
DEVICE=eth0
BOOTPROTO=none
BROADCAST=208.81.180.255
HWADDR=00:1e:c9:51:a6:b9
IPADDR=208.81.180.178
NETMASK=255.255.255.128
NETWORK=208.81.180.128
ONBOOT=yes
GATEWAY=208.81.180.129
TYPE=Ethernet
[dashley@pamc ~]$ cat /etc/sysconfig/network-scripts/ifcfg-eth0:0
# Broadcom Corporation NetXtreme BCM5722 Gigabit Ethernet PCI Express
DEVICE=eth0:0
BOOTPROTO=none
BROADCAST=208.81.180.255
HWADDR=00:1e:c9:51:a6:b9
IPADDR=208.81.180.179
NETMASK=255.255.255.128
NETWORK=208.81.180.128
ONBOOT=yes
GATEWAY=208.81.180.129
TYPE=Ethernet
\end{verbatim}
\end{small}

Once the additional file is created, the \texttt{ifup} command can
used to activate the interface without rebooting the system, i.e.
\texttt{ifup eth0:0}.  When the system is rebooted, the interface will
be activated automatically if \texttt{ONBOOT=yes} is specified.

The network to which the server is connected must be configured to
accept the additional IP addresses.  More information can be found
in various \emph{Linux} networking tutorials on the Internet.


%%%%%%%%%%%%%%%%%%%%%%%%%%%%%%%%%%%%%%%%%%%%%%%%%%%%%%%%%%%%%%%%%%%%%%%%%%%%%%%
%%%%%%%%%%%%%%%%%%%%%%%%%%%%%%%%%%%%%%%%%%%%%%%%%%%%%%%%%%%%%%%%%%%%%%%%%%%%%%%
%%%%%%%%%%%%%%%%%%%%%%%%%%%%%%%%%%%%%%%%%%%%%%%%%%%%%%%%%%%%%%%%%%%%%%%%%%%%%%%
\section{Creation of Multiple Instances of \emph{apache}}
%Section tag:  cmi0
\label{cist0:scmi0}

A single instance of \emph{apache}, running under a single UID/GID,
can be configured to listen on multiple IP addresses and
serve multiple domains via \emph{https}.  In some server deployments,
this will work well.

However, in some server deployments, it is desirable to serve multiple
domains from the same server, and using a single instance
of \emph{apache} may raise security issues if not
all of the web scripts are under the control of the same individual
or organization.  It would be possible for one author of web content
to write a script that compromises private files of another author---files
containing hash keys, cryptographic keys, or database passwords, for example.

Running multiple instances of \emph{apache}, each running under a different
UID/GID and listening on a different IP address or port, can alleviate
security concerns.  For example, in some server deployments it would be
possible to run \emph{\productbasename{}-\productversion{}} using a second
instance of \emph{apache} and a separate UID/GID, thus securing it against
attacks launched from the UID/GID of other instance(s).

A naming schema should be chosen for the multiple 
instances of \emph{apache}.  One naming schema would be to designate
the IP addresses as \emph{a}, \emph{b}, etc. so that the
instance of \emph{apache} listening on port 80 on the first interface
would be named \emph{httpd80a}.

The startup scripts in \texttt{/etc/rc.d/init.d} should be copied and modified
so that there is one startup script per instance of \emph{apache},
appropriately named to coincide with the naming schema chosen.
The difference listing below indicates how to modify each startup
script.  Note that some modifications (the first ones in the listing)
are to comments and are unnecessary.

\begin{small}
\begin{verbatim}
[dashley@pamc ~]$ diff /etc/rc.d/init.d/httpd /etc/rc.d/init.d/httpd80a
3c3
< # httpd        Startup script for the Apache HTTP Server
---
> # httpd80a     Startup script for the Apache HTTP Server
8,11c8,11
< # processname: httpd
< # config: /etc/httpd/conf/httpd.conf
< # config: /etc/sysconfig/httpd
< # pidfile: /var/run/httpd.pid
---
> # processname: httpd80a
> # config: /etc/httpd/conf/httpd80a.conf
> # config: /etc/sysconfig/httpd80a
> # pidfile: /var/run/httpd80a.pid
16,17c16,17
< if [ -f /etc/sysconfig/httpd ]; then
<         . /etc/sysconfig/httpd
---
> if [ -f /etc/sysconfig/httpd80a ]; then
>         . /etc/sysconfig/httpd80a
33,36c33,36
< httpd=${HTTPD-/usr/sbin/httpd}
< prog=httpd
< pidfile=${PIDFILE-/var/run/httpd.pid}
< lockfile=${LOCKFILE-/var/lock/subsys/httpd}
---
> httpd=${HTTPD-/usr/sbin/httpd80a}
> prog=httpd80a
> pidfile=${PIDFILE-/var/run/httpd80a.pid}
> lockfile=${LOCKFILE-/var/lock/subsys/httpd80a}
41c41
<       CONFFILE=/etc/httpd/conf/httpd.conf
---
>       CONFFILE=/etc/httpd/conf/httpd80a.conf
\end{verbatim}
\end{small}

The executable files in \texttt{/sbin} should 
be copied so that \texttt{httpd}, \texttt{httpd.worker}, and
\texttt{httpd.event} each have appropriately named copies corresponding
to the naming schema chosen.  The listing below shows the files
in a typical server.

\begin{small}
\begin{verbatim}
[dashley@pamc ~]$ ls -al /usr/sbin/http*
-rwxr-xr-x 1 root root 315284 Jul 15 09:04 /usr/sbin/httpd
-rwxr-xr-x 1 root root 315284 Oct  4 01:29 /usr/sbin/httpd443a
-rwxr-xr-x 1 root root 327708 Oct  4 01:29 /usr/sbin/httpd443a.event
-rwxr-xr-x 1 root root 327708 Oct  4 01:30 /usr/sbin/httpd443a.worker
-rwxr-xr-x 1 root root 315284 Nov  1 23:20 /usr/sbin/httpd443b
-rwxr-xr-x 1 root root 327708 Nov  1 23:20 /usr/sbin/httpd443b.event
-rwxr-xr-x 1 root root 327708 Nov  1 23:20 /usr/sbin/httpd443b.worker
-rwxr-xr-x 1 root root 315284 Oct  4 01:29 /usr/sbin/httpd80a
-rwxr-xr-x 1 root root 327708 Oct  4 01:29 /usr/sbin/httpd80a.event
-rwxr-xr-x 1 root root 327708 Oct  4 01:30 /usr/sbin/httpd80a.worker
-rwxr-xr-x 1 root root 315284 Nov  1 23:19 /usr/sbin/httpd80b
-rwxr-xr-x 1 root root 327708 Nov  1 23:20 /usr/sbin/httpd80b.event
-rwxr-xr-x 1 root root 327708 Nov  1 23:20 /usr/sbin/httpd80b.worker
-rwxr-xr-x 1 root root 327708 Jul 15 09:04 /usr/sbin/httpd.event
-rwxr-xr-x 1 root root 327708 Jul 15 09:04 /usr/sbin/httpd.worker
\end{verbatim}
\end{small}

The runlevel links can then be modified.  Need to add information about this.

\emph{Dave Ashley note:}
When attempting to use four instances of \emph{apache} to listen on two
IP addresses, ran into an issue with port binding to 0:0:0:0.  Need to
resolve this issue definitively.  For now, am using two instances of
\emph{apache}, one listening on two IP addresses on port 80, and the other
listening on two IP addresses on port 443.  I should be able, however, to use
four instances.


%%%%%%%%%%%%%%%%%%%%%%%%%%%%%%%%%%%%%%%%%%%%%%%%%%%%%%%%%%%%%%%%%%%%%%%%%%%%%%%
%%%%%%%%%%%%%%%%%%%%%%%%%%%%%%%%%%%%%%%%%%%%%%%%%%%%%%%%%%%%%%%%%%%%%%%%%%%%%%%
%%%%%%%%%%%%%%%%%%%%%%%%%%%%%%%%%%%%%%%%%%%%%%%%%%%%%%%%%%%%%%%%%%%%%%%%%%%%%%%
\section{Generation of an SSL Certificate for \emph{apache}}
%Section tag:  gsl0
\label{cist0:sgsl0}

An SSL certificate allows a browser (when using \emph{https}) to verify that
the site connected to is the actual site rather the result of intercepted
transmission.

An SSL certificate is required to serve \emph{\productbasename{}-\productversion{}}
via \emph{https}.

There are two types of SSL certificates that may used:

\begin{itemize}
\item \textbf{A purchased certificate (\S\ref{cist0:sgsl0:spsl0})\@.}
      A purchased certificate typically costs around \$30 (for a 1-year
      certificate), but is traceable to a certification authority already
      accepted by browsers and so introduces no complexity in configuring
      a browser to accept the certificate.
\item \textbf{A self-signed certificate (\S\ref{cist0:sgsl0:sgss0})\@.}
      A self-signed certificate is free, but introduces complexity in
      configuring a browser to accept the certificate without nags or
      perhaps to accept the certificate at all.
\end{itemize}


%%%%%%%%%%%%%%%%%%%%%%%%%%%%%%%%%%%%%%%%%%%%%%%%%%%%%%%%%%%%%%%%%%%%%%%%%%%%%%%
%%%%%%%%%%%%%%%%%%%%%%%%%%%%%%%%%%%%%%%%%%%%%%%%%%%%%%%%%%%%%%%%%%%%%%%%%%%%%%%
%%%%%%%%%%%%%%%%%%%%%%%%%%%%%%%%%%%%%%%%%%%%%%%%%%%%%%%%%%%%%%%%%%%%%%%%%%%%%%%
\subsection{Purchase of an SSL Certificate}
%Subsection tag:  psl0
\label{cist0:sgsl0:spsl0}

TBD.


%%%%%%%%%%%%%%%%%%%%%%%%%%%%%%%%%%%%%%%%%%%%%%%%%%%%%%%%%%%%%%%%%%%%%%%%%%%%%%%
%%%%%%%%%%%%%%%%%%%%%%%%%%%%%%%%%%%%%%%%%%%%%%%%%%%%%%%%%%%%%%%%%%%%%%%%%%%%%%%
%%%%%%%%%%%%%%%%%%%%%%%%%%%%%%%%%%%%%%%%%%%%%%%%%%%%%%%%%%%%%%%%%%%%%%%%%%%%%%%
\subsection{Generating a Self-Signed SSL Certificate}
%Subsection tag:  gss0
\label{cist0:sgsl0:sgss0}

TBD.


%%%%%%%%%%%%%%%%%%%%%%%%%%%%%%%%%%%%%%%%%%%%%%%%%%%%%%%%%%%%%%%%%%%%%%%%%%%%%%%
%%%%%%%%%%%%%%%%%%%%%%%%%%%%%%%%%%%%%%%%%%%%%%%%%%%%%%%%%%%%%%%%%%%%%%%%%%%%%%%
%%%%%%%%%%%%%%%%%%%%%%%%%%%%%%%%%%%%%%%%%%%%%%%%%%%%%%%%%%%%%%%%%%%%%%%%%%%%%%%
\section{Setup of \emph{apache} to Serve Web Content}
%Section tag:  sap0
\label{cist0:ssap0}

TBD.


%%%%%%%%%%%%%%%%%%%%%%%%%%%%%%%%%%%%%%%%%%%%%%%%%%%%%%%%%%%%%%%%%%%%%%%%%%%%%%%
%%%%%%%%%%%%%%%%%%%%%%%%%%%%%%%%%%%%%%%%%%%%%%%%%%%%%%%%%%%%%%%%%%%%%%%%%%%%%%%
%%%%%%%%%%%%%%%%%%%%%%%%%%%%%%%%%%%%%%%%%%%%%%%%%%%%%%%%%%%%%%%%%%%%%%%%%%%%%%%
\section{Installation of the \emph{cc\_kt1\_auth\_php} Program}
%Section tag:  ipg0
\label{cist0:sipg0}

The \emph{cc\_kt1\_auth\_php} program is called from \emph{PHP} scripts to
authenticate the \emph{CryptoCard} KT-1 token.

The \emph{cc\_kt1\_auth\_php} program operates in the following way:

\begin{itemize}
\item A \emph{PHP} script invokes the \emph{cc\_kt1\_auth\_php} program,
      opening two pipes\footnote{Pipes (more precisely, anonymous pipes) are used because a pipe provides
      secure communication between processes.  Passing sensitive information
      (such as token keys) as a command-line parameter is not secure, as
      command-line parameters are world-visible on a \emph{Linux} system.}
      to communicate bidirectionally with the program.
\item The \emph{cc\_kt1\_auth\_php} accepts all of the data provided
      by the \emph{PHP} script via a pipe.  The data includes
      a token key, token state, and other parameters.
\item The \emph{cc\_kt1\_auth\_php} calls a library provided by 
      \emph{CryptoCard} to predict what a token should display.
\item The \emph{cc\_kt1\_auth\_php} returns this information to 
      the \emph{PHP} script via a pipe.
\item The \emph{cc\_kt1\_auth\_php} terminates.
\item The \emph{PHP} script uses the information provided by the
      \emph{cc\_kt1\_auth\_php} program to authenticate a token.
\end{itemize}

The \emph{cc\_kt1\_auth\_php} can be installed using the following steps:

\begin{enumerate}
\item Obtain the \emph{AuthEngine SDK} product from \emph{CryptoCard}.
\item Install the shared libraries (\texttt{libAuthentication.so}
      \texttt{libAuthentication.a}) in the recommended location for
      the target system,\footnote{On a standard \emph{Linux} system,
      the appropriate location is \texttt{/usr/lib}.}
      and set ownership and permissions appropriately.
\item Place the program file (\texttt{cc\_kt1\_auth\_php.c}) and
      the header file from \emph{CryptoCard} (\texttt{Authentication.h})
      in a directory for compilation.
\item Compile the program using the instructions contained in the source
      code.  The source code also contains a description of steps to
      take if \texttt{libcrypto.so.4} is missing.
\item Copy the executable to a location suitable for the target system and
      set ownership and permissions appropriately (this is described
      in the source code).
\end{enumerate}


%%%%%%%%%%%%%%%%%%%%%%%%%%%%%%%%%%%%%%%%%%%%%%%%%%%%%%%%%%%%%%%%%%%%%%%%%%%%%%%
%%%%%%%%%%%%%%%%%%%%%%%%%%%%%%%%%%%%%%%%%%%%%%%%%%%%%%%%%%%%%%%%%%%%%%%%%%%%%%%
%%%%%%%%%%%%%%%%%%%%%%%%%%%%%%%%%%%%%%%%%%%%%%%%%%%%%%%%%%%%%%%%%%%%%%%%%%%%%%%
\section{Copying of \emph{PHP} Web Content Files}
%Section tag:  wcf0
\label{cist0:swcf0}

TBD.


%%%%%%%%%%%%%%%%%%%%%%%%%%%%%%%%%%%%%%%%%%%%%%%%%%%%%%%%%%%%%%%%%%%%%%%%%%%%%%%
%%%%%%%%%%%%%%%%%%%%%%%%%%%%%%%%%%%%%%%%%%%%%%%%%%%%%%%%%%%%%%%%%%%%%%%%%%%%%%%
%%%%%%%%%%%%%%%%%%%%%%%%%%%%%%%%%%%%%%%%%%%%%%%%%%%%%%%%%%%%%%%%%%%%%%%%%%%%%%%
\section{Copying of \emph{PHP} Library Files}
%Section tag:  CPH0
\label{cist0:scph0}

TBD.


%%%%%%%%%%%%%%%%%%%%%%%%%%%%%%%%%%%%%%%%%%%%%%%%%%%%%%%%%%%%%%%%%%%%%%%%%%%%%%%
%%%%%%%%%%%%%%%%%%%%%%%%%%%%%%%%%%%%%%%%%%%%%%%%%%%%%%%%%%%%%%%%%%%%%%%%%%%%%%%
%%%%%%%%%%%%%%%%%%%%%%%%%%%%%%%%%%%%%%%%%%%%%%%%%%%%%%%%%%%%%%%%%%%%%%%%%%%%%%%
\section{Database Initialization}
%Section tag:  DIZ0
\label{cist0:sdiz0}

TBD.


%%%%%%%%%%%%%%%%%%%%%%%%%%%%%%%%%%%%%%%%%%%%%%%%%%%%%%%%%%%%%%%%%%%%%%%%%%%%%%%
%%%%%%%%%%%%%%%%%%%%%%%%%%%%%%%%%%%%%%%%%%%%%%%%%%%%%%%%%%%%%%%%%%%%%%%%%%%%%%%
%%%%%%%%%%%%%%%%%%%%%%%%%%%%%%%%%%%%%%%%%%%%%%%%%%%%%%%%%%%%%%%%%%%%%%%%%%%%%%%
\section{Initial Testing}
%Section tag:  ITS0
\label{cist0:sits0}

TBD.


%%%%%%%%%%%%%%%%%%%%%%%%%%%%%%%%%%%%%%%%%%%%%%%%%%%%%%%%%%%%%%%%%%%%%%%%%%
\noindent\begin{figure}[!b]
\noindent\rule[-0.25in]{\textwidth}{1pt}
\begin{tiny}
\begin{verbatim}
$RCSfile: c_ist0.tex,v $
$Source: /home/dashley/cvsrep/e3ft_gpl01/e3ft_gpl01/webprojs/pamc/gen_a/docs/manual/man_a/c_ist0/c_ist0.tex,v $
$Revision: 1.9 $
$Author: dashley $
$Date: 2009/11/04 16:50:19 $
\end{verbatim}
\end{tiny}
\noindent\rule[0.25in]{\textwidth}{1pt}
\end{figure}

%%%%%%%%%%%%%%%%%%%%%%%%%%%%%%%%%%%%%%%%%%%%%%%%%%%%%%%%%%%%%%%%%%%%%%%%%%%%%%%
%$Log: c_ist0.tex,v $
%Revision 1.9  2009/11/04 16:50:19  dashley
%Edits.
%
%Revision 1.8  2009/11/02 04:53:28  dashley
%Edits.
%
%Revision 1.7  2009/11/02 02:00:04  dashley
%Edits.
%
%Revision 1.6  2007/06/24 21:19:24  dashley
%Minor extra word (that won't work) for MySQL command removed.
%
%Revision 1.5  2007/06/12 02:47:17  dashley
%Edits.
%
%Revision 1.4  2007/06/10 18:03:20  dashley
%Edits.
%
%Revision 1.3  2007/06/06 02:23:58  dashley
%Edits.
%
%Revision 1.2  2007/06/04 03:26:55  dashley
%Edits.
%
%Revision 1.1  2007/06/04 00:12:03  dashley
%Initial checkin.
%
%End of $RCSfile: c_ist0.tex,v $.
