%$Header: /cvsroot/esrg/sfesrg/esrgpcpj/hyreach/doc/hyreachm/s_inv0/s_inv0.tex,v 1.6 2002/01/30 00:51:04 dtashley Exp $
%
\section{Invocation Of The Software}
\label{sinv0}

\swname{} is an ordinary Win32 console application, produced using Microsoft 
Visual C++ Version 6.0.  When \swname{} is invoked, it undertakes the following actions:

\begin{itemize}
\item \index{command-line parameters}Command-line 
      parameters are parsed.  Fatal errors are possible if 
      the command-line parameters are ill-formatted.
\item The input file (which specifies the model to verify) is parsed.  
      Fatal errors are possible if the model file is ill-formatted.
\item 
      \index{model design rules}Design-rule 
      checks are applied to the model specified.
      Fatal errors are possible if the model specified violates the design 
      rules.
\item A reachability algorithm is applied to determine if any of the configurations
      marked as illegal in the model are reachable.  When invoked and
      unless stopped by the user, \swname{} will do 
      one of the following:
      
      \begin{itemize}
      \item Verify that no illegal state is reachable.
      \item Determine that an illegal configuration is reachable, and indicate how, 
            starting from the initial configurations, an illegal configuration can be reached.
      \item Terminate abnormally without a reachability determination because 
            not enough computer resources (memory) are available to perform 
            the verification.
      \end{itemize}
\end{itemize}

Once invoked, \swname{} can be terminated before it completes normally in
two ways:

\begin{itemize}
\item Using the CTRL-C or CTRL-BREAK key combinations while the console
      window in which \swname{} is running has the focus.
\item Using the CTRL-ALT-DELETE key combination to obtain a list of applications
      or processes, and then electing to terminte \swname{}.
\end{itemize}

Of the two termination alternatives, the first is preferred, as it works using
the 'C' runtime library and is a gentler type of termination.


\subsection{Invocation Format}
%Subsection Tag: SINF0
\label{sinv0:sinf0}

The general invocation format is: \\

\texttt{hyreach option\_1 option\_2 \ldots option\_N} \\

\noindent{}where the options are any of the options 
described in Section \ref{sinv0:sclo0}.  Note that 
one of the options (the \texttt{-mf}, or \emph{model file} 
option) is not truly optional, as \swname{} will not 
do any useful work without it.

Note that each option consists of an option flag
(``\texttt{-mf}'', for example) followed by zero
or more option parameters.  The number of required
option parameters is fixed for each option.

Options may always appear in any order, but may not
appear twice.


%%%%%%%%%%%%%%%%%%%%%%%%%%%%%%%%%%%%%%%%%%%%%%%%%%%%%%%%%%%%%%%%%%%%%%%%%%%%%%%
%%%%%%%%%%%%%%%%%%%%%%%%%%%%%%%%%%%%%%%%%%%%%%%%%%%%%%%%%%%%%%%%%%%%%%%%%%%%%%%
%%%%%%%%%%%%%%%%%%%%%%%%%%%%%%%%%%%%%%%%%%%%%%%%%%%%%%%%%%%%%%%%%%%%%%%%%%%%%%%
\subsection{Command-Line Options}
%Subsection Tag: SCLO0
\label{sinv0:sclo0}

This section describes the command-line options that may be used to modify the 
behavior of \swname{}.

\subsubsection{-mf Command-Line Option (1 Option Parameter)}
The \index{-mf@\texttt{-mf}}\texttt{-mf} (``model file'') specifies to 
\swname{} the name of the model file to be used (the format of
the model file is described in Section \ref{siff0}).
\swname{} performs no processing or interpretation of the string
supplied, so the filename specified may be any filename accepted
by the underlying operating system.

If the model file name specified contains spaces, the name must be
surrounded by quotes.\footnote{Strictly speaking, this isn't always true.
The issue is how \swname{} receives its arguments (the \texttt{argv[]} 
array).  If a file name containing spaces is specified from within a DOS
box, the quotes prevent the DOS command interpreter from interpreting the
file name as multiple parameters and breaking it up.  However, if a
scripting language such a Tcl is used, it may not be necessary to quote
such a file name (and in fact it may cause errors) because the 
spawning mechanisms (the \texttt{exec} command in Tcl for example) 
bypass the DOS command interpeter.}

\subsubsection{-ch Command-Line Option (No Option Parameters)}
The \index{-ch@\texttt{-ch}}\texttt{-ch} (``convex hull'') command-line 
option causes \swname{} to transform unions of convex zones to
the minimal convex zone (the \index{convex hull}convex hull) 
that will enclose the union.  
(Note that the convex hull is at least as large as union.)
Since these 
unions are used to represent the subset of $\mathbb{R}^N$ that 
clock variables may be in, any states that are reachable using the exact
representation (the union) are also reachable using the convex hull 
approximation; but the converse is not true.  Thus:

\begin{itemize}
\item A result of ``no illegal configurations reachable'' when the 
      \texttt{-ch} option is used implies that the configurations will also
      not be reachable when the more exact representation is used.
\item A result of ``illegal configurations reachable'' when the 
      \texttt{-ch} option is used does not necessarily imply that 
      illegal configurations will be reachable when the more exact
      representation is used.
\end{itemize}

Thus, with respect to the safeness of systems, the \texttt{-ch}
option cannot lead to a false determination of \emph{safe}, but
can lead to a false determination of \emph{unsafe}.

\subsubsection{-q Command-Line Option (No Option Parameters)}
The \index{-q@\texttt{-q}}\texttt{-q} (``quiet'') command-line option causes \swname{}
to produce a minimal amount of output.  This option is approximately opposite
to the \texttt{-d} (debug level) and \texttt{-v} (verbose)
command-line options.

\subsubsection{-v Command-Line Option (No Option Parameters)}
The \index{-v@\texttt{-v}}\texttt{-v} (``verbose'') command-line option causes 
\swname{} to produce more than the normal 
amount of output.  The output is of an informational nature, and not 
designed to assist in debugging.

\subsubsection{-d Command-Line Option (2 Option Parameters)}
The \index{-d@\texttt{-d}}\texttt{-d <debug\_level> <debug\_file\_name>} (``debug'') 
command-line option causes 
\swname{} to produce information 
useful in debugging.  The \texttt{debug\_level}
integral parameter to this option 
must be one of the numerical parameters described in Table 
\ref{tbl:sinv0:sclo0:01}.

\begin{table}
\begin{center}
\begin{tabular}{|c|l|}
\hline
\texttt{-d} Option Value & Interpretation \\
\hline
\hline
0 (default) & No debugging information is produced.  Although a              \\
            & file name must be specified because of the syntax              \\
            & of the option, no debug file will be produced.                 \\
\hline
1           & Information is provided to trace what the program              \\
            & is doing and when.  Certain abstract information,              \\
            & such as size or complexity measures for data                   \\
            & structures or timing information will be produced.             \\
\hline
2           & Abstract data structure information will be                    \\
            & produced.  Generally by \emph{abstract} we mean that           \\
            & the interpretation of data structures (the                     \\
            & represented polytopes, for example) will be                    \\
            & produced, but the implementation details of data               \\
            & structures (pointer values, bitfield values, etc)              \\
            & will not be produced.                                          \\
\hline
3           & Detailed data structure information (including                 \\
            & implementation details) will be produced.                      \\
\hline
4           & All available information will be produced.                    \\
\hline
\end{tabular}
\end{center}
\caption{\texttt{-d} Option Parameters And Their Interpretation}
\label{tbl:sinv0:sclo0:01}
\end{table}

The \texttt{debug\_file\_name} parameter must be a file that can be
opened by the operating system.  At the higher debug levels, operation
of \swname{} will be slowed substantially, and the debug information
file may become quite large.  Note also that \swname{} can be compiled
without debug support (for the sake of speed), in which case 
the \texttt{-d} option will not be allowed.

%%%%%%%%%%%%%%%%%%%%%%%%%%%%%%%%%%%%%%%%%%%%%%%%%%%%%%%%%%%%%%%%%%%%%%%%%%
\noindent\begin{figure}[!b]
\noindent\rule[-0.25in]{\textwidth}{1pt}
\begin{tiny}
\begin{verbatim}
$RCSfile: s_inv0.tex,v $
$Source: /cvsroot/esrg/sfesrg/esrgpcpj/hyreach/doc/hyreachm/s_inv0/s_inv0.tex,v $
$Revision: 1.6 $
$Author: dtashley $
$Date: 2002/01/30 00:51:04 $
\end{verbatim}
\end{tiny}
\noindent\rule[0.25in]{\textwidth}{1pt}
\end{figure}
%%%%%%%%%%%%%%%%%%%%%%%%%%%%%%%%%%%%%%%%%%%%%%%%%%%%%%%%%%%%%%%%%%%%%%%%%%
%$Log: s_inv0.tex,v $
%Revision 1.6  2002/01/30 00:51:04  dtashley
%Nightly safety checkin.
%
%Revision 1.5  2002/01/29 17:04:00  dtashley
%Version control info added, and minor edits.
%
%Revision 1.4  2001/10/12 22:32:30  dtashley
%Substantial edits.
%
%Revision 1.3  2001/10/10 02:09:07  dtashley
%Edits.
%
%Revision 1.2  2001/09/26 04:51:13  dtashley
%Edits.
%
%Revision 1.1  2001/09/26 02:31:14  dtashley
%Initial checkin, and some edits of main TEX file.
%
%End of S_INV0.TEX.
