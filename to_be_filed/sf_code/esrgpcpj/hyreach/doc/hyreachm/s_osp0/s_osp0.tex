%$Header: /cvsroot/esrg/sfesrg/esrgpcpj/hyreach/doc/hyreachm/s_osp0/s_osp0.tex,v 1.6 2002/01/29 17:04:00 dtashley Exp $
%
\section{Overview, Scope, And Purpose}
\label{sosp0}

This document describes the \swname{} software, Version \swversion{}.  
\swname{} is a software 
program to determine the reachability of certain states of timed automata models 
under a very restricted modeling framework.  The software was specifically crafted 
to solve the interrupt latency compatibility problem for practical embedded systems.

The mnemonic behind the name \swname{} is \emph{hy}brid system 
\emph{reach}ability.  \swname{} is not 
related in any way to the hybrid system tool from Stanford, \emph{HyTech}.

Although the source code for \swname{} is very portable (primarily because it does 
not include a graphical interface), at the present time only an executable for Windows 
platforms is provided.

Section \ref{sinv0} describes the invocation of the software, 
including command-line parameters 
and output.

Section \ref{smfs0} describes the modeling framework supported. 
To achieve good performance on 
models of practical size, a very restricted modeling framework is supported.

Section \ref{siff0} describes the input file format required by \swname{}.  
The input file is a 
text file which specifies the system to be analyzed for reachability.

Section \ref{srae0} supplies the details of the reachability algorithm employed.

Section \ref{ssym0} describes the modeling of practical systems in idioms 
that \swname{} can accept.

Section \ref{siop0} describes the design of \swname{}---primarily 
the optimizations for speed.

Section \ref{spbd0} describes the benchmarking of \swname{}.  \swname{} is benchmarked 
against UPPAAL (a verification tool with a similar philosophy but with far more 
verification capability).  The verification time of \swname{} is extrapolated with 
respect to the number of automatons and their number of states.

Section \ref{sack0} contains acknowledgements to the organizations and individuals
who have assisted us in this work.

Section \ref{sglo0} provides a glossary of terms, since the nomenclature we use
may in some cases be non-standard.

Section \ref{sglm0} provides a glossary of mathematical notation and symbols.

An index is also provided at the end of the document.


%%%%%%%%%%%%%%%%%%%%%%%%%%%%%%%%%%%%%%%%%%%%%%%%%%%%%%%%%%%%%%%%%%%%%%%%%%
\noindent\begin{figure}[!b]
\noindent\rule[-0.25in]{\textwidth}{1pt}
\begin{tiny}
\begin{verbatim}
$RCSfile: s_osp0.tex,v $
$Source: /cvsroot/esrg/sfesrg/esrgpcpj/hyreach/doc/hyreachm/s_osp0/s_osp0.tex,v $
$Revision: 1.6 $
$Author: dtashley $
$Date: 2002/01/29 17:04:00 $
\end{verbatim}
\end{tiny}
\noindent\rule[0.25in]{\textwidth}{1pt}
\end{figure}
%%%%%%%%%%%%%%%%%%%%%%%%%%%%%%%%%%%%%%%%%%%%%%%%%%%%%%%%%%%%%%%%%%%%%%%%%%
%$Log: s_osp0.tex,v $
%Revision 1.6  2002/01/29 17:04:00  dtashley
%Version control info added, and minor edits.
%
%Revision 1.5  2001/10/31 04:31:37  dtashley
%Nightly safety checkin.
%
%Revision 1.4  2001/10/12 22:32:30  dtashley
%Substantial edits.
%
%Revision 1.3  2001/10/10 02:09:07  dtashley
%Edits.
%
%Revision 1.2  2001/09/26 04:51:13  dtashley
%Edits.
%
%Revision 1.1  2001/09/26 02:31:14  dtashley
%Initial checkin, and some edits of main TEX file.
%
%End of S_OSP0.TEX.
