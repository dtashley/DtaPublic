%$Header: /cvsroot/esrg/sfesrg/esrgpcpj/doc/engman01/engman01.tex,v 1.5 2002/06/26 01:01:49 dtashley Exp $
%
\documentclass[letterpaper,10pt,titlepage]{custbook}
\pagestyle{headings}
\usepackage{amsmath}
\usepackage{amsfonts}
\usepackage{amssymb}
\usepackage[ansinew]{inputenc}
\usepackage[OT1]{fontenc}
\usepackage{makeidx}
\usepackage{graphicx}
%

%Standard preamable.
\makeindex
%
%User-defined environments.
%$Header: /cvsroot/esrg/sfesrg/esrgpcpj/hyreach/doc/hyreachm/general/newenvs.tex,v 1.1 2001/10/01 05:58:28 dtashley Exp $
%
%%%%%%%%%%%%%%%%%%%%%%%%%%%%%%%%%%%%%%%%%%%%%%%%%%%%%%%%%%%%%%%%%%%%%%%%%%%%%
%% GLOSSARY OF TERMS
%%%%%%%%%%%%%%%%%%%%%%%%%%%%%%%%%%%%%%%%%%%%%%%%%%%%%%%%%%%%%%%%%%%%%%%%%%%%%
%
%The following environment is for the glossary of terms.
\newenvironment{glossaryenum}{\begin{list}
               {}{\setlength{\labelwidth}{0mm}
                  \setlength{\leftmargin}{4mm}
                  \setlength{\itemindent}{-4mm}
                  \setlength{\parsep}{0.85mm}}}
               {\end{list}}
%
%%%%%%%%%%%%%%%%%%%%%%%%%%%%%%%%%%%%%%%%%%%%%%%%%%%%%%%%%%%%%%%%%%%%%%%%%%%%%%%%%%%%%%%%%%
%%%%%%%%%%%%%%%%%%%%%%%%%%%%%%%%%%%%%%%%%%%%%%%%%%%%%%%%%%%%%%%%%%%%%%%%%%%%%%%%%%%%%%%%%%
%%%%%%%%%%%%%%%%%%%%%%%%%%%%%%%%%%%%%%%%%%%%%%%%%%%%%%%%%%%%%%%%%%%%%%%%%%%%%%%%%%%%%%%%%%
%$Log: newenvs.tex,v $
%Revision 1.1  2001/10/01 05:58:28  dtashley
%Initial checkin.
%
%End of file NEWENVS.TEX


\begin{document}

%The name of the tool set is still in a state of flux.  Let's make this
%a command.
\newcommand{\tsname}{ESRG Tool Set}

%The version of the tool set may change.  Let's localize that.
\newcommand{\tsversion}{1.06}

%The title of the companion book.
\newcommand{\compbooktitlelong}{A Practitioner's Guide To The Design And Development Of Small
                               Embedded Software}

%The command to be used to set text slightly inward, like when putting
%a long e-mail address on its own line.
\newcommand{\longthingindent}{\hspace*{14mm}}

%The order and number of authors may change.  Let's assume for now
%that there will be only two authors.  I do not know who should be
%listed first.  Swapping the strings below should change everything
%in the document.
\newcommand{\authors}{David T. Ashley}
\newcommand{\authorone}{David T. Ashley}

\title{No Title Defined}
\author{No Authors Defined}
\date{No Date Defined}

%Make the title page of the document.
%$Header: /cvsroot/esrg/sfesrg/esrgpcpj/hyreach/doc/hyreachm/general/tpage.tex,v 1.6 2002/01/29 17:04:00 dtashley Exp $

\thispagestyle{empty}
\vspace*{0.0cm}
\begin{flushright}
\Huge\bfseries
\swname{} User Manual
\end{flushright}
\vspace{0.0cm}
\noindent\rule{\textwidth}{2pt}
\begin{flushright}
\huge\bfseries
Version \swversion{}
\end{flushright}
\vfill
\begin{flushright}
\begin{small}
\authors{}
\end{small}
\end{flushright}
\vspace{0.2cm}
\begin{flushright}
\begin{small}
Document compiled from \LaTeX{} source code on \today .
\end{small}
\end{flushright}
\vspace*{0.0cm}

\pagebreak
\thispagestyle{empty}
\begin{small}
  \noindent Copyright \copyright 2001 \authors{}.
\end{small}

\vfill

%%%%%%%%%%%%%%%%%%%%%%%%%%%%%%%%%%%%%%%%%%%%%%%%%%%%%%%%%%%%%%%%%%%%%%%%%%
\noindent\begin{figure}[!b]
\noindent\rule[-0.25in]{\textwidth}{1pt}
\begin{tiny}
\begin{verbatim}
$RCSfile: tpage.tex,v $
$Source: /cvsroot/esrg/sfesrg/esrgpcpj/hyreach/doc/hyreachm/general/tpage.tex,v $
$Revision: 1.6 $
$Author: dtashley $
$Date: 2002/01/29 17:04:00 $
\end{verbatim}
\end{tiny}
\noindent\rule[0.25in]{\textwidth}{1pt}
\end{figure}
%%%%%%%%%%%%%%%%%%%%%%%%%%%%%%%%%%%%%%%%%%%%%%%%%%%%%%%%%%%%%%%%%%%%%%%%%%
%$Log: tpage.tex,v $
%Revision 1.6  2002/01/29 17:04:00  dtashley
%Version control info added, and minor edits.
%
%Revision 1.5  2001/10/31 04:31:37  dtashley
%Nightly safety checkin.
%
%Revision 1.4  2001/10/10 02:09:07  dtashley
%Edits.
%
%Revision 1.3  2001/10/10 00:30:07  dtashley
%Version control information which appears in output removed.
%
%Revision 1.2  2001/09/26 04:51:13  dtashley
%Edits.
%
%Revision 1.1  2001/09/26 03:45:49  dtashley
%Initial checkin.
%
%End of TPAGE.TEX

\clearpage

%After this, we want a ragged bottom.  When a document has little material
%(and hence not enough to move text around), this works better.
\raggedbottom

%Everything here is front matter.
\frontmatter{}

%Table of contents
\tableofcontents
\clearpage

%List of figures
\listoffigures
\clearpage

%List of tables
\listoftables

%Preface
%$Header: /cvsroot/esrg/sfesrg/esrgpcpj/doc/engman01/c_prf0/c_prf0.tex,v 1.4 2002/06/26 01:01:49 dtashley Exp $
\chapter{Preface}

\emph{The \tsname{} Version \tsversion{}} is a research-oriented tool set
developed to be useful in microcontroller and embedded control work.
The tool set contains a companion research work, the book entitled
\emph{\compbooktitlelong}.

Although the companion work is shamelessly and
deliberately somewhat of a catchall, it did not
seem useful to include tool set build instructions and other tool
engineering details in that work.  For the most part, building
\emph{The \tsname{}} from source code or modifying or contributing to
it is something that most users and casual readers will not
desire to do.  For this reason, those details are not
included in the companion work; instead, tool set engineering 
details here in this engineering manual.

This engineering manual is intended for individuals who desire
to:

\begin{itemize}
\item Build \emph{The \tsname{}} from source code.
\item Modify the tool set.
\item Collaborate with the ESRG to extend the tool set. 
\end{itemize}

Please direct all comments and suggestions about this engineering
manual to \texttt{toolbugs@esrg.org}.

\vspace{4mm}

\noindent{}---Dave Ashley (\texttt{dtashley@esrg.org}) \\
06/25/02, Detroit, Michigan, USA.


%%%%%%%%%%%%%%%%%%%%%%%%%%%%%%%%%%%%%%%%%%%%%%%%%%%%%%%%%%%%%%%%%%%%%%%%%%
\noindent\begin{figure}[!b]
\noindent\rule[-0.25in]{\textwidth}{1pt}
\begin{tiny}
\begin{verbatim}
$RCSfile: c_prf0.tex,v $
$Source: /cvsroot/esrg/sfesrg/esrgpcpj/doc/engman01/c_prf0/c_prf0.tex,v $
$Revision: 1.4 $
$Author: dtashley $
$Date: 2002/06/26 01:01:49 $
\end{verbatim}
\end{tiny}
\noindent\rule[0.25in]{\textwidth}{1pt}
\end{figure}
%%%%%%%%%%%%%%%%%%%%%%%%%%%%%%%%%%%%%%%%%%%%%%%%%%%%%%%%%%%%%%%%%%%%%%%%%%
% $Log: c_prf0.tex,v $
% Revision 1.4  2002/06/26 01:01:49  dtashley
% Edits, safety checkin.
%
% Revision 1.3  2002/06/25 02:44:16  dtashley
% Edits.
%
% Revision 1.2  2002/06/24 18:09:19  dtashley
% Formatting of log corrected.
%
% Revision 1.1  2002/06/24 18:08:44  dtashley
% Initial checkin.
%%%%%%%%%%%%%%%%%%%%%%%%%%%%%%%%%%%%%%%%%%%%%%%%%%%%%%%%%%%%%%%%%%%%%%%%%%
%End of file C_PRF0.TEX.


%Acknowledgements
%$Header: /cvsroot/esrg/sfesrg/esrgpcpj/doc/engman01/c_ack0/c_ack0.tex,v 1.3 2002/06/24 20:49:14 dtashley Exp $
\chapter{Acknowledgements}

We are grateful to \ldots{}

%%%%%%%%%%%%%%%%%%%%%%%%%%%%%%%%%%%%%%%%%%%%%%%%%%%%%%%%%%%%%%%%%%%%%%%%%%

\noindent\begin{figure}[!b]
\noindent\rule[-0.25in]{\textwidth}{1pt}
\begin{tiny}
\begin{verbatim}
$RCSfile: c_ack0.tex,v $
$Source: /cvsroot/esrg/sfesrg/esrgpcpj/doc/engman01/c_ack0/c_ack0.tex,v $
$Revision: 1.3 $
$Author: dtashley $
$Date: 2002/06/24 20:49:14 $
\end{verbatim}
\end{tiny}
\noindent\rule[0.25in]{\textwidth}{1pt}
\end{figure}

%%%%%%%%%%%%%%%%%%%%%%%%%%%%%%%%%%%%%%%%%%%%%%%%%%%%%%%%%%%%%%%%%%%%%%%%%%
% $Log: c_ack0.tex,v $
% Revision 1.3  2002/06/24 20:49:14  dtashley
% Edits.
%
% Revision 1.2  2002/06/24 18:08:19  dtashley
% Formatting of log corrected.
%
% Revision 1.1  2002/06/24 18:07:56  dtashley
% Initial checkin.
%%%%%%%%%%%%%%%%%%%%%%%%%%%%%%%%%%%%%%%%%%%%%%%%%%%%%%%%%%%%%%%%%%%%%%%%%%
%End of file C_ACK0.TEX.


%Everything here is the main matter.
\mainmatter

%COSP0: Introduction, Overview, Scope, And Purpose
\cleardoublepage{}
%$Header: /cvsroot/esrg/sfesrg/esrgpcpj/doc/engman01/c_osp0/c_osp0.tex,v 1.3 2002/06/25 02:44:16 dtashley Exp $
%
\chapter{Introduction, Overview, Scope, And Purpose}
\label{cosp0}

Content TBD.


%%%%%%%%%%%%%%%%%%%%%%%%%%%%%%%%%%%%%%%%%%%%%%%%%%%%%%%%%%%%%%%%%%%%%%%%%%
\noindent\begin{figure}[!b]
\noindent\rule[-0.25in]{\textwidth}{1pt}
\begin{tiny}
\begin{verbatim}
$RCSfile: c_osp0.tex,v $
$Source: /cvsroot/esrg/sfesrg/esrgpcpj/doc/engman01/c_osp0/c_osp0.tex,v $
$Revision: 1.3 $
$Author: dtashley $
$Date: 2002/06/25 02:44:16 $
\end{verbatim}
\end{tiny}
\noindent\rule[0.25in]{\textwidth}{1pt}
\end{figure}
%%%%%%%%%%%%%%%%%%%%%%%%%%%%%%%%%%%%%%%%%%%%%%%%%%%%%%%%%%%%%%%%%%%%%%%%%%
%$Log: c_osp0.tex,v $
%Revision 1.3  2002/06/25 02:44:16  dtashley
%Edits.
%
%Revision 1.2  2002/06/24 20:49:15  dtashley
%Edits.
%
%Revision 1.1  2002/06/24 18:22:20  dtashley
%Initial checkin.
%
%End of C_OSP0.TEX.


%CRSI0: Required Software Installations
\cleardoublepage{}
%$Header: /cvsroot/esrg/sfesrg/esrgpcpj/doc/engman01/c_rsi0/c_rsi0.tex,v 1.2 2002/07/29 16:53:09 dtashley Exp $
%
\chapter{Required Software Installations}
\label{crsi0}

The build process for Windows executables, Linux executables, 
Windows InstallShield Express installation executables, 
and \LaTeX{} documents requires many pieces of software.
Some of this software is free, and some is commercial
(and quite expensive).  In this chapter we describe 
all software required \emph{The \tsname{}} and all supporting
documentation on both Windows and *Nix platforms.  Note that
depending on what is being built, not all software may
be required.


%%%%%%%%%%%%%%%%%%%%%%%%%%%%%%%%%%%%%%%%%%%%%%%%%%%%%%%%%%%%%%%%%%%%%%%%%%
%%%%%%%%%%%%%%%%%%%%%%%%%%%%%%%%%%%%%%%%%%%%%%%%%%%%%%%%%%%%%%%%%%%%%%%%%%
%%%%%%%%%%%%%%%%%%%%%%%%%%%%%%%%%%%%%%%%%%%%%%%%%%%%%%%%%%%%%%%%%%%%%%%%%%
\section{Windows Software}
\label{crsi0:swso0}


%%%%%%%%%%%%%%%%%%%%%%%%%%%%%%%%%%%%%%%%%%%%%%%%%%%%%%%%%%%%%%%%%%%%%%%%%%
%%%%%%%%%%%%%%%%%%%%%%%%%%%%%%%%%%%%%%%%%%%%%%%%%%%%%%%%%%%%%%%%%%%%%%%%%%
%%%%%%%%%%%%%%%%%%%%%%%%%%%%%%%%%%%%%%%%%%%%%%%%%%%%%%%%%%%%%%%%%%%%%%%%%%
\subsection{Adobe Acrobat Version 4.0}
\label{crsi0:swso0:saaf0}

\index{Adobe!Acrobat}\index{Adobe!Acrobat Distiller}
\emph{Adobe Acrobat}, which costs abour \$250, contains a program
called \emph{Acrobat Distiller}.
Acrobat Distiller 
is the best application
we've found for 
creating \index{.PDF file@\texttt{.PDF} file}\texttt{.PDF} files from
\LaTeX{} source code.  The most successful
conversion approach seems to be 
to create an output file for a PostScript printer,
then to use Acrobat Distiller to convert it to 
\texttt{.PDF}.

To the best of our knowledge, it is not possible to purchase
Acrobat Distiller separate from Adobe Acrobat---Adobe
Acrobat must be purchased in order to obtain Acrobat Distiller.

When Adobe Acrobat is installed, all of the installation defaults
should be accepted.  \index{Tcl/Tk@\emph{Tcl/Tk}}Tcl/Tk 
scripts and \index{.BAT file@\texttt{.BAT} file} \texttt{.BAT}
files used in building \emph{The \tsname{}} rely on the 
executables being in the default locations.


%%%%%%%%%%%%%%%%%%%%%%%%%%%%%%%%%%%%%%%%%%%%%%%%%%%%%%%%%%%%%%%%%%%%%%%%%%
%%%%%%%%%%%%%%%%%%%%%%%%%%%%%%%%%%%%%%%%%%%%%%%%%%%%%%%%%%%%%%%%%%%%%%%%%%
%%%%%%%%%%%%%%%%%%%%%%%%%%%%%%%%%%%%%%%%%%%%%%%%%%%%%%%%%%%%%%%%%%%%%%%%%%
\subsection{\tsname{} Version \tsversion{}}
\label{crsi0:swso0:sets0}

%Note:  must hard code in the index entry below:  reason is processing
%       order of MAKEINDEX and LATEX.  As it is below, a global name 
%       change will get correct 

\index{ESRG Tool Set@\emph{\tsname{}}}
The \index{Tcl/Tk@\emph{Tcl/Tk}}Tcl/Tk interpreter 
(\index{EsrgConsole@\emph{EsrgConsole}}\emph{EsrgConsole}, 
the statically linked version
of \index{Wish@\emph{Wish}}\emph{Wish}, with extensions)
from \emph{The \tsname{}} is necessary to build
\emph{The \tsname{}} on Windows platforms.  Thus, \emph{The \tsname{}}
is necessary to build \emph{The \tsname{}}.  

At the present time, EsrgConsole is used only in the production
of the companion book.  However, in future revisions of
the tool set, the build complexity may become great enough that
it is used in the production of the Windows executables.

When installing \emph{The \tsname{}}, it is not necessary to
install the entire tool set---only the statically linked
Tcl/Tk interpreter (EsrgConsole) is required.  Copying the executable
from one computer to another will usually be adequate.  It is also not
necessary that the executable go in a certain place, since starting
the script interpreter is always a manual step in the build process.
No specific version of EsrgConsole relative to the version of the
tool set being built is \emph{actually} required; but
because it is not a good idea to allow freedom in the build process,
we canonically require that the version of EsrgConsole used for builds
should be the last stable release before the version being built.


%%%%%%%%%%%%%%%%%%%%%%%%%%%%%%%%%%%%%%%%%%%%%%%%%%%%%%%%%%%%%%%%%%%%%%%%%%
%%%%%%%%%%%%%%%%%%%%%%%%%%%%%%%%%%%%%%%%%%%%%%%%%%%%%%%%%%%%%%%%%%%%%%%%%%
%%%%%%%%%%%%%%%%%%%%%%%%%%%%%%%%%%%%%%%%%%%%%%%%%%%%%%%%%%%%%%%%%%%%%%%%%%
\subsection{InstallShield Express Version 3.03}
\label{crsi0:swso0:sise0}

\index{InstallShield Express@\emph{InstallShield Express}}
\emph{InstallShield Express}, which costs about
\$200, is the program used to create
the installation executable for Windows platforms.

Version 3.00 of the program contained some serious bugs, which were
corrected in subsequent maintenance releases.  At the time of this
writing, Version 3.03 is the most modern maintenance release.

When installing InstallShield Express, if the installation media
(typically a CD) are for a version earlier than version 3.03, it is
recommended to download maintenance release 3.03 from 
Installshield's web site and to install the maintenance release
without first installing from the installation media (note that the
maintenance release will require the serial number from the installation
media).  The reason for this recommendation is that the maintenance release
is standalone and just writes over earlier versions, anyway.
When installing InstallShield Express, it is not required that
installation defaults be accepted.  The generation of
Windows installation executables is at this time a manual process,
and no \index{Tcl/Tk@\emph{Tcl/Tk}}Tcl/Tk 
scripts or \index{.BAT file@\texttt{.BAT} file} \texttt{.BAT}
files rely on the location of the InstallShield Express executables.


%%%%%%%%%%%%%%%%%%%%%%%%%%%%%%%%%%%%%%%%%%%%%%%%%%%%%%%%%%%%%%%%%%%%%%%%%%
%%%%%%%%%%%%%%%%%%%%%%%%%%%%%%%%%%%%%%%%%%%%%%%%%%%%%%%%%%%%%%%%%%%%%%%%%%
%%%%%%%%%%%%%%%%%%%%%%%%%%%%%%%%%%%%%%%%%%%%%%%%%%%%%%%%%%%%%%%%%%%%%%%%%%
\section{*Nix Software}
\label{crsi0:snix0}





%%%%%%%%%%%%%%%%%%%%%%%%%%%%%%%%%%%%%%%%%%%%%%%%%%%%%%%%%%%%%%%%%%%%%%%%%%
%%%%%%%%%%%%%%%%%%%%%%%%%%%%%%%%%%%%%%%%%%%%%%%%%%%%%%%%%%%%%%%%%%%%%%%%%%
%%%%%%%%%%%%%%%%%%%%%%%%%%%%%%%%%%%%%%%%%%%%%%%%%%%%%%%%%%%%%%%%%%%%%%%%%%
\noindent\begin{figure}[!b]
\noindent\rule[-0.25in]{\textwidth}{1pt}
\begin{tiny}
\begin{verbatim}
$RCSfile: c_rsi0.tex,v $
$Source: /cvsroot/esrg/sfesrg/esrgpcpj/doc/engman01/c_rsi0/c_rsi0.tex,v $
$Revision: 1.2 $
$Author: dtashley $
$Date: 2002/07/29 16:53:09 $
\end{verbatim}
\end{tiny}
\noindent\rule[0.25in]{\textwidth}{1pt}
\end{figure}
%%%%%%%%%%%%%%%%%%%%%%%%%%%%%%%%%%%%%%%%%%%%%%%%%%%%%%%%%%%%%%%%%%%%%%%%%%
%$Log: c_rsi0.tex,v $
%Revision 1.2  2002/07/29 16:53:09  dtashley
%Safety checkins before being moved to WSU server Kalman.
%
%Revision 1.1  2002/06/25 02:24:21  dtashley
%Initial checkin.
%
%End of C_RSI0.TEX.


%CCVS0: Using CVS To Obtain Source Code And To Collaborate Electronically
\cleardoublepage{}
%$Header: /cvsroot/esrg/sfesrg/esrgpcpj/doc/engman01/c_cvs0/c_cvs0.tex,v 1.1 2002/06/25 02:30:09 dtashley Exp $
%
\chapter{Using CVS To Obtain Source Code And Collaborate Electronically}
\label{ccvs0}

Content TBD.


%%%%%%%%%%%%%%%%%%%%%%%%%%%%%%%%%%%%%%%%%%%%%%%%%%%%%%%%%%%%%%%%%%%%%%%%%%
\noindent\begin{figure}[!b]
\noindent\rule[-0.25in]{\textwidth}{1pt}
\begin{tiny}
\begin{verbatim}
$RCSfile: c_cvs0.tex,v $
$Source: /cvsroot/esrg/sfesrg/esrgpcpj/doc/engman01/c_cvs0/c_cvs0.tex,v $
$Revision: 1.1 $
$Author: dtashley $
$Date: 2002/06/25 02:30:09 $
\end{verbatim}
\end{tiny}
\noindent\rule[0.25in]{\textwidth}{1pt}
\end{figure}
%%%%%%%%%%%%%%%%%%%%%%%%%%%%%%%%%%%%%%%%%%%%%%%%%%%%%%%%%%%%%%%%%%%%%%%%%%
%$Log: c_cvs0.tex,v $
%Revision 1.1  2002/06/25 02:30:09  dtashley
%Initial checkin.
%
%End of C_CVS0.TEX.


%CBEW0: Building Tool Set Executables On A Microsoft Windows Platform
\cleardoublepage{}
%$Header: /cvsroot/esrg/sfesrg/esrgpcpj/doc/engman01/c_bew0/c_bew0.tex,v 1.1 2002/06/25 02:27:46 dtashley Exp $
%
\chapter{Building Tool Set Executables On A Microsoft Windows Platform}
\label{cbew0}

Content TBD.


%%%%%%%%%%%%%%%%%%%%%%%%%%%%%%%%%%%%%%%%%%%%%%%%%%%%%%%%%%%%%%%%%%%%%%%%%%
\noindent\begin{figure}[!b]
\noindent\rule[-0.25in]{\textwidth}{1pt}
\begin{tiny}
\begin{verbatim}
$RCSfile: c_bew0.tex,v $
$Source: /cvsroot/esrg/sfesrg/esrgpcpj/doc/engman01/c_bew0/c_bew0.tex,v $
$Revision: 1.1 $
$Author: dtashley $
$Date: 2002/06/25 02:27:46 $
\end{verbatim}
\end{tiny}
\noindent\rule[0.25in]{\textwidth}{1pt}
\end{figure}
%%%%%%%%%%%%%%%%%%%%%%%%%%%%%%%%%%%%%%%%%%%%%%%%%%%%%%%%%%%%%%%%%%%%%%%%%%
%$Log: c_bew0.tex,v $
%Revision 1.1  2002/06/25 02:27:46  dtashley
%Initial checkin.
%
%End of C_BEW0.TEX.


%CBEX0: Building Tool Set Executables On A *Nix Platform
\cleardoublepage{}
%$Header: /cvsroot/esrg/sfesrg/esrgpcpj/doc/engman01/c_bex0/c_bex0.tex,v 1.1 2002/06/25 02:33:36 dtashley Exp $
%
\chapter{Building Tool Set Executables On A *Nix Platform}
\label{cbex0}

Content TBD.


%%%%%%%%%%%%%%%%%%%%%%%%%%%%%%%%%%%%%%%%%%%%%%%%%%%%%%%%%%%%%%%%%%%%%%%%%%
\noindent\begin{figure}[!b]
\noindent\rule[-0.25in]{\textwidth}{1pt}
\begin{tiny}
\begin{verbatim}
$RCSfile: c_bex0.tex,v $
$Source: /cvsroot/esrg/sfesrg/esrgpcpj/doc/engman01/c_bex0/c_bex0.tex,v $
$Revision: 1.1 $
$Author: dtashley $
$Date: 2002/06/25 02:33:36 $
\end{verbatim}
\end{tiny}
\noindent\rule[0.25in]{\textwidth}{1pt}
\end{figure}
%%%%%%%%%%%%%%%%%%%%%%%%%%%%%%%%%%%%%%%%%%%%%%%%%%%%%%%%%%%%%%%%%%%%%%%%%%
%$Log: c_bex0.tex,v $
%Revision 1.1  2002/06/25 02:33:36  dtashley
%Initial checkin.
%
%End of C_BEX0.TEX.


%CBCB0: Building The Companion Book On A Microsoft Windows Platform
\cleardoublepage{}
%$Header: /cvsroot/esrg/sfesrg/esrgpcpj/doc/engman01/c_bcb0/c_bcb0.tex,v 1.1 2002/06/25 02:35:20 dtashley Exp $
%
\chapter{Building The Companion Book On A Microsoft Windows Platform}
\label{cbcb0}

Content TBD.


%%%%%%%%%%%%%%%%%%%%%%%%%%%%%%%%%%%%%%%%%%%%%%%%%%%%%%%%%%%%%%%%%%%%%%%%%%
\noindent\begin{figure}[!b]
\noindent\rule[-0.25in]{\textwidth}{1pt}
\begin{tiny}
\begin{verbatim}
$RCSfile: c_bcb0.tex,v $
$Source: /cvsroot/esrg/sfesrg/esrgpcpj/doc/engman01/c_bcb0/c_bcb0.tex,v $
$Revision: 1.1 $
$Author: dtashley $
$Date: 2002/06/25 02:35:20 $
\end{verbatim}
\end{tiny}
\noindent\rule[0.25in]{\textwidth}{1pt}
\end{figure}
%%%%%%%%%%%%%%%%%%%%%%%%%%%%%%%%%%%%%%%%%%%%%%%%%%%%%%%%%%%%%%%%%%%%%%%%%%
%$Log: c_bcb0.tex,v $
%Revision 1.1  2002/06/25 02:35:20  dtashley
%Initial checkin.
%
%End of C_BCB0.TEX.


%CBCB0: Building The InstallShield Express Installation Executable On
%       A Microsoft Windows Platform
\cleardoublepage{}
%$Header: /cvsroot/esrg/sfesrg/esrgpcpj/doc/engman01/c_bis0/c_bis0.tex,v 1.1 2002/06/25 02:38:34 dtashley Exp $
%
\chapter{Building The InstallShield Express Installation Executable On
         A Microsoft Windows Platform}
\label{cbis0}

Content TBD.


%%%%%%%%%%%%%%%%%%%%%%%%%%%%%%%%%%%%%%%%%%%%%%%%%%%%%%%%%%%%%%%%%%%%%%%%%%
\noindent\begin{figure}[!b]
\noindent\rule[-0.25in]{\textwidth}{1pt}
\begin{tiny}
\begin{verbatim}
$RCSfile: c_bis0.tex,v $
$Source: /cvsroot/esrg/sfesrg/esrgpcpj/doc/engman01/c_bis0/c_bis0.tex,v $
$Revision: 1.1 $
$Author: dtashley $
$Date: 2002/06/25 02:38:34 $
\end{verbatim}
\end{tiny}
\noindent\rule[0.25in]{\textwidth}{1pt}
\end{figure}
%%%%%%%%%%%%%%%%%%%%%%%%%%%%%%%%%%%%%%%%%%%%%%%%%%%%%%%%%%%%%%%%%%%%%%%%%%
%$Log: c_bis0.tex,v $
%Revision 1.1  2002/06/25 02:38:34  dtashley
%Initial checkin.
%
%End of C_BIS0.TEX.


%CSUP0: Obtaining Support And Collaborating Electronically
\cleardoublepage{}
%$Header: /cvsroot/esrg/sfesrg/esrgpcpj/doc/engman01/c_sup0/c_sup0.tex,v 1.2 2002/06/26 01:01:49 dtashley Exp $
%
\chapter[Support And Collaboration]
        {Obtaining Support And Collaborating Electronically}
\label{csup0}

This chapter outlines support options; specifically:

\begin{itemize}
\item How to obtain support if \emph{The \tsname{}} does not
      behave as expected.
\item How to obtain support if there is difficulty building the
      tool set from source code.
\item How to obtain support in collaboration (in contributing to 
      the book, or in extending the tool set).
\item How to report tool bugs or errors/omissions in documentation.
\end{itemize}

%%%%%%%%%%%%%%%%%%%%%%%%%%%%%%%%%%%%%%%%%%%%%%%%%%%%%%%%%%%%%%%%%%%%%%%%%%
%%%%%%%%%%%%%%%%%%%%%%%%%%%%%%%%%%%%%%%%%%%%%%%%%%%%%%%%%%%%%%%%%%%%%%%%%%
%%%%%%%%%%%%%%%%%%%%%%%%%%%%%%%%%%%%%%%%%%%%%%%%%%%%%%%%%%%%%%%%%%%%%%%%%%
\section{Mailing List Support And Collaboration}
\label{csup0:smls0}

\index{mailing lists}\index{discussion lists}
The foremost support option is to join one of the mailing lists
listed in Table \ref{tbl:csup0:smls0:01}.

\begin{table}
\begin{center}
\begin{tabular}{|l|l|l|}
\hline
Mailing List         & Scope                    & Subscription, Moderation, \\
Base Name            &                          & And Archive Availability  \\
                     &                          & Policy                    \\
\hline
\hline
\texttt{esrgmgmt}    & Used by the management   & Subscription by invitation\\
                     & of the ESRG to discuss   & only.  Posts unmoderated  \\
                     & research and policy      & but must come from sub-   \\
                     &                          & scribed members.  Archives\\
                     &                          & available to subscribed   \\
                     &                          & members only.             \\
\hline
\texttt{esrgtsbug}   & Discussion of tool set   & Unmoderated subscription. \\
                     & bugs and suspected bugs. & Unmoderated posts.        \\
                     & research and policy      & Archives available to     \\
                     & issues.                  & subscribed members only.  \\
\hline
\texttt{esrgtsdev}   & Discussion of tool set   & Unmoderated subscription. \\
                     & software development.    & Unmoderated posts.        \\
                     &                          & Archives available to     \\
                     &                          & subscribed members only.  \\
\hline
\texttt{esrgtslic}   & Discussion of tool set   & Unmoderated subscription. \\
                     & license questions and    & Unmoderated posts.        \\
                     & issues.                  & Archives available to     \\
                     &                          & subscribed members only.  \\
\hline
\texttt{esrgtsusr}   & Discussion of tool set   & Unmoderated subscription. \\
                     & issues that arise through& Unmoderated posts.        \\
                     & using the tool set, i.e. & Archives available to     \\
                     & ordinary support         & subscribed members only.  \\
                     & questions.               &                           \\
\hline
\hline
\texttt{esrgubka}    & Discussion of all issues & Unmoderated subscription. \\
                     & related to the authorship& Unmoderated posts.        \\
                     & and production of the    & Archives available to     \\
                     & companion book.          & subscribed members only.  \\
\hline
\hline
\texttt{esrgucsw}    & Discussion of [primarily & Unmoderated subscription. \\
                     & research and best        & Unmoderated posts.        \\
                     & practice] issues in      & Archives available to     \\
                     & microcontroller software & subscribed members only.  \\
                     & development.             &                           \\
\hline
\end{tabular}
\end{center}
\caption{Support And Collaboration Mailing Lists Maintained By The ESRG}
\label{tbl:csup0:smls0:01}
\end{table}

To join one of the mailing lists listed in 
Table \ref{tbl:csup0:smls0:01}, it is only necessary to 
send an e-mail to the \index{mailing lists!subscribing}
list manager \index{ezmlm-idx@\texttt{ezmlm-idx}}(\texttt{ezmlm-idx}).
The request to subscribe must be sent to the e-mail address \\
\longthingindent\texttt{list-}\emph{basename}\texttt{-subscribe@esrg.org},   \\
where \emph{basename} is the base name supplied in 
Table \ref{tbl:csup0:smls0:01}.  For example, to join the
\texttt{esrgtsusr} mailing list, an e-mail must be sent to \\
\longthingindent \texttt{list-esrgtsusr-subscribe@esrg.org}. \\
The 
\texttt{ezmlm-idx} list manager will
respond with a confirmation e-mail which must be replied to, and then
the subscription will be complete.  After subscription, the subscribed
e-mail address will receive all postings to the list.

\index{mailing lists!posting}To post to a 
list (after subscription), send an e-mail to \\
\longthingindent\texttt{list-}\emph{basename}\texttt{@esrg.org}, \\
where 
\emph{basename} is the base name supplied in 
Table \ref{tbl:csup0:smls0:01}.  For example, to post to the
\texttt{esrgtsusr} mailing list, an e-mail must be sent to \\
\longthingindent\texttt{list-esrgtsusr@esrg.org}.  \\
All lists allow attachments.

\index{mailing lists!unsubscribing}To unsubscribe 
from a list, send an e-mail to \\ 
\longthingindent\texttt{list-}\emph{basename}\texttt{-unsubscribe@esrg.org}, \\
where 
\emph{basename} is the base name supplied in 
Table \ref{tbl:csup0:smls0:01}.  For example, to unsubscribe from the
\texttt{esrgtsusr} mailing list, an e-mail must be sent to \\
\longthingindent\texttt{list-esrgtsusr-unsubscribe@esrg.org}.

The \texttt{ezmlm-idx} list manager will also allow past messages
to be automatically retrieved, and has other useful features.  
To automatically obtain a list of
all commands that can be issued to the 
\texttt{ezmlm-idx} list manager, an e-mail can be sent to \\
\longthingindent\texttt{list-}\emph{basename}\texttt{-help@esrg.org}, \\
where 
\emph{basename} is the base name supplied in 
Table \ref{tbl:csup0:smls0:01}.
For example, to obtain help for the
\texttt{esrgtsusr} mailing list, an e-mail can be sent to \\
\longthingindent\texttt{list-esrgtsusr-help@esrg.org}. \\
The \texttt{ezmlm-idx} list manager will reply with
a help message.


%%%%%%%%%%%%%%%%%%%%%%%%%%%%%%%%%%%%%%%%%%%%%%%%%%%%%%%%%%%%%%%%%%%%%%%%%%
%%%%%%%%%%%%%%%%%%%%%%%%%%%%%%%%%%%%%%%%%%%%%%%%%%%%%%%%%%%%%%%%%%%%%%%%%%
%%%%%%%%%%%%%%%%%%%%%%%%%%%%%%%%%%%%%%%%%%%%%%%%%%%%%%%%%%%%%%%%%%%%%%%%%%
\section{Contacting Individuals Within The ESRG}
\label{csup0:scin0}

Directly contacting individuals within the ESRG is not recommended.
There are several reasons for this:

\begin{itemize}
\item A post to one of the mailing lists is more likely to put the 
      problem or question in the hands of the person who is best able
      to answer it.
\item E-mails sent to individuals are not automatically archived.
\item Answers to e-mails sent to individuals do not receive the
      benefits of peer review.
\item E-mail sent to individuals may result in an inordinate
      support load on some individuals.
\end{itemize}

For this reason, please do not send e-mails with questions 
or requests for support directly
to individuals within the ESRG.


%%%%%%%%%%%%%%%%%%%%%%%%%%%%%%%%%%%%%%%%%%%%%%%%%%%%%%%%%%%%%%%%%%%%%%%%%%
%%%%%%%%%%%%%%%%%%%%%%%%%%%%%%%%%%%%%%%%%%%%%%%%%%%%%%%%%%%%%%%%%%%%%%%%%%
%%%%%%%%%%%%%%%%%%%%%%%%%%%%%%%%%%%%%%%%%%%%%%%%%%%%%%%%%%%%%%%%%%%%%%%%%%



%%%%%%%%%%%%%%%%%%%%%%%%%%%%%%%%%%%%%%%%%%%%%%%%%%%%%%%%%%%%%%%%%%%%%%%%%%
%%%%%%%%%%%%%%%%%%%%%%%%%%%%%%%%%%%%%%%%%%%%%%%%%%%%%%%%%%%%%%%%%%%%%%%%%%
%%%%%%%%%%%%%%%%%%%%%%%%%%%%%%%%%%%%%%%%%%%%%%%%%%%%%%%%%%%%%%%%%%%%%%%%%%
\noindent\begin{figure}[!b]
\noindent\rule[-0.25in]{\textwidth}{1pt}
\begin{tiny}
\begin{verbatim}
$RCSfile: c_sup0.tex,v $
$Source: /cvsroot/esrg/sfesrg/esrgpcpj/doc/engman01/c_sup0/c_sup0.tex,v $
$Revision: 1.2 $
$Author: dtashley $
$Date: 2002/06/26 01:01:49 $
\end{verbatim}
\end{tiny}
\noindent\rule[0.25in]{\textwidth}{1pt}
\end{figure}
%%%%%%%%%%%%%%%%%%%%%%%%%%%%%%%%%%%%%%%%%%%%%%%%%%%%%%%%%%%%%%%%%%%%%%%%%%
%$Log: c_sup0.tex,v $
%Revision 1.2  2002/06/26 01:01:49  dtashley
%Edits, safety checkin.
%
%Revision 1.1  2002/06/25 02:42:44  dtashley
%Initial checkin.
%
%End of C_SUP0.TEX.


%CGLO0: Glossary Of Terms
\cleardoublepage
\addcontentsline{toc}{chapter}{Glossary Of Terms}
\chapter*{Glossary Of Terms}
\markboth{GLOSSARY OF TERMS}{GLOSSARY OF TERMS}

\label{cglo0}

\begin{vworktermglossaryenum}

\item \textbf{axiom}\index{axiom}

      A statement used in the premises of arguments and assumed to be true
	  without proof.  In some cases axioms are held to be self-evident, as in 
	  Euclidian geometry, while in others they are assumptions put forward for
	  the sake of argument.
      (Taken verbatim from \cite{bibref:b:penguindictionaryofmathematics:2ded}.)

\item \textbf{cardinality}\index{cardinality}

      The cardinality of a set is the
      number of elements in the set.  In this work, the cardinality
      of a set is denoted $n()$.  For example, 
      $n(\{12,29,327\}) = 3$.

\item \textbf{coprime}\index{coprime}

      Two integers that share no prime factors are \emph{coprime}.
      \emph{Example:}
      6 and 7 are coprime, whereas 6 and 8 are not.

\item \textbf{GMP}\index{GMP}

      The \emph{G}NU \emph{M}ultiple \emph{P}recision library.
      The GMP is an arbitrary-precision integer, rational number,
      and floating-point library that places no restrictions on
      size of integers or number of significant digits in floating-point
      numbers.  This 
      library is famous because it is the fastest of its
      kind, and generally uses asymptotically superior algorithms.

\item \textbf{greatest common divisor (g.c.d.)}

      The greatest common divisor of two integers is the largest
      integer which divides both integers without a remainder.
      \emph{Example:} the g.c.d. of 30 and 42 is 6.

\item \textbf{integer}\index{integer}\index{sets of integers}\index{Z@$\vworkintset$}%
      \index{integer!Z@$\vworkintset$}\index{integer!sets of}

      (Nearly verbatim from \cite{bibref:w:wwwwhatiscom}) An \emph{integer}
      (pronounced \emph{IN-tuh-jer}) is a whole number
      (not a fractional number) that can be positive, negative, or zero. 

      Examples of integers are: -5, 1, 5, 8, 97, and 3,043. 

      Examples of numbers that are not integers are: -1.43, 1 3/4, 3.14, 
      0.09, and 5,643.1. 

      The set of integers, denoted $\vworkintset{}$, is formally defined as:

      \begin{equation}
      \vworkintset{} = \{\ldots{}, -3, -2, -1, 0, 1, 2, 3, \ldots{} \}
      \end{equation}

      In mathematical equations, unknown or unspecified integers are 
      represented by lowercase, italicized letters from the 
      ``late middle'' of the alphabet.  The most common 
      are $p$, $q$, $r$, and $s$.

\item \textbf{irreducible}

      A rational number $p/q$ where $p$ and $q$ are coprime
      is said to be \emph{irreducible}.
      Equivalently, it may be stated that $p$ and $q$ share no prime factors
      or that the greatest common divisor of
      $p$ and $q$ is 1.

\item \textbf{KPH}

      Kilometers per hour.

\item \textbf{limb}\index{limb}

      An integer of a size which a machine can manipulate natively
      that is arranged in an array to create a larger
      integer which the machine cannot manipulate natively and must be
      manipulated through arithmetic subroutines.

\item \textbf{limbsize}\index{limbsize}

      The size, in bits, of a limb.  The limbsize usually represents
      the size of integer that a machine can manipulate directly
      through machine instructions.  For an inexpensive microcontroller,
      8 or 16 is a typical limbsize.  For a personal computer or 
      workstation, 32 or 64 is a typical limbsize.

\item \textbf{MPH}

      Miles per hour.

\item \textbf{mediant}\index{mediant}

      The mediant of two fractions $m/n$ and $m'/n'$ is the fraction 
	  $\frac{m+m'}{n+n'}$ (see Definition 
	  \ref{def:cfry0:spfs:02}).  Note that the
	  mediant of two fractions with non-negative integer components
	  is always between them, but not usually exactly at the 
	  midpoint (see Lemma \ref{lem:cfry0:spfs:02c}).

\item \textbf{natural number}\index{natural number}\index{integer!natural number}%
      \index{sets of integers}\index{N@$\vworkintsetpos$}%
      \index{integer!N@$\vworkintsetpos$}\index{integer!sets of}
         
      (Nearly verbatim from \cite{bibref:w:wwwwhatiscom})
      A \emph{natural number}
      is a number that occurs commonly and obviously in nature.  
      As such, it is a whole, non-negative number.  
      The set of natural numbers, denoted $\vworkintsetpos{}$, 
      can be defined in either of two ways:

      \begin{equation}
      \label{cglo0:eq0001}
      \vworkintsetpos{} = \{ 0, 1, 2, 3, \ldots{} \}
      \end{equation}

      \begin{equation}
      \label{cglo0:eq0002}
      \vworkintsetpos{} = \{ 1, 2, 3, 4, \ldots{} \}
      \end{equation}
      
      In mathematical equations, unknown or unspecified natural numbers 
      are represented by lowercase, italicized letters from the 
      middle of the alphabet.  The most common is $n$, followed by 
      $m$, $p$, and $q$.  
      In subscripts, the lowercase $i$ is sometimes used to represent 
      a non-specific natural number when denoting the elements in a 
      sequence or series.  However, $i$ is more often used to represent 
      the positive square root of -1, the unit imaginary number.

      \textbf{Important Note:}  The definition above is reproduced nearly
      verbatim from \cite{bibref:w:wwwwhatiscom}, and (\ref{cglo0:eq0001})
      is supplied only for perspective.  In this work, a natural
      number is defined by (\ref{cglo0:eq0002}) rather than (\ref{cglo0:eq0001}).
      In this work, the set of non-negative integers is denoted by
      $\vworkintsetnonneg{}$ rather than $\vworkintsetpos{}$.\index{Z+@$\vworkintsetnonneg$}%
      \index{integer!Z+@$\vworkintsetnonneg$}\index{integer!non-negative}

\item \textbf{postulate}\index{postulate!definition}

      An axiom (see \emph{axiom} earlier in this glossary).  The term is usually
	  used in certain contexts, e.g. Euclid's postulates or Peano's postulates.
	  (Taken verbatim from \cite{bibref:b:penguindictionaryofmathematics:2ded}.)

\item \textbf{prime number}\index{prime number!definition}

      (Nearly verbatim from \cite{bibref:w:wwwwhatiscom}) A \emph{prime number}
      is a whole number greater than 1, whose only two whole-number 
      factors are 1 and itself.  The first few prime numbers are 
      2, 3, 5, 7, 11, 13, 17, 19, 23, and 29.  As we proceed in the set of 
      natural numbers $\vworkintsetpos{} = \{ 1, 2, 3, \ldots{} \} $, the 
      primes become less and less frequent in general.  
      However, there is no largest prime number.  
      For every prime number $p$, there exists a prime number $p'$ such that 
      $p'$ is greater than $p$.  This was demonstrated in ancient times by the 
      Greek mathematician \index{Euclid}Euclid.\index{prime number!no largest prime number}%
      \index{Euclid!Second Theorem}

      Suppose $n$ is a whole number, and we want to test it to see if it is prime.   
      First, we take the square root (or the 1/2 power) of $n$; then we round this 
      number up to the next highest whole number.  Call the result $m$.  
      We must find all of the following quotients:

      \begin{equation}
      \begin{array}{rcl}
         q_m     & =        & n / m              \\
         q_{m-1} & =        & n / (m-1)          \\
         q_{m-2} & =        & n / (m-2)          \\
         q_{m-3} & =        & n / (m-3)          \\
                 & \ldots{} &                    \\
         q_3     & =        & n / 3              \\
         q_2     & =        & n / 2              \\
      \end{array}
      \end{equation}

      The number $n$ is prime if and only if none of the $q$'s, as 
      derived above, are whole numbers.

      A computer can be used to test extremely large numbers to see if they are prime.  
      But, because there is no limit to how large a natural number can be, 
      there is always a point where testing in this manner becomes too great 
      a task even for the most powerful supercomputers.  
      Various algorithms have been formulated in an attempt to generate 
      ever-larger prime numbers.  These schemes all have limitations.

\end{vworktermglossaryenum}

%End of file c_glo0.tex



%CGLO0: Glossary Of Symbols And Mathematical Notation
\cleardoublepage
\addcontentsline{toc}{chapter}{Glossary Of Symbols And Mathematical Notation}
%$Header: /cvsroot/esrg/sfesrg/esrgpcpj/doc/engman01/c_glm0/c_glm0.tex,v 1.1 2002/06/24 20:37:42 dtashley Exp $
%
\chapter*{Glossary Of Symbols And Mathematical Notation}
\markboth{GLOSSARY OF NOTATION}{GLOSSARY OF NOTATION}
\label{cglm0}

\begin{glossaryenum}
\item {\boldmath{}$\mathbb{R}^N$}\index{RN@$\mathbb{R}^N$} \\
      Under construction.
\end{glossaryenum}


%%%%%%%%%%%%%%%%%%%%%%%%%%%%%%%%%%%%%%%%%%%%%%%%%%%%%%%%%%%%%%%%%%%%%%%%%%
\noindent\begin{figure}[!b]
\noindent\rule[-0.25in]{\textwidth}{1pt}
\begin{tiny}
\begin{verbatim}
$RCSfile: c_glm0.tex,v $
$Source: /cvsroot/esrg/sfesrg/esrgpcpj/doc/engman01/c_glm0/c_glm0.tex,v $
$Revision: 1.1 $
$Author: dtashley $
$Date: 2002/06/24 20:37:42 $
\end{verbatim}
\end{tiny}
\noindent\rule[0.25in]{\textwidth}{1pt}
\end{figure}
%%%%%%%%%%%%%%%%%%%%%%%%%%%%%%%%%%%%%%%%%%%%%%%%%%%%%%%%%%%%%%%%%%%%%%%%%%
%$Log: c_glm0.tex,v $
%Revision 1.1  2002/06/24 20:37:42  dtashley
%Initial checkin.
%
%End of S_GLM0.TEX.


%CBIB0: Bibliography
\cleardoublepage
\addcontentsline{toc}{chapter}{Bibliography}
%$Header: /cvsroot/esrg/sfesrg/esrgpcpj/doc/engman01/c_bib0/c_bib0.tex,v 1.1 2002/06/24 20:45:21 dtashley Exp $
%
%\chapter*{Glossary Of Terms}
%\label{cglo0}

%\chapter*{Bibliography}
%\markboth{BIBLIOGRAPHY}{BIBLIOGRAPHY}
\label{cbib0}

%Note:  This environment (thebibliography) has been hacked in ARTICLE.CLS.

\begin{thebibliography}{99}
\bibitem{bib:p:hybridsysintlat:lin90} Feng Lin,
   David T. Ashley, Michael Heymann, and Michael
   J. Burke:  \emph{A Hybrid System Solution Of The
   Interrupt Latency Compatibility Problem}, SAE
   Paper 99PC-314; \\
   \texttt{http://www.ece.eng.wayne.edu/$\sim$flin/reprints/interrupt.ps}.

\bibitem{bib:p:theoryofta:alurdill90} Rajeev Alur and 
   David L. Dill:  in \emph{A Theory Of Timed Automata}; 
   \emph{Theoretical Computer Science} 126:183-235, 1994 
   (preliminary versions appeared in Proc. 17th ICALP, LNCS 443, 1990, 
   and \emph{Real Time: Theory in Practice}, LNCS 600, 1991);\\
   \texttt{http://www.cis.upenn.edu/$\sim$alur/Icalp90.ps.gz}.

\bibitem{bib:i:mburke:patriciabouyer}Patricia Bouyer, 
   \texttt{bouyer@lsv.ens-cachan.fr}.

\bibitem{bib:i:mburke:visteon} Michael J. Burke, 
   \texttt{mburke@visteon.com}.

\bibitem{bib:b:modelchecking:clark1999} Edmund M. Clarke, Jr.,
   Orna Grumberg, and Doron A. Peled; \emph{Model Checking};
   1999, MIT Press, ISBN 0-262-03270-8.

\bibitem{bib:i:jdevoe:visteon} Joseph P. DeVoe, 
   \texttt{jdevoe@visteon.com}.

\bibitem{bib:p:memblktrav:flppwy} Fredrik Larsson, Paul
   Pettersson, and Wang Yi:  \emph{On Memory-Block Traversal Problems 
   in Model-Checking Timed Systems}; 
   in \emph{Proceedings of the 6th International Conference on Tools and Algorithms 
   for the Construction and Analysis of Systems}, (TACAS'2000); Berlin, Germany, 
   March 27-April 1, 2000; LNCS 1785, pages 127-141, Susanne Graf and Michael Schwartzbach 
   (Eds.);\\
   \texttt{http://www.docs.uu.se/docs/rtmv/papers/lpw-tacas00.pdf}.

\bibitem{bib:p:thsu2k:kglppet} Kim G. Larsen and Paul Pettersson;
   \emph{Timed And Hybrid Systems In UPPAAL2k};
   presented at MOVEP'2k, June 21, 2000; \\
   \texttt{http://www.cs.auc.dk/$\sim$paupet/talks/MOVEP2k.pdf}.
\end{thebibliography}

%%%%%%%%%%%%%%%%%%%%%%%%%%%%%%%%%%%%%%%%%%%%%%%%%%%%%%%%%%%%%%%%%%%%%%%%%%
\noindent\begin{figure}[!b]
\noindent\rule[-0.25in]{\textwidth}{1pt}
\begin{tiny}
\begin{verbatim}
$RCSfile: c_bib0.tex,v $
$Source: /cvsroot/esrg/sfesrg/esrgpcpj/doc/engman01/c_bib0/c_bib0.tex,v $
$Revision: 1.1 $
$Author: dtashley $
$Date: 2002/06/24 20:45:21 $
\end{verbatim}
\end{tiny}
\noindent\rule[0.25in]{\textwidth}{1pt}
\end{figure}
%%%%%%%%%%%%%%%%%%%%%%%%%%%%%%%%%%%%%%%%%%%%%%%%%%%%%%%%%%%%%%%%%%%%%%%%%%
%$Log: c_bib0.tex,v $
%Revision 1.1  2002/06/24 20:45:21  dtashley
%Initial checkin.
%
%End of C_BIB0.TEX.


%CIDX0: Index
\cleardoublepage
\addcontentsline{toc}{chapter}{Index}
%$Header: /cvsroot/esrg/sfesrg/esrgpcpj/doc/engman01/c_idx0/c_idx0.tex,v 1.2 2002/06/24 20:49:14 dtashley Exp $
%

\printindex{}

%$Log: c_idx0.tex,v $
%Revision 1.2  2002/06/24 20:49:14  dtashley
%Edits.
%
%Revision 1.1  2002/06/24 20:45:56  dtashley
%Initial checkin.
%
%End of C_IDX0.TEX.


\end{document}
%
%%%%%%%%%%%%%%%%%%%%%%%%%%%%%%%%%%%%%%%%%%%%%%%%%%%%%%%%%%%%%%%%%%%%%%%%%%%%%%
%$Log: engman01.tex,v $
%Revision 1.5  2002/06/26 01:01:49  dtashley
%Edits, safety checkin.
%
%Revision 1.4  2002/06/25 02:44:15  dtashley
%Edits.
%
%Revision 1.3  2002/06/24 20:49:14  dtashley
%Edits.
%
%Revision 1.2  2002/06/24 18:17:17  dtashley
%Formatting of log area corrected.
%
%Revision 1.1  2002/06/24 18:16:34  dtashley
%Initial checkin.
%%%%%%%%%%%%%%%%%%%%%%%%%%%%%%%%%%%%%%%%%%%%%%%%%%%%%%%%%%%%%%%%%%%%%%%%%%%%%%
%
%End of ENGMAN01.TEX.
