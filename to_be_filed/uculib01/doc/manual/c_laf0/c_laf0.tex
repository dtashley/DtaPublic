%$Header: /home/dashley/cvsrep/uculib01/uculib01/doc/manual/c_laf0/c_laf0.tex,v 1.7 2010/05/12 18:56:18 dashley Exp $

\chapter[Large-Operand and Extended-Precision Arithmetic Functions]
        {Large-Operand and Extended-Precision Arithmetic Functions}

\chaptermark{Large-Operand and Extended-Precision}

\label{claf0}

%%%%%%%%%%%%%%%%%%%%%%%%%%%%%%%%%%%%%%%%%%%%%%%%%%%%%%%%%%%%%%%%%%%%%%%%%%%%%%%
%%%%%%%%%%%%%%%%%%%%%%%%%%%%%%%%%%%%%%%%%%%%%%%%%%%%%%%%%%%%%%%%%%%%%%%%%%%%%%%
%%%%%%%%%%%%%%%%%%%%%%%%%%%%%%%%%%%%%%%%%%%%%%%%%%%%%%%%%%%%%%%%%%%%%%%%%%%%%%%
\section{Introduction and Overview}
\label{claf0:siov0}


%%%%%%%%%%%%%%%%%%%%%%%%%%%%%%%%%%%%%%%%%%%%%%%%%%%%%%%%%%%%%%%%%%%%%%%%%%%%%%%
%%%%%%%%%%%%%%%%%%%%%%%%%%%%%%%%%%%%%%%%%%%%%%%%%%%%%%%%%%%%%%%%%%%%%%%%%%%%%%%
%%%%%%%%%%%%%%%%%%%%%%%%%%%%%%%%%%%%%%%%%%%%%%%%%%%%%%%%%%%%%%%%%%%%%%%%%%%%%%%
\section{Unsigned Integer Division Functions}
\label{claf0:suid0}

TBD.


%%%%%%%%%%%%%%%%%%%%%%%%%%%%%%%%%%%%%%%%%%%%%%%%%%%%%%%%%%%%%%%%%%%%%%%%%%%%%%%
%%%%%%%%%%%%%%%%%%%%%%%%%%%%%%%%%%%%%%%%%%%%%%%%%%%%%%%%%%%%%%%%%%%%%%%%%%%%%%%
%%%%%%%%%%%%%%%%%%%%%%%%%%%%%%%%%%%%%%%%%%%%%%%%%%%%%%%%%%%%%%%%%%%%%%%%%%%%%%%
%\subsection[\emph{UcuLoU32U64U32DivQFRxn(\protect\mbox{\protect$\cdot$})}]
%           {\emph{UcuLoU32U64U32DivQFRxn(\protect\mbox{\protect\boldmath $\cdot$})}}
%\label{claf0:suid0:sdsf0}
%
%\index{UcuLoU32U64U32DivQFRxn()@\emph{UcuLoU32U64U32DivQFRxn($\cdot$)}}%
%
%\noindent\textbf{PROTOTYPE}
%\begin {list}{}{\setlength{\leftmargin}{0.25in}\setlength{\topsep}{0.0in}}
%\item
%\begin{verbatim}
%void UcuLoU32U64U32DivQFRxn( 
%                                 UCU_UINT32  *out_quotient, 
%                           const UCU_UNION64 *in_dividend,
%                           const UCU_UINT32  *in_divisor
%                           )
%\end{verbatim}
%\end{list}
%\vspace{2.8ex}
%
%\noindent\textbf{SYNOPSIS}
%\begin{list}{}{\setlength{\leftmargin}{0.25in}\setlength{\topsep}{0.0in}}
%\item Calculates
%
%      \begin{equation}
%      *out\_quotient = \left\lfloor \frac{*in\_dividend}{*in\_divisor} \right\rfloor
%      \end{equation}
%
%      using a 32-iteration classic shift/compare/subtract division
%      algorithm where it is known in advance that the quotient will not exceed 32 bits.\footnote{The
%      general case of division of a 64-bit unsigned integer dividend by a 32-bit unsigned integer
%      divisor may produce up to a 64-bit unsigned integer quotient.  This function is not a 
%      general division function and should only be used when it is known that the quotient
%      cannot exceed 32 bits.}
%\end{list}
%\vspace{2.8ex}
%
%\noindent\textbf{INPUT}
%\begin{list}{}{\setlength{\leftmargin}{0.5in}\setlength{\itemindent}{-0.25in}\setlength{\topsep}{0.0in}\setlength{\partopsep}{0.0in}}
%\item \emph{\textbf{in\_dividend}}\\
%      Pointer to a buffer containing the 64-bit unsigned integer dividend.
%      The buffer is not modified by the function.
%
%      This pointer may not be NULL or otherwise invalid.
%
%      This buffer may not be coincident with the \emph{*in\_quotient} or
%      \emph{*in\_divisor} buffers.  
%\item \emph{\textbf{in\_divisor}}\\
%      Pointer to a buffer containing the 32-bit unsigned integer divisor.
%      The buffer is not modified by the function.
%
%      This pointer may not be NULL or otherwise invalid.
%
%      This buffer may not be coincident with the \emph{*in\_quotient} or
%      \emph{*in\_dividend} buffers.  
%\end{list}
%\vspace{2.8ex}
%
%\noindent\textbf{OUTPUT}
%\begin{list}{}{\setlength{\leftmargin}{0.5in}\setlength{\itemindent}{-0.25in}\setlength{\topsep}{0.0in}\setlength{\partopsep}{0.0in}}
%\item \emph{\textbf{out\_quotient}}\\
%      Pointer to a buffer that will be assigned with the quotient.
%
%      This pointer may not be NULL or otherwise invalid.
%
%      This buffer may not be coincident with the \emph{*in\_dividend} or
%      \emph{*in\_divisor} buffers.  
%\end{list}
%\vspace{2.8ex}
%
%\noindent\textbf{INTERRUPT COMPATIBILITY}
%\begin{list}{}{\setlength{\leftmargin}{0.25in}\setlength{\topsep}{0.0in}}
%\item This function may be used from both non-ISR and ISR software.
%\item This function is thread-safe.
%\item This function does not ensure atomic access to the buffers pointed to, so
%      it is not thread-safe when processes in different threads use this
%      function to access the \emph{same} buffer(s).
%\end{list}
%\vspace{2.8ex}
%
%\noindent\textbf{EXECUTION TIME}
%\begin{list}{}{\setlength{\leftmargin}{0.25in}\setlength{\topsep}{0.0in}}
%\item TBD.
%\end{list}
%\vspace{2.8ex}
%
%\noindent\textbf{FUNCTION NAME MNEMONIC}
%\begin{list}{}{\setlength{\leftmargin}{0.25in}\setlength{\topsep}{0.0in}}
%\item \emph{U32}:     produces a UCU\_UINT32 result.
%      \emph{U64}:     accepts an unsigned 64-bit argument as input.
%      \emph{U32}:     accepts a UCU\_UINT32 argument as input.
%      \emph{Div}:     division.
%      \emph{Q}:       produces only a quotient (no remainder).
%      \emph{F}:       the quotient is rounded down, consistent with the traditional 
%                      \emph{floor($\cdot$)} function.
%\end{list}
%
%
%%%%%%%%%%%%%%%%%%%%%%%%%%%%%%%%%%%%%%%%%%%%%%%%%%%%%%%%%%%%%%%%%%%%%%%%%%%%%%%
%%%%%%%%%%%%%%%%%%%%%%%%%%%%%%%%%%%%%%%%%%%%%%%%%%%%%%%%%%%%%%%%%%%%%%%%%%%%%%%
%%%%%%%%%%%%%%%%%%%%%%%%%%%%%%%%%%%%%%%%%%%%%%%%%%%%%%%%%%%%%%%%%%%%%%%%%%%%%%%
\section{Unsigned Integer Squaring Functions}
\label{claf0:ssqf0}

TBD.


%%%%%%%%%%%%%%%%%%%%%%%%%%%%%%%%%%%%%%%%%%%%%%%%%%%%%%%%%%%%%%%%%%%%%%%%%%%%%%%
%%%%%%%%%%%%%%%%%%%%%%%%%%%%%%%%%%%%%%%%%%%%%%%%%%%%%%%%%%%%%%%%%%%%%%%%%%%%%%%
%%%%%%%%%%%%%%%%%%%%%%%%%%%%%%%%%%%%%%%%%%%%%%%%%%%%%%%%%%%%%%%%%%%%%%%%%%%%%%%
%\subsection[\emph{UcuLoU64U32SquareInPlaceRxn(\protect\mbox{\protect$\cdot$})}]
%           {\emph{UcuLoU64U32SquareInPlaceRxn(\protect\mbox{\protect\boldmath $\cdot$})}}
%\label{claf0:ssqf0:susf0}
%
%\index{UcuLoU64U32SquareInPlaceRxn()@\emph{UcuLoU64U32SquareInPlaceRxn($\cdot$)}}%
%
%\noindent\textbf{PROTOTYPE}
%\begin {list}{}{\setlength{\leftmargin}{0.25in}\setlength{\topsep}{0.0in}}
%\item
%\begin{verbatim}
%void UcuLoU64U32SquareInPlaceRxn( UCU_UNION64 *in_u64 )
%\end{verbatim}
%\end{list}
%\vspace{2.8ex}
%
%\noindent\textbf{SYNOPSIS}
%\begin{list}{}{\setlength{\leftmargin}{0.25in}\setlength{\topsep}{0.0in}}
%\item
%Calculates the square of a 32-bit unsigned integer, yielding a
%64-bit unsigned integer result.  The squaring is done ``in place'', meaning that the
%32-bit integer to be squared is placed right-aligned into
%a 64-bit buffer, and when the function returns, the same buffer contains the
%result.
%\end{list}
%\vspace{2.8ex}
%
%\noindent\textbf{INPUT}
%\begin{list}{}{\setlength{\leftmargin}{0.5in}\setlength{\itemindent}{-0.25in}\setlength{\topsep}{0.0in}\setlength{\partopsep}{0.0in}}
%\item \emph{\textbf{in\_u64}}\\
%      Pointer to a buffer containing the unsigned 32-bit integer whose square is to be calculated.
%      The integer should be right-aligned (placed in the least significant
%      bit positions of the buffer).
%
%      The 32 most significant bit positions of the buffer are ignored and
%      overwritten, and will not affect the calculation result.  These
%      bit positions do not need to be assigned prior to the function call.
%
%      This pointer may not be NULL or otherwise invalid.
%\end{list}
%\vspace{2.8ex}
%
%\noindent\textbf{OUTPUT}
%\begin{list}{}{\setlength{\leftmargin}{0.25in}\setlength{\topsep}{0.0in}}
%\item The buffer pointed to by \emph{in\_u64}
%      will contain the unsigned 64-bit integer square of the unsigned 32-bit integer
%      provided in the rightmost position in the buffer before the function call.
%
%      The original unsigned 32-bit integer provided is overwritten by the result.
%\end{list}
%\vspace{2.8ex}
%
%\noindent\textbf{INTERRUPT COMPATIBILITY}
%\begin{list}{}{\setlength{\leftmargin}{0.25in}\setlength{\topsep}{0.0in}}
%\item This function may be used from both non-ISR and ISR software.
%\item This function is thread-safe.
%\item This function does not ensure atomic access to \emph{*in\_u64}, so
%      it is not thread-safe when processes in different threads use this
%      function to access the \emph{same} buffer.
%\end{list}
%\vspace{2.8ex}
%
%\noindent\textbf{EXECUTION TIME}
%\begin{list}{}{\setlength{\leftmargin}{0.25in}\setlength{\topsep}{0.0in}}
%\item TBD.
%\end{list}
%\vspace{2.8ex}
%
%\noindent\textbf{FUNCTION NAME MNEMONIC}
%\begin{list}{}{\setlength{\leftmargin}{0.25in}\setlength{\topsep}{0.0in}}
%\item \emph{U64}:     produces an unsigned 64-bit result.
%      \emph{U32}:     accepts a UCU\_UINT32 as input.
%      \emph{Square}:  calculates the square.
%      \emph{InPlace}: calculates ``in place'' (uses the same input and output buffer).
%\end{list}
%
%
%%%%%%%%%%%%%%%%%%%%%%%%%%%%%%%%%%%%%%%%%%%%%%%%%%%%%%%%%%%%%%%%%%%%%%%%%%
\noindent\begin{figure}[!b]
\noindent\rule[-0.25in]{\textwidth}{1pt}
\begin{tiny}
\begin{verbatim}
$RCSfile: c_laf0.tex,v $
$Source: /home/dashley/cvsrep/uculib01/uculib01/doc/manual/c_laf0/c_laf0.tex,v $
$Revision: 1.7 $
$Author: dashley $
$Date: 2010/05/12 18:56:18 $
\end{verbatim}
\end{tiny}
\noindent\rule[0.25in]{\textwidth}{1pt}
\end{figure}

%%%%%%%%%%%%%%%%%%%%%%%%%%%%%%%%%%%%%%%%%%%%%%%%%%%%%%%%%%%%%%%%%%%%%%%%%%%%%%%
%$Log: c_laf0.tex,v $
%Revision 1.7  2010/05/12 18:56:18  dashley
%Removal of UcuLoU64U32SquareInPlaceRxn() function.
%
%Revision 1.6  2010/05/12 18:41:46  dashley
%Removal of UcuLoU32U64U32DivQFRxn() function.
%
%Revision 1.5  2010/04/15 17:09:38  dashley
%Addition of UcuLoU32U64U32DivQFRxn() function.
%
%Revision 1.4  2010/04/15 15:56:41  dashley
%Addition of UcuU64U32SquareInPlaceRxn() function.
%
%Revision 1.3  2010/01/28 21:18:32  dashley
%a)Chapter start quotes removed.
%b)Aesthetic comment line added at the bottom of most files.
%
%Revision 1.2  2010/01/24 05:37:27  dashley
%Addition and reorganization of content.
%
%Revision 1.1  2007/10/08 18:07:42  dtashley
%Initial checkin.
%
%End of $RCSfile: c_laf0.tex,v $.
%%%%%%%%%%%%%%%%%%%%%%%%%%%%%%%%%%%%%%%%%%%%%%%%%%%%%%%%%%%%%%%%%%%%%%%%%%%%%%%

