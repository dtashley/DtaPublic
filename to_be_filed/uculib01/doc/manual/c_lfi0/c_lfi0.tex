%$Header: /home/dashley/cvsrep/uculib01/uculib01/doc/manual/c_lfi0/c_lfi0.tex,v 1.10 2010/02/24 19:37:49 dashley Exp $

\chapter{Linear Filter Functions}

\label{clfi0}

%%%%%%%%%%%%%%%%%%%%%%%%%%%%%%%%%%%%%%%%%%%%%%%%%%%%%%%%%%%%%%%%%%%%%%%%%%%%%%%
%%%%%%%%%%%%%%%%%%%%%%%%%%%%%%%%%%%%%%%%%%%%%%%%%%%%%%%%%%%%%%%%%%%%%%%%%%%%%%%
%%%%%%%%%%%%%%%%%%%%%%%%%%%%%%%%%%%%%%%%%%%%%%%%%%%%%%%%%%%%%%%%%%%%%%%%%%%%%%%
\section{Introduction and Overview}
%Section tag:  iov0
\label{clfi0:siov0}

This chapter describes functions that approximate linear filters.
Here, \emph{linear} filter means a filter which satisfies
additive and homogeneity properties.

In order to satisfy the additive property, given two input signals $x_1(k)$ and $x_2(k)$, $\forall k>0$:

\begin{equation}
\label{eq:clfi0:siov0:01}
y(x_1(k) + x_2(k)) = y(x_1(k)) + y(x_2(k)).
\end{equation}

\noindent{}In order to satisfy the homogeneity property,

\begin{equation}
\label{eq:clfi0:siov0:02}
y(\alpha x_1(k)) = \alpha y(x_1(k)).
\end{equation}

\noindent{}A filter that does not satisfy both
(\ref{eq:clfi0:siov0:01}) and (\ref{eq:clfi0:siov0:02}) is by 
definition a non-linear filter (see Chapter
\ref{cnfi0}, p. \pageref{cnfi0}).

It is hard to find appropriate names for linear filters (especially if
filters differ from each other only in precision of arithmetic or the
size and range of input arguments), so for now the filters
are simply named using letters of the alphabet (\emph{Linear Filter A}, for 
example).


%%%%%%%%%%%%%%%%%%%%%%%%%%%%%%%%%%%%%%%%%%%%%%%%%%%%%%%%%%%%%%%%%%%%%%%%%%%%%%%
%%%%%%%%%%%%%%%%%%%%%%%%%%%%%%%%%%%%%%%%%%%%%%%%%%%%%%%%%%%%%%%%%%%%%%%%%%%%%%%
%%%%%%%%%%%%%%%%%%%%%%%%%%%%%%%%%%%%%%%%%%%%%%%%%%%%%%%%%%%%%%%%%%%%%%%%%%%%%%%
\section{Linear Filter A}
%Section tag:  lfa0
\label{clfi0:slfa0}


%%%%%%%%%%%%%%%%%%%%%%%%%%%%%%%%%%%%%%%%%%%%%%%%%%%%%%%%%%%%%%%%%%%%%%%%%%%%%%%
%%%%%%%%%%%%%%%%%%%%%%%%%%%%%%%%%%%%%%%%%%%%%%%%%%%%%%%%%%%%%%%%%%%%%%%%%%%%%%%
%%%%%%%%%%%%%%%%%%%%%%%%%%%%%%%%%%%%%%%%%%%%%%%%%%%%%%%%%%%%%%%%%%%%%%%%%%%%%%%
\subsection{Design}
%Subsection tag:  dsn0
\label{clfi0:slfa0:dsn0}

%%%%%%%%%%%%%%%%%%%%%%%%%%%%%%%%%%%%%%%%%%%%%%%%%%%%%%%%%%%%%%%%%%%%%%%%%%%%%%%%

\paragraph{General Design}

\emph{Linear Filter A} approximates the
discrete time difference equation

\begin{equation}
\label{eq:clfi0:slfa0:dsn0:01}
y_{k} = \frac{2^{16}-h}{2^{16}}x_k + \frac{h}{2^{16}}y_{k-1},
\end{equation}

\noindent{}where $x_k$ is the input and $y_k$ is the output
of the filter at discrete time $k$.
$h$ specifies the ``stiffness'' of the filter---a 
larger value of $h$ causes the filter to attenuate high frequencies more
aggressively.

%%%%%%%%%%%%%%%%%%%%%%%%%%%%%%%%%%%%%%%%%%%%%%%%%%%%%%%%%%%%%%%%%%%%%%%%%%%%%%%%

\paragraph{Design Equivalence}

In technical discussions, a filter of the design

\begin{equation}
\label{eq:clfi0:slfa0:dsn0:01a}
y_{k} = y_{k-1} + \frac{x_k - y_{k-1}}{c}
\end{equation}

\noindent{}was discussed, where $c$ specifies the
stiffness of the filter.

(\ref{eq:clfi0:slfa0:dsn0:01a}) can be rearranged to

\begin{equation}
\label{eq:clfi0:slfa0:dsn0:01b}
y_{k} = 
x_k \left( \frac{1}{c} \right)
+ y_{k-1} \left( {1-\frac{1}{c}} \right).
\end{equation}

\noindent{}It can be seen by comparing 
(\ref{eq:clfi0:slfa0:dsn0:01}) with 
(\ref{eq:clfi0:slfa0:dsn0:01b}) that the substitution 

\begin{equation}
\label{eq:clfi0:slfa0:dsn0:01c}
c = \frac{1}{1-h/2^{16}}
\end{equation}

\noindent{}equates (\ref{eq:clfi0:slfa0:dsn0:01}) with 
(\ref{eq:clfi0:slfa0:dsn0:01b}), i.e. the filter described  by 
(\ref{eq:clfi0:slfa0:dsn0:01}) and the filter 
described by (\ref{eq:clfi0:slfa0:dsn0:01b}) are 
mathematically equivalent.

(\ref{eq:clfi0:slfa0:dsn0:01c}) can also be solved for $h$:

\begin{equation}
\label{eq:clfi0:slfa0:dsn0:01d}
h = 2^{16}\left( 1 - \frac{1}{c} \right).
\end{equation}

%%%%%%%%%%%%%%%%%%%%%%%%%%%%%%%%%%%%%%%%%%%%%%%%%%%%%%%%%%%%%%%%%%%%%%%%%%%%%%%%

\paragraph{Time Constant}

The filter time constant $\tau$ can be expressed as

\begin{equation}
\label{eq:clfi0:slfa0:dsn0:02}
\tau = \frac{h \Delta t}{2^{16} - h} ,
\end{equation}

\noindent{}where $\Delta t$ is the discrete time interval.%
\footnote{(\ref{eq:clfi0:slfa0:dsn0:02}) comes from the
\emph{Discrete-time realization} section of the Wikipedia 
entry on low-pass filters \cite{bibref:w:wikipedialowpassfilter}---at this
time I do not
know how it is derived.}

Alternatively, (\ref{eq:clfi0:slfa0:dsn0:02}) may be solved for $h$:

\begin{equation}
\label{eq:clfi0:slfa0:dsn0:03}
h = \frac{2^{16} \tau}{\Delta t + \tau} .
\end{equation}

For example, at 100Hz ($\Delta t$=0.01 seconds) with a desired time constant
of 100ms ($\tau$=0.1 seconds), the required value of $h$ is:

\begin{equation}
\label{eq:clfi0:slfa0:dsn0:04}
h = \frac{(65536) (0.1)}{0.01 + 0.1} =  59578 .
\end{equation}

\begin{table}
\caption{Values of $h$ for Common Choices of $\Delta t$ and $\tau$}
\label{tbl:clfi0:slfa0:dsn0:01}
\begin{center}
\begin{tabular}{|c|c|c|c|}
\hline
               & \small{$f$=100Hz}       & \small{$f$=400Hz}         & \small{$f$=1000Hz}             \\
\small{$\tau$} & \small{$\Delta t$=0.01} & \small{$\Delta t$=0.0025} & \small{$\Delta t$=0.001}       \\
\hline
\hline
\small{0.01}   & \small{32,768}          & \small{52,429}            & \small{59,578}                 \\
\hline
\small{0.05}   & \small{54,613}          & \small{62,415}            & \small{64,251}                 \\
\hline
\small{0.1}    & \small{59,578}          & \small{63,938}            & \small{64,887}                 \\
\hline
\small{0.2}    & \small{62,415}          & \small{64,727}            & \small{65,210}                 \\
\hline
\small{0.3}    & \small{63,422}          & \small{64,994}            & \small{65,318}                 \\
\hline
\small{0.4}    & \small{63,938}          & \small{65,129}            & \small{65,373}                 \\
\hline
\small{0.5}    & \small{64,251}          & \small{65,210}            & \small{65,405}                 \\
\hline
\small{1.0}    & \small{64,887}          & \small{65,373}            & \small{65,470}                 \\
\hline
\end{tabular}
\end{center}
\end{table}

Table \ref{tbl:clfi0:slfa0:dsn0:01} provides the correct values of $h$ for common
choices of $\Delta t$ and $\tau$.

%%%%%%%%%%%%%%%%%%%%%%%%%%%%%%%%%%%%%%%%%%%%%%%%%%%%%%%%%%%%%%%%%%%%%%%%%%%%%%%%

\paragraph{Numeric Implementation}

The software uses large integers to implement the filter.  The filter state
($y_k$) is implemented as a 48-bit fixed-point integer with the radix point after the first 
16 bits.  An upper-case ``y'' ($Y_k$ rather than $y_k$) is used to denote the 
48-bit integer corresponding to $y_k$.

The implementation of the filter can be described by the following steps:

\begin{enumerate}
\item $Y_k$ is calculated:
      \begin{equation}
      \label{eq:clfi0:slfa0:dsn0:05}
      Y_k = \left\lfloor \frac{2^{32}(2^{16}-h)x_k + h Y_{k-1}}{2^{16}} \right\rfloor .
      \end{equation}
\item $y_k$ is calculated from $Y_k$:
      \begin{equation}
      \label{eq:clfi0:slfa0:dsn0:06}
      y_k = \left\lfloor \frac{Y_k + 2^{31}}{2^{32}} \right\rfloor .
      \end{equation}
\end{enumerate}

The steps are implemented differently than directly suggested by
(\ref{eq:clfi0:slfa0:dsn0:05}) and (\ref{eq:clfi0:slfa0:dsn0:06}).  For example, 
the division by $2^{16}$ and the floor function
in (\ref{eq:clfi0:slfa0:dsn0:05}) are implemented by discarding the last two bytes of
an integer multiplication.  As a second example, the 
addition of $2^{31}$ and the floor function in (\ref{eq:clfi0:slfa0:dsn0:06}) are implemented
by testing bit 31 of $Y_k$ and conditionally incrementing $y_k$ rather than by adding.

%%%%%%%%%%%%%%%%%%%%%%%%%%%%%%%%%%%%%%%%%%%%%%%%%%%%%%%%%%%%%%%%%%%%%%%%%%%%%%%%

\paragraph{Overflow}

It needs to be verified that the filter cannot state $Y_k$ cannot overflow
(exceed $2^{48}-1$).
In order overflow to occur, the numerator of
(\ref{eq:clfi0:slfa0:dsn0:05}) would have to reach $2^{64}$\@.  Guaranteeing that the
filter cannot overflow is equivalent to guaranteeing that $\forall h$,
$\forall x_k$, $\forall Y_{k-1}$:

\begin{equation}
\label{eq:clfi0:slfa0:dsn0:10}
2^{32}(2^{16}-h)x_k + h Y_{k-1} < 2^{64} .
\end{equation}

We can substitute the worst cases of $x_k = 2^{16}-1$ and $Y_{k-1} = 2^{48}-1$ into
(\ref{eq:clfi0:slfa0:dsn0:10}):

\begin{equation}
\label{eq:clfi0:slfa0:dsn0:11}
2^{48}(2^{16}-1) + (2^{32}-1)h < 2^{64} .
\end{equation}

The worst case of $h=2^{16}-1$ can be substituted into
(\ref{eq:clfi0:slfa0:dsn0:11}) to yield:

\begin{equation}
\label{eq:clfi0:slfa0:dsn0:12}
2^{64} - 2^{32} - 2^{16} + 1 < 2^{64} .
\end{equation}

By inspection, (\ref{eq:clfi0:slfa0:dsn0:12}) is met, so filter overflow is not
possible.

In addition to the analysis above, as part of the unit test plan, 
the filter was tested extensively to be
sure it could not roll over beyond $Y_k = 2^{48}-1$\@.  

%%%%%%%%%%%%%%%%%%%%%%%%%%%%%%%%%%%%%%%%%%%%%%%%%%%%%%%%%%%%%%%%%%%%%%%%%%%%%%%%

\paragraph{Convergence}

By \emph{convergence} of the filter I mean that, given enough time with
a constant $x_k$, $y_k$ will reach the same value.

Convergence wasn't formally analyzed (it should be), but there are three
clues that the filter will probably always converge.

\begin{enumerate}
\item In unit testing with typical values of $h$, the filter always converged.
\item In unit testing, one test performed (not in the test plan) was to set
      the initial value of the filter to 65534, the input to $x_k=65535$, and 
      $h=65535$.  The filter output did reach 65535.  The fact that the filter
      will close a gap of $\Delta x_k = 1$ with $h=65535$ implies that the filter will probably
      always converge.
\item In unit testing, a second test performed (not in the test plan) was to
      set the initial value of the filter to 65000, $h=65535$,
      $x_k = 65535$ and to measure the number of discrete time
      quanta required to reach $65535 - (65535-65000)/e = 65338.18$ (i.e.
      to measure the time constant).

      The number of time quanta required can be predicted from
      (\ref{eq:clfi0:slfa0:dsn0:02}) by removing the $\Delta t$ term:

      \begin{equation}
      \label{eq:clfi0:slfa0:dsn0:20}
      n = \frac{h}{2^{16} - h} = \frac{65535}{65536-65535} = 65535.
      \end{equation}

      In testing, this was the exact number of quanta required.  This suggests
      that the precision of the filter (32 bits after the radix point) is
      adequate even with the largest valid $h$.
\end{enumerate}

Convergence should still be formally analyzed.


%%%%%%%%%%%%%%%%%%%%%%%%%%%%%%%%%%%%%%%%%%%%%%%%%%%%%%%%%%%%%%%%%%%%%%%%%%%%%%%
%%%%%%%%%%%%%%%%%%%%%%%%%%%%%%%%%%%%%%%%%%%%%%%%%%%%%%%%%%%%%%%%%%%%%%%%%%%%%%%
%%%%%%%%%%%%%%%%%%%%%%%%%%%%%%%%%%%%%%%%%%%%%%%%%%%%%%%%%%%%%%%%%%%%%%%%%%%%%%%
\subsection[\emph{UcuLfU16FiltAInitRxn(\protect\mbox{\protect$\cdot$})}]
           {\emph{UcuLfU16FiltAInitRxn(\protect\mbox{\protect\boldmath $\cdot$})}}
%Section tag:  lcp0
\label{clfi0:slai0}

\index{UcuLfU16FiltAInitRxn()@\emph{UcuLfU16FiltAInitRxn($\cdot$)}}%

\noindent\textbf{PROTOTYPE}
\begin {list}{}{\setlength{\leftmargin}{0.25in}\setlength{\topsep}{0.0in}}
\item
\begin{verbatim}
void UcuLfU16FiltAInitRxn(UCU_UNION48 *in_fs,
                          UCU_UINT16   in_x_k_initial)
\end{verbatim}
\end{list}
\vspace{2.8ex}

\noindent\textbf{SYNOPSIS}
\begin{list}{}{\setlength{\leftmargin}{0.25in}\setlength{\topsep}{0.0in}}
\item Initializes the state \texttt{in\_fs} of a Linear Filter A
      to the passed value \texttt{in\_x\_k\_initial}.  This consists of
      setting the most significant two bytes of the state to \texttt{in\_x\_k\_initial}
      and the least significant four bytes to zero.
\item This function is typically used to either:
      \begin{itemize}
      \item Set the filter state to be the same as the input value (for example,
            at software startup).
      \item Eliminate filtering lag on a one-time basis and cause the filter output to track a step jump
            in the filter input (for example, when the software makes a major mode change
            or wishes to produce a step jump in output).
      \end{itemize}
\end{list}
\vspace{2.8ex}

\noindent\textbf{INPUTS}
\begin{list}{}{\setlength{\leftmargin}{0.5in}\setlength{\itemindent}{-0.25in}\setlength{\topsep}{0.0in}\setlength{\partopsep}{0.0in}}
\item \emph{\textbf{in\_fs}}\\
      The filter state.
\item \emph{\textbf{in\_x\_k\_initial}}\\
      The value to initialize the filter state to.
\end{list}
\vspace{2.8ex}

\noindent\textbf{OUTPUTS}
\begin{list}{}{\setlength{\leftmargin}{0.25in}\setlength{\topsep}{0.0in}}
\item None.
\end{list}
\vspace{2.8ex}

\noindent\textbf{INTERRUPT CONSIDERATIONS}
\begin{list}{}{\setlength{\leftmargin}{0.25in}\setlength{\topsep}{0.0in}}
\item This function can be used from both ISR and non-ISR software.
\item This function does not ensure atomic access to \texttt{in\_fs}, so it is
      not thread-safe when processes in different threads use this function
      to access the \emph{same} filter state.
\end{list}
\vspace{2.8ex}

\noindent\textbf{EXECUTION TIME}
\begin{list}{}{\setlength{\leftmargin}{0.25in}\setlength{\topsep}{0.0in}}
\item TBD.
\end{list}
\vspace{2.8ex}

\noindent\textbf{FUNCTION NAME MNEMONIC}
\begin{list}{}{\setlength{\leftmargin}{0.25in}\setlength{\topsep}{0.0in}}
\item \emph{U16}:    the data type to be filtered is UCU\_UINT16.
      \emph{FiltA}:  Linear Filter A.
      \emph{Init}:   initialize.
\end{list}

%%%%%%%%%%%%%%%%%%%%%%%%%%%%%%%%%%%%%%%%%%%%%%%%%%%%%%%%%%%%%%%%%%%%%%%%%%%%%%%
%%%%%%%%%%%%%%%%%%%%%%%%%%%%%%%%%%%%%%%%%%%%%%%%%%%%%%%%%%%%%%%%%%%%%%%%%%%%%%%
%%%%%%%%%%%%%%%%%%%%%%%%%%%%%%%%%%%%%%%%%%%%%%%%%%%%%%%%%%%%%%%%%%%%%%%%%%%%%%%
\subsection[\emph{UcuLfU16FiltAFiltRxn(\protect\mbox{\protect$\cdot$})}]
           {\emph{UcuLfU16FiltAFiltRxn(\protect\mbox{\protect\boldmath $\cdot$})}}
%Section tag:  lcp0
\label{clfi0:slaf0}

\index{UcuLfU16FiltAFiltRxn()@\emph{UcuLfU16FiltAFiltRxn($\cdot$)}}%

\noindent\textbf{PROTOTYPE}
\begin {list}{}{\setlength{\leftmargin}{0.25in}\setlength{\topsep}{0.0in}}
\item
\begin{verbatim}
UCU_UINT16 UcuLfU16FiltAFiltRxn(
                               UCU_UNION48 *in_fs, 
                               UCU_UINT16   in_x_k, 
                               UCU_UINT16   in_h
                               )
\end{verbatim}
\end{list}
\vspace{2.8ex}

\noindent\textbf{SYNOPSIS}
\begin{list}{}{\setlength{\leftmargin}{0.25in}\setlength{\topsep}{0.0in}}
\item Filters the data input \texttt{in\_x\_k} using the specified
      time constant specifier \texttt{in\_h} and filter state
      \texttt{in\_fs}, and returns the filtered value.
      The filtering applied is defined by 
      (\ref{eq:clfi0:slfa0:dsn0:05}) and (\ref{eq:clfi0:slfa0:dsn0:06}).
\end{list}
\vspace{2.8ex}

\noindent\textbf{INPUTS}
\begin{list}{}{\setlength{\leftmargin}{0.5in}\setlength{\itemindent}{-0.25in}\setlength{\topsep}{0.0in}\setlength{\partopsep}{0.0in}}
\item \emph{\textbf{in\_fs}}\\
      The filter state.
\item \emph{\textbf{in\_x\_k}}\\
      The input value to be filtered (i.e. $x_k$).
\item \emph{\textbf{in\_h}}\\
      The time constant of the filter as defined by (\ref{eq:clfi0:slfa0:dsn0:03}).
\end{list}
\vspace{2.8ex}

\noindent\textbf{OUTPUTS}
\begin{list}{}{\setlength{\leftmargin}{0.25in}\setlength{\topsep}{0.0in}}
\item The filtered value, as defined by 
      (\ref{eq:clfi0:slfa0:dsn0:05}) and (\ref{eq:clfi0:slfa0:dsn0:06}).
\end{list}
\vspace{2.8ex}

\noindent\textbf{USAGE NOTES}
\begin{list}{}{\setlength{\leftmargin}{0.25in}\setlength{\topsep}{0.0in}}
\item The filter state is a fixed-point number, and this function
      implements a simple difference equation.  It is safe to modify
      $h$ on the fly without unexpected effects (although this would be very 
      rare, as most applications would always apply the same time constant to
      a given filter state).  There is no state retained beyond the filter state block
      and no requirement
      that $h$ be the same value on every function call involving the same filter
      state block.
\end{list}
\vspace{2.8ex}

\noindent\textbf{INTERRUPT CONSIDERATIONS}
\begin{list}{}{\setlength{\leftmargin}{0.25in}\setlength{\topsep}{0.0in}}
\item This function can be used from both ISR and non-ISR software.
\item This function does not ensure atomic access to \texttt{in\_fs}, so it is
      not thread-safe when processes in different threads use this function
      to access the \emph{same} filter state.
\end{list}
\vspace{2.8ex}

\noindent\textbf{EXECUTION TIME}
\begin{list}{}{\setlength{\leftmargin}{0.25in}\setlength{\topsep}{0.0in}}
\item TBD.
\end{list}
\vspace{2.8ex}

\noindent\textbf{FUNCTION NAME MNEMONIC}
\begin{list}{}{\setlength{\leftmargin}{0.25in}\setlength{\topsep}{0.0in}}
\item \emph{U16}:    the data type to be filtered is UCU\_UINT16.
      \emph{FiltA}:  Linear Filter A.
      \emph{Filt}:   filter.
\end{list}


%%%%%%%%%%%%%%%%%%%%%%%%%%%%%%%%%%%%%%%%%%%%%%%%%%%%%%%%%%%%%%%%%%%%%%%%%%
\noindent\begin{figure}[!b]
\noindent\rule[-0.25in]{\textwidth}{1pt}
\begin{tiny}
\begin{verbatim}
$RCSfile: c_lfi0.tex,v $
$Source: /home/dashley/cvsrep/uculib01/uculib01/doc/manual/c_lfi0/c_lfi0.tex,v $
$Revision: 1.10 $
$Author: dashley $
$Date: 2010/02/24 19:37:49 $
\end{verbatim}
\end{tiny}
\noindent\rule[0.25in]{\textwidth}{1pt}
\end{figure}

%%%%%%%%%%%%%%%%%%%%%%%%%%%%%%%%%%%%%%%%%%%%%%%%%%%%%%%%%%%%%%%%%%%%%%%%%%%%%%%
%$Log: c_lfi0.tex,v $
%Revision 1.10  2010/02/24 19:37:49  dashley
%Linear Filter A documentation corrected and enhanced.
%
%Revision 1.9  2010/01/28 21:18:32  dashley
%a)Chapter start quotes removed.
%b)Aesthetic comment line added at the bottom of most files.
%
%Revision 1.8  2010/01/28 02:45:46  dashley
%Information about Linear Filter A completed.
%
%Revision 1.7  2010/01/27 16:08:53  dashley
%Formatting difficulties corrected.
%
%Revision 1.6  2010/01/27 16:04:16  dashley
%Formatting difficulty corrections.
%
%Revision 1.5  2010/01/27 00:26:33  dashley
%Name change of Linear Filter A functions.
%
%Revision 1.4  2010/01/26 21:49:44  dashley
%Edits.
%
%Revision 1.3  2010/01/26 21:10:26  dashley
%Linear Filter A design section completed.
%
%Revision 1.2  2010/01/24 05:40:22  dashley
%Minor title changes.
%
%Revision 1.1  2010/01/24 05:38:26  dashley
%Initial checkin.
%End of $RCSfile: c_lfi0.tex,v $.
%%%%%%%%%%%%%%%%%%%%%%%%%%%%%%%%%%%%%%%%%%%%%%%%%%%%%%%%%%%%%%%%%%%%%%%%%%%%%%%

