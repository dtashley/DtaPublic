%$Header: /home/dashley/cvsrep/uculib01/uculib01/doc/manual/comps/worknenv.tex,v 1.5 2010/01/27 16:08:53 dashley Exp $
%
%%%%%%%%%%%%%%%%%%%%%%%%%%%%%%%%%%%%%%%%%%%%%%%%%%%%%%%%%%%%%%%%%%%%%%%%%%%%%
%% SEPARATORS
%%%%%%%%%%%%%%%%%%%%%%%%%%%%%%%%%%%%%%%%%%%%%%%%%%%%%%%%%%%%%%%%%%%%%%%%%%%%%
%%
%Digit separation distance for denoting thousands, etc.  Choosing space rather than comma.
\newcommand{\vworkthousandsdigsepinmathmode}{\;}
%%
%%%%%%%%%%%%%%%%%%%%%%%%%%%%%%%%%%%%%%%%%%%%%%%%%%%%%%%%%%%%%%%%%%%%%%%%%%%%%
%% CHAPTER QUOTES
%%%%%%%%%%%%%%%%%%%%%%%%%%%%%%%%%%%%%%%%%%%%%%%%%%%%%%%%%%%%%%%%%%%%%%%%%%%%%
%%
%Quote which begins each chapter.
\newcommand{\beginchapterquote}[2]{\textbf{#1}\begin{flushright}\emph{---#2}\end{flushright}}

%Source when included with quote which begins each chapter.
\newcommand{\chapquoteshortsrc}[1]{\begin{flushright}\emph{---#1}\end{flushright}}
%%
%%%%%%%%%%%%%%%%%%%%%%%%%%%%%%%%%%%%%%%%%%%%%%%%%%%%%%%%%%%%%%%%%%%%%%%%%%%%%
%% EXAMPLES
%%%%%%%%%%%%%%%%%%%%%%%%%%%%%%%%%%%%%%%%%%%%%%%%%%%%%%%%%%%%%%%%%%%%%%%%%%%%%
%%
%General-purpose example start.  Used to generate the beginning of the example
%only, should enclose only an optional label.
\newtheorem{vworkgenexample}{Example}

%General-purpose example solution start.
\newcommand{\vworkgenexamplesolutionhead}{\noindent\textbf{Solution: }}

%General-purpose example solution body.  This is the environment in which
%the example's solution is stated.  For the present time, there are no
%changes to the text.
\newenvironment{vworkgenexamplesolutionbody}{\noindent\textbf{Solution: }}{}

%Define a new counter used to hold the example number within a chapter.
%\newcounter{vworkexamplecounter}[chapter]

%Alternate numbering for examples.
%\renewcommand{\thevworkexamplecounter}{\thechapter.\arabic{vworkexamplecounter}}

%Define the marking which delimits the start of an example.  For now, this
%is a little bit of space and a horizontal bar.
\newcommand{\vworkexampleheader}{\par\nopagebreak\vspace{0.01in}%
                                 \noindent\rule{\textwidth}{1pt}%
                                 \par\nopagebreak}

%Define the marking which delimits the end of an example.  For now, this is
%a horizontal line.  An example must be "footed" manually because there are
%so many variations on what can be in an example.
\newcommand{\vworkexamplefooter}{\par\nopagebreak%
                                 \vspace{0.01in}\noindent\rule{\textwidth}{1pt}%
                                 \par\nopagebreak}

%Define a new environment to hold the body of an example, and the numbering of an
%example.
\newenvironment{vworkexamplestatement}%
               {\vworkexampleheader\noindent%
               \setlength{\parindent}{0em}%
               \setlength{\parskip}{1ex}%
               \refstepcounter{vworktheoremcounter}%
               \textbf{Example \nopagebreak[2]\thevworktheoremcounter{}: }}{\par}

%Define a new environment to hold a "solution" or "remarks" block within an example.
%The presence or absence of a colon is something that may change, so better to code
%that in here.
\newenvironment{vworkexampleparsection}[1]{\par\noindent%
               \setlength{\parindent}{0em}%
               \setlength{\parskip}{1ex}%
               \textbf{#1:\nopagebreak[2] }}{\par}
%%
%%%%%%%%%%%%%%%%%%%%%%%%%%%%%%%%%%%%%%%%%%%%%%%%%%%%%%%%%%%%%%%%%%%%%%%%%%%%%
%% DEFINITIONS
%%%%%%%%%%%%%%%%%%%%%%%%%%%%%%%%%%%%%%%%%%%%%%%%%%%%%%%%%%%%%%%%%%%%%%%%%%%%%
%%
%Counter for "artifical" theorems.  In this context, "artificial" means that the
%built-in theorem environment is not used.
%\newcounter{vworkdefinitioncounter}[chapter]

%Alternate numbering for definitions.
%\renewcommand{\thevworkdefinitioncounter}{\thechapter.\arabic{vworkdefinitioncounter}}

%Define the markings which delimit the start and ends of a definition.  For now, these
%are NULL.  Later, I may define something else--a little extra space or a line.
\newcommand{\vworkdefinitionheader}{\par\nopagebreak%
                                    \vspace{0.01in}\noindent\rule{\textwidth}{1pt}%
                                    \par\nopagebreak}
\newcommand{\vworkdefinitionfooter}{\par\nopagebreak%
                                    \vspace{0.01in}\noindent\rule{\textwidth}{1pt}%
                                    \par\nopagebreak}

%Environment to hold definitions.  This was defined (rather than using the built-in
%environment) because I cannot stand having a number without a colon.
\newenvironment{vworkdefinitionstatement}%
               {\vworkdefinitionheader\noindent%
               \setlength{\parindent}{0em}%
               \setlength{\parskip}{1ex}%
               \refstepcounter{vworktheoremcounter}%
               \textbf{Definition \nopagebreak[2]\thevworktheoremcounter{}: }}{\par}

%Environment to begin a definition that also has descriptive text.
\newenvironment{vworkdefinitionstatementpar}[1]%
               {\vworkdefinitionheader\noindent%
               \setlength{\parindent}{0em}%
               \setlength{\parskip}{1ex}%
               \refstepcounter{vworktheoremcounter}%
               \textbf{Definition \nopagebreak[2]\thevworktheoremcounter{} (#1): }}{\par}

%Define a new environment to hold a "solution" or "remarks" block within a definition.
%The presence or absence of a colon is something that may change, so better to code
%that in here.
\newenvironment{vworkdefinitionparsection}[1]{\par\noindent%
               \setlength{\parindent}{0em}%
               \setlength{\parskip}{1ex}%
               \textbf{#1:\nopagebreak[2] }}{\par}
%%
%%%%%%%%%%%%%%%%%%%%%%%%%%%%%%%%%%%%%%%%%%%%%%%%%%%%%%%%%%%%%%%%%%%%%%%%%%%%%
%% LEMMAS
%%%%%%%%%%%%%%%%%%%%%%%%%%%%%%%%%%%%%%%%%%%%%%%%%%%%%%%%%%%%%%%%%%%%%%%%%%%%%
%%
%% Because lemmas and theorems are so similar, decided that they should
%% use the same counters.  I think this makes it easier for readers, so they
%% don't have to separate the two out when searching.
%%
%\newcounter{vworktheoremcounter}[chapter]

%Alternate numbering for lemmas.
%\renewcommand{\thevworktheoremcounter}{\thechapter.\arabic{vworktheoremcounter}}

%Define the markings which delimit the start and ends of a lemma.  For now, these
%are NULL.  Later, I may define something else--a little extra space or a line.
\newcommand{\vworklemmaheader}{\par\nopagebreak%
                               \vspace{0.01in}\noindent\rule{\textwidth}{1pt}%
                               \par\nopagebreak}
\newcommand{\vworklemmafooter}{\par\nopagebreak%
                               \vspace{0.01in}\noindent\rule{\textwidth}{1pt}%
                               \par\nopagebreak}

%Environment to hold lemmas.  This was defined (rather than using the built-in
%environment) because I cannot stand having a number without a colon.
\newenvironment{vworklemmastatement}%
               {\vworklemmaheader\noindent%
               \setlength{\parindent}{0em}%
               \setlength{\parskip}{1ex}%
               \refstepcounter{vworktheoremcounter}%
               \textbf{Lemma \nopagebreak[2]\thevworktheoremcounter{}: }}{\par}

%Environment to begin a lemma that also has descriptive text.
\newenvironment{vworklemmastatementpar}[1]%
               {\vworklemmaheader\noindent%
               \setlength{\parindent}{0em}%
               \setlength{\parskip}{1ex}%
               \refstepcounter{vworktheoremcounter}%
               \textbf{Lemma \nopagebreak[2]\thevworktheoremcounter{} (#1): }}{\par}

%Define a new environment to hold the proof of a lemma.
\newenvironment{vworklemmaproof}%
               {\par\noindent%
               \setlength{\parindent}{0em}%
               \setlength{\parskip}{1ex}%
               \textbf{Proof:\nopagebreak[2] }}{\hfill\rule{1.5ex}{1.5ex}\par}

%Define a new environment to hold a "solution" or "remarks" block within a lemma.
%The presence or absence of a colon is something that may change, so better to code
%that in here.
\newenvironment{vworklemmaparsection}[1]{\par\noindent%
               \setlength{\parindent}{0em}%
               \setlength{\parskip}{1ex}%
               \textbf{#1:\nopagebreak[2] }}{\par}
%%
%%%%%%%%%%%%%%%%%%%%%%%%%%%%%%%%%%%%%%%%%%%%%%%%%%%%%%%%%%%%%%%%%%%%%%%%%%%%%
%% THEOREMS
%%%%%%%%%%%%%%%%%%%%%%%%%%%%%%%%%%%%%%%%%%%%%%%%%%%%%%%%%%%%%%%%%%%%%%%%%%%%%
%%
%Counter for "artifical" theorems.  In this context, "artificial" means that the
%built-in theorem environment is not used.
\newcounter{vworktheoremcounter}[chapter]

%Alternate numbering for theorems.
\renewcommand{\thevworktheoremcounter}{\thechapter.\arabic{vworktheoremcounter}}

%Define the markings which delimit the start and ends of a theorem.  For now, these
%are NULL.  Later, I may define something else--a little extra space or a line.
\newcommand{\vworktheoremheader}{\par\nopagebreak%
                                 \vspace{0.01in}\noindent\rule{\textwidth}{1pt}%
                                 \par\nopagebreak}
\newcommand{\vworktheoremfooter}{\par\nopagebreak%
                                 \vspace{0.01in}\noindent\rule{\textwidth}{1pt}%
                                 \par\nopagebreak}

%Environment to hold theorems.  This was defined (rather than using the built-in
%environment) because I cannot stand having a number without a colon.
\newenvironment{vworktheoremstatement}%
               {\vworktheoremheader\noindent%
               \setlength{\parindent}{0em}%
               \setlength{\parskip}{1ex}%
               \refstepcounter{vworktheoremcounter}%
               \textbf{Theorem \nopagebreak[2]\thevworktheoremcounter{}: }}{\par}

%Environment to begin a theorem that also has descriptive text.
\newenvironment{vworktheoremstatementpar}[1]%
               {\vworktheoremheader\noindent%
               \setlength{\parindent}{0em}%
               \setlength{\parskip}{1ex}%
               \refstepcounter{vworktheoremcounter}%
               \textbf{Theorem \nopagebreak[2]\thevworktheoremcounter{} (#1): }}{\par}

%Define a new environment to hold the proof of a theorem.
\newenvironment{vworktheoremproof}%
               {\par\noindent%
               \setlength{\parindent}{0em}%
               \setlength{\parskip}{1ex}%
               \textbf{Proof:\nopagebreak[2] }}{\hfill\rule{1.5ex}{1.5ex}\par}

%Define a new environment to hold a "solution" or "remarks" block within a theorem.
%The presence or absence of a colon is something that may change, so better to code
%that in here.
\newenvironment{vworktheoremparsection}[1]{\par\noindent%
               \setlength{\parindent}{0em}%
               \setlength{\parskip}{1ex}%
               \textbf{#1:\nopagebreak[2] }}{\par}
%%
%%%%%%%%%%%%%%%%%%%%%%%%%%%%%%%%%%%%%%%%%%%%%%%%%%%%%%%%%%%%%%%%%%%%%%%%%%%%%
%% ALGORITHMS
%%%%%%%%%%%%%%%%%%%%%%%%%%%%%%%%%%%%%%%%%%%%%%%%%%%%%%%%%%%%%%%%%%%%%%%%%%%%%
%%
%Counter for "artifical" theorems.  In this context, "artificial" means that the
%built-in theorem environment is not used.
%\newcounter{vworktheoremcounter}[chapter]

%Alternate numbering for theorems.
%\renewcommand{\thevworktheoremcounter}{\thechapter.\arabic{vworktheoremcounter}}

%Define the markings which delimit the start and ends of an algorithm.  For now, these
%are NULL.  Later, I may define something else--a little extra space or a line.
\newcommand{\vworkalgorithmheader}{\par\nopagebreak%
                                 \vspace{0.01in}\noindent\rule{\textwidth}{1pt}%
                                 \par\nopagebreak}
\newcommand{\vworkalgorithmfooter}{\par\nopagebreak%
                                 \vspace{0.01in}\noindent\rule{\textwidth}{1pt}%
                                 \par\nopagebreak}

%Environment to hold algorithms.  This was defined (rather than using the built-in
%environment) because I cannot stand having a number without a colon.
\newenvironment{vworkalgorithmstatement}%
               {\vworkalgorithmheader\noindent%
               \setlength{\parindent}{0em}%
               \setlength{\parskip}{1ex}%
               \refstepcounter{vworktheoremcounter}%
               \textbf{Algorithm \nopagebreak[2]\thevworktheoremcounter{}: }}{\par}

%Environment to begin an algorithm that also has descriptive text.
\newenvironment{vworkalgorithmstatementpar}[1]%
               {\vworkalgorithmheader\noindent%
               \setlength{\parindent}{0em}%
               \setlength{\parskip}{1ex}%
               \refstepcounter{vworktheoremcounter}%
               \textbf{Algorithm \nopagebreak[2]\thevworktheoremcounter{} (#1): }}{\par}

%Define a new environment to hold the proof of an algorithm.
\newenvironment{vworkalgorithmproof}%
               {\par\noindent%
               \setlength{\parindent}{0em}%
               \setlength{\parskip}{1ex}%
               \textbf{Proof:\nopagebreak[2] }}{\hfill\rule{1.5ex}{1.5ex}\par}

%Define a new environment to hold a "solution" or "remarks" block within an algorithm.
%The presence or absence of a colon is something that may change, so better to code
%that in here.
\newenvironment{vworkalgorithmparsection}[1]{\par\noindent%
               \setlength{\parindent}{0em}%
               \setlength{\parskip}{1ex}%
               \textbf{#1:\nopagebreak[2] }}{\par}
%The "itemize" environment defined by the book style files didn't seem to have
%quite the right appearance for presenting our algorithms.  For this reason, the
%following environments were defined.  Level 0 is flush left with bullets, Level 1
%is slightly more indented, etc.
\newenvironment{alglvl0}{\begin{list}
               {$\bullet$}{\setlength{\labelwidth}{3mm}\setlength{\leftmargin}{6mm}}}
               {\end{list}}
\newenvironment{alglvl1}{\begin{list}
               {$\bullet$}{\setlength{\labelwidth}{3mm}\setlength{\leftmargin}{6mm}}}
               {\end{list}}
\newenvironment{alglvl2}{\begin{list}
               {$\bullet$}{\setlength{\labelwidth}{3mm}\setlength{\leftmargin}{6mm}}}
               {\end{list}}
%The environments above have been obsoleted.
\newcounter{alglvl0ctr}
\newenvironment{algblvl0}{\begin{list}
               {\textbf{[\arabic{alglvl0ctr}]}}{\usecounter{alglvl0ctr}%
               \setlength{\labelwidth}{6mm}\setlength{\leftmargin}{9mm}}}
               {\end{list}}
%%
%%%%%%%%%%%%%%%%%%%%%%%%%%%%%%%%%%%%%%%%%%%%%%%%%%%%%%%%%%%%%%%%%%%%%%%%%%%%%
%% EXERCISES
%%%%%%%%%%%%%%%%%%%%%%%%%%%%%%%%%%%%%%%%%%%%%%%%%%%%%%%%%%%%%%%%%%%%%%%%%%%%%
%%
%Counter for exercises at the end of each chapter.
\newcounter{vworkexercisecounter}[chapter]

%We'd like to number an exercise slightly differently, using the chapter number
%followed by "." followed by the equation number.
\renewcommand{\thevworkexercisecounter}{\thechapter.\arabic{vworkexercisecounter}}

%Define the markings which delimit the start and ends of an exercise.  For now, these
%are NULL.  Later, I may define something else--a little extra space or a line.
\newcommand{\vworkexerciseheader}{\par\vspace{0.01in}\par}
\newcommand{\vworkexercisefooter}{\par\vspace{0.01in}\par}

%Environment to hold exercises.
\newenvironment{vworkexercisestatement}%
               {\vworkexerciseheader\noindent%
               \setlength{\parindent}{0em}%
               \setlength{\parskip}{1ex}%
               \refstepcounter{vworkexercisecounter}%
               \textbf{[\thevworkexercisecounter{}]}\nopagebreak[2] }{\par}

%Define a new environment to hold a "remarks" block within an exercise.
%The presence or absence of a colon is something that may change, so better to code
%that in here.
\newenvironment{vworkexerciseparsection}[1]{\par\noindent%
               \setlength{\parindent}{0em}%
               \setlength{\parskip}{1ex}%
               \textbf{#1: }}{\par}
%%
%%%%%%%%%%%%%%%%%%%%%%%%%%%%%%%%%%%%%%%%%%%%%%%%%%%%%%%%%%%%%%%%%%%%%%%%%%%%%
%% LESSONS LEARNED
%%%%%%%%%%%%%%%%%%%%%%%%%%%%%%%%%%%%%%%%%%%%%%%%%%%%%%%%%%%%%%%%%%%%%%%%%%%%%
%%
%Counter for lessons learned.
\newcounter{vworklessonslearnedcounter}[chapter]

%Alternate numbering for lessons learned.
\renewcommand{\thevworklessonslearnedcounter}{\thechapter.\arabic{vworklessonslearnedcounter}}

%Define the markings which delimit the start and ends of a lesson learned.
\newcommand{\vworklessonslearnedheader}{\par\nopagebreak%
                                 \vspace{0.01in}\noindent\rule{\textwidth}{1pt}%
                                 \par\nopagebreak}
\newcommand{\vworklessonslearnedfooter}{\par\nopagebreak%
                                 \vspace{0.01in}\noindent\rule{\textwidth}{1pt}%
                                 \par\nopagebreak}

%Environment to hold lessons learned.
\newenvironment{vworklessonslearnedstatement}%
               {\vworklessonslearnedheader\noindent%
               \setlength{\parindent}{0em}%
               \setlength{\parskip}{1ex}%
               \refstepcounter{vworklessonslearnedcounter}%
               \textbf{Lesson Learned \nopagebreak[2]\thevworklessonslearnedcounter{}: }}{\par}
%%
%%%%%%%%%%%%%%%%%%%%%%%%%%%%%%%%%%%%%%%%%%%%%%%%%%%%%%%%%%%%%%%%%%%%%%%%%%%%%
%% QUOTES
%%%%%%%%%%%%%%%%%%%%%%%%%%%%%%%%%%%%%%%%%%%%%%%%%%%%%%%%%%%%%%%%%%%%%%%%%%%%%
%%
%Environment to hold quotes.
\newenvironment{indentedquote}{\begin{quote}}{\end{quote}}
%%
%%%%%%%%%%%%%%%%%%%%%%%%%%%%%%%%%%%%%%%%%%%%%%%%%%%%%%%%%%%%%%%%%%%%%%%%%%%%%
%% EXERCISE SOLUTIONS
%%%%%%%%%%%%%%%%%%%%%%%%%%%%%%%%%%%%%%%%%%%%%%%%%%%%%%%%%%%%%%%%%%%%%%%%%%%%%
%%
%Environment To Hold A Solution To An Exercise
\newenvironment{vworkexercisesolution}[1]{\par\noindent%
               \setlength{\parindent}{0em}%
               \setlength{\parskip}{1ex}%
               \textbf{Solution To Exercise #1}\par}{\par}

%Command which begins section of solutions for each chapter, command which
%separates solutions, and command which ends solutions in each chapter.
\newcommand{\vworkexercisechapterheader}{\par\nopagebreak\vspace{0.01in}%
                                 \noindent\rule{\textwidth}{1pt}%
                                 \par\nopagebreak}
\newcommand{\vworkexerciseseparator}{\par\nopagebreak\vspace{0.01in}%
                                 \noindent\rule{\textwidth}{1pt}%
                                 \par\nopagebreak}
\newcommand{\vworkexercisechapterfooter}{\par\nopagebreak%
                                 \vspace{0.01in}\noindent\rule{\textwidth}{1pt}%
                                 \par\nopagebreak}
%%
%%%%%%%%%%%%%%%%%%%%%%%%%%%%%%%%%%%%%%%%%%%%%%%%%%%%%%%%%%%%%%%%%%%%%%%%%%%%%
%% GLOSSARY OF TERMS
%%%%%%%%%%%%%%%%%%%%%%%%%%%%%%%%%%%%%%%%%%%%%%%%%%%%%%%%%%%%%%%%%%%%%%%%%%%%%
%5
%The following environment is for the glossary of terms at the end.
\newenvironment{vworktermglossaryenum}{\begin{list}
               {}{\setlength{\labelwidth}{0mm}
                  \setlength{\leftmargin}{4mm}
                  \setlength{\itemindent}{-4mm}
                  \setlength{\parsep}{0.85mm}}}
               {\end{list}}
%%
%%%%%%%%%%%%%%%%%%%%%%%%%%%%%%%%%%%%%%%%%%%%%%%%%%%%%%%%%%%%%%%%%%%%%%%%%%%%%
%% GLOSSARY OF MATHEMATICAL TERMS
%%%%%%%%%%%%%%%%%%%%%%%%%%%%%%%%%%%%%%%%%%%%%%%%%%%%%%%%%%%%%%%%%%%%%%%%%%%%%
%%
%The following environment is for the glossary of mathematical terms at the end.
\newenvironment{vworkmathtermglossaryenum}{\begin{list}
               {}{\setlength{\labelwidth}{0mm}
                  \setlength{\leftmargin}{4mm}
                  \setlength{\itemindent}{-4mm}
                  \setlength{\parsep}{0.85mm}}}
               {\end{list}}

%%%%%%%%%%%%%%%%%%%%%%%%%%%%%%%%%%%%%%%%%%%%%%%%%%%%%%%%%%%%%%%%%%%%%%%%%%%%%%%%%%%%%%%%%%
%%%%%%%%%%%%%%%%%%%%%%%%%%%%%%%%%%%%%%%%%%%%%%%%%%%%%%%%%%%%%%%%%%%%%%%%%%%%%%%%%%%%%%%%%%
% TCL COMMAND DOCUMENTATION
%
%Environment to hold the "NAME" portion of a Tcl command description.
\newenvironment{tclcommandname}[1]{\noindent\textbf{NAME}\vspace{-1.0ex}\begin{list}%
               {}{\setlength{\leftmargin}{0.25in}}\item\textbf{#1}---}{\end{list}}

%Command to display a synopsis line inside the synopsis environment.
\newcommand{\tclcommandsynopsisline}[2]{\par\textbf{#1} \textit{#2}\par}

%Environment to hold the "SYNOPSIS" portion of a Tcl command description.
\newenvironment{tclcommandsynopsis}{\noindent\textbf{SYNOPSIS}\vspace{-1.0ex}\begin{list}%
               {}{\setlength{\leftmargin}{0.25in}}\item{}}{\end{list}}

%Environment to hold the "DESCRIPTION" portion of a Tcl command description.
\newenvironment{tclcommanddescription}{\noindent\textbf{DESCRIPTION}\vspace{-1.0ex}\begin{list}%
               {}{\setlength{\leftmargin}{0.25in}}\item{}}{\end{list}}

%Environment to hold the "ACKNOWLEDGEMENTS" portion of a Tcl command description.
\newenvironment{tclcommandacknowledgements}{\noindent\textbf{ACKNOWLEDGEMENTS}\vspace{-1.0ex}\begin{list}%
               {}{\setlength{\leftmargin}{0.25in}}\item{}}{\end{list}}

%Environment to hold a specific subdescription within the DESCRIPTION--useful for enumerating
%possible legal commands and command forms.
\newenvironment{tclcommandinternaldescription}{\noindent{}\vspace{-5.0ex}\begin{list}%
               {}{\setlength{\leftmargin}{0.25in}}\item{}}{\end{list}}

%Command to use to format a command prototype preceding the environment above (i.e. a synopsis).
\newcommand{\tclcommanddescsynopsisline}[2]{\par\textbf{#1} \textit{#2}\par}

%Environment to hold the "USAGE NOTES" portion of a Tcl command description.
\newenvironment{tclcommandusagenotes}{\noindent\textbf{USAGE NOTES}\vspace{-1.0ex}\begin{list}%
               {}{\setlength{\leftmargin}{0.25in}}\item{}}{\end{list}}

%Environment to hold the "SAMPLE INVOCATIONS" portion of a TCL command or extension description.
\newenvironment{tclcommandsampleinvocations}{\noindent\textbf{SAMPLE INVOCATIONS}\vspace{-1.0ex}\begin{list}%
               {}{\setlength{\leftmargin}{0.25in}}\item{}}{\end{list}}

%Environment to hold the "SEE ALSO" portion of a TCL command or extension utility description.
\newenvironment{tclcommandseealso}{\noindent\textbf{SEE ALSO}\vspace{-1.0ex}\begin{list}%
               {}{\setlength{\leftmargin}{0.25in}}\item{}}{\end{list}}

%%%%%%%%%%%%%%%%%%%%%%%%%%%%%%%%%%%%%%%%%%%%%%%%%%%%%%%%%%%%%%%%%%%%%%%%%%%%%%%%%%%%%%%%%%
%%%%%%%%%%%%%%%%%%%%%%%%%%%%%%%%%%%%%%%%%%%%%%%%%%%%%%%%%%%%%%%%%%%%%%%%%%%%%%%%%%%%%%%%%%
% DOS COMMAND-LINE UTILITY DOCUMENTATION
%
%Environment to hold the "NAME" portion of a DOS command-line utility.
\newenvironment{dosutilcommandname}[1]{\noindent\textbf{NAME}\vspace{-1.0ex}\begin{list}%
               {}{\setlength{\leftmargin}{0.25in}}\item\textbf{#1}---}{\end{list}}

%Command to display a synopsis line inside the synopsis environment.
\newcommand{\dosutilcommandsynopsisline}[2]{\par\textbf{#1} \textit{#2}\par}

%Environment to hold the "SYNOPSIS" portion of a DOS command-line utility description.
\newenvironment{dosutilcommandsynopsis}{\noindent\textbf{SYNOPSIS}\vspace{-1.0ex}\begin{list}%
               {}{\setlength{\leftmargin}{0.25in}}\item{}}{\end{list}}

%Environment to hold the "DESCRIPTION" portion of a DOS command-line utility description.
\newenvironment{dosutilcommanddescription}{\noindent\textbf{DESCRIPTION}\vspace{-1.0ex}\begin{list}%
               {}{\setlength{\leftmargin}{0.25in}}\item{}}{\end{list}}

%Environment to hold the "ACKNOWLEDGEMENTS" portion of a DOS command-line utility description.
\newenvironment{dosutilcommandacknowledgements}{\noindent\textbf{ACKNOWLEDGEMENTS}\vspace{-1.0ex}\begin{list}%
               {}{\setlength{\leftmargin}{0.25in}}\item{}}{\end{list}}

%Environment to hold a specific subdescription within the DESCRIPTION--useful for enumerating
%possible legal commands and command forms.
\newenvironment{dosutilcommandinternaldescription}{\noindent{}\vspace{-5.0ex}\begin{list}%
               {}{\setlength{\leftmargin}{0.25in}}\item{}}{\end{list}}

%Command to use to format a command prototype preceding the environment above (i.e. a synopsis).
\newcommand{\dosutilcommanddescsynopsisline}[2]{\par\textbf{#1} \textit{#2}\par}

%Environment to hold the "USAGE NOTES" portion of a DOS command-line utility description.
\newenvironment{dosutilcommandusagenotes}{\noindent\textbf{USAGE NOTES}\vspace{-1.0ex}\begin{list}%
               {}{\setlength{\leftmargin}{0.25in}}\item{}}{\end{list}}

%Environment to hold the "REMARKS" portion of a DOS command-line utility description.
\newenvironment{dosutilcommandremarks}{\noindent\textbf{REMARKS}\vspace{-1.0ex}\begin{list}%
               {}{\setlength{\leftmargin}{0.25in}}\item{}}{\end{list}}

%Environment to hold the "COMMAND-LINE OPTIONS" portion of a DOS command-line utility description.
\newenvironment{dosutilcommandcommandlineoptions}{\noindent\textbf{COMMAND-LINE OPTIONS}\vspace{-1.0ex}\begin{list}%
               {}{\setlength{\leftmargin}{0.25in}}\item{}}{\end{list}}

%Environment to hold the "SAMPLE INVOCATION" portion of a DOS command-line utility description.
\newenvironment{dosutilcommandsampleinvocations}{\noindent\textbf{SAMPLE INVOCATIONS}\vspace{-1.0ex}\begin{list}%
               {}{\setlength{\leftmargin}{0.25in}}\item{}}{\end{list}}

%Environment to hold the "SEE ALSO" portion of a DOS command-line utility description.
\newenvironment{dosutilcommandseealso}{\noindent\textbf{SEE ALSO}\vspace{-1.0ex}\begin{list}%
               {}{\setlength{\leftmargin}{0.25in}}\item{}}{\end{list}}
%%
%%%%%%%%%%%%%%%%%%%%%%%%%%%%%%%%%%%%%%%%%%%%%%%%%%%%%%%%%%%%%%%%%%%%%%%%%%%%%
%% DESIRABLE PROPERTIES
%%%%%%%%%%%%%%%%%%%%%%%%%%%%%%%%%%%%%%%%%%%%%%%%%%%%%%%%%%%%%%%%%%%%%%%%%%%%%
%Command to identify the prefix used for desirable properties.
%"DP" seems as good as anything.
\newcommand{\desirablepropertyprefix}{DP}
%
%% This counter is used only to number the desirable properties of
%% embedded software used in the "Holy Grail" chapter.
\newcounter{vworkdesirablepropertiescounter}

%Command to increment the desirable properties counter, emit the tag,
%and set the \ref value.
\newcommand{\desirablepropertyemit}{\refstepcounter{vworkdesirablepropertiescounter}%
                                    [\desirablepropertyprefix-\thevworkdesirablepropertiescounter]}

%Command to cite a desirable property.
\newcommand{\desirablepropertycite}[1]{[\desirablepropertyprefix-\ref{#1}]}

%%%%%%%%%%%%%%%%%%%%%%%%%%%%%%%%%%%%%%%%%%%%%%%%%%%%%%%%%%%%%%%%%%%%%%%%%%%%%%%%%%%%%%%%%%
%%%%%%%%%%%%%%%%%%%%%%%%%%%%%%%%%%%%%%%%%%%%%%%%%%%%%%%%%%%%%%%%%%%%%%%%%%%%%%%%%%%%%%%%%%
%%%%%%%%%%%%%%%%%%%%%%%%%%%%%%%%%%%%%%%%%%%%%%%%%%%%%%%%%%%%%%%%%%%%%%%%%%%%%%%%%%%%%%%%%%
%$Log: worknenv.tex,v $
%Revision 1.5  2010/01/27 16:08:53  dashley
%Formatting difficulties corrected.
%
%Revision 1.4  2010/01/26 02:06:26  dashley
%Edits.
%
%Revision 1.3  2007/10/09 02:11:40  dtashley
%Edits.
%
%Revision 1.2  2007/10/08 19:51:59  dtashley
%Edits toward getting function documentation looking good.
%
%Revision 1.1  2007/08/30 14:27:27  dtashley
%Initial checkin.
%
%End of $RCSfile: worknenv.tex,v $.
