%$Header: /home/dashley/cvsrep/uculib01/uculib01/doc/manual/c_bsf0/c_bsf0.tex,v 1.9 2010/05/13 14:11:43 dashley Exp $

\chapter{Bit-Mapped Set Functions}        
\label{cbsf0}

%%%%%%%%%%%%%%%%%%%%%%%%%%%%%%%%%%%%%%%%%%%%%%%%%%%%%%%%%%%%%%%%%%%%%%%%%%%%%%%
%%%%%%%%%%%%%%%%%%%%%%%%%%%%%%%%%%%%%%%%%%%%%%%%%%%%%%%%%%%%%%%%%%%%%%%%%%%%%%%
%%%%%%%%%%%%%%%%%%%%%%%%%%%%%%%%%%%%%%%%%%%%%%%%%%%%%%%%%%%%%%%%%%%%%%%%%%%%%%%
\section{Introduction and Overview}
\label{cbsf0:siov0}

This chapter describes functions and data tables that operate on data
as collections of bits.

\begin{itemize}
\item \S{}\ref{cbsf0:sctb0} (p. \pageref{cbsf0:sctb0})
      describes constant lookup tables that are stored
      in FLASH or ROM memory.  The primary purpose of these tables is
      to support the functions described by this chapter, but the tables
      are available publicly as well.
\item \S{}\ref{cbsf0:sbcf0} (p. \pageref{cbsf0:sbcf0})
      describes functions that calculate
      bit cardinality (number of bits set or cleared).
\item \S{}\ref{cbsf0:srof0} (p. \pageref{cbsf0:srof0})
      describes functions that rotate
      arrays of bits.
\end{itemize}


%%%%%%%%%%%%%%%%%%%%%%%%%%%%%%%%%%%%%%%%%%%%%%%%%%%%%%%%%%%%%%%%%%%%%%%%%%%%%%%
%%%%%%%%%%%%%%%%%%%%%%%%%%%%%%%%%%%%%%%%%%%%%%%%%%%%%%%%%%%%%%%%%%%%%%%%%%%%%%%
%%%%%%%%%%%%%%%%%%%%%%%%%%%%%%%%%%%%%%%%%%%%%%%%%%%%%%%%%%%%%%%%%%%%%%%%%%%%%%%
\section{Constant Lookup Tables}
\label{cbsf0:sctb0}


%%%%%%%%%%%%%%%%%%%%%%%%%%%%%%%%%%%%%%%%%%%%%%%%%%%%%%%%%%%%%%%%%%%%%%%%%%%%%%%
%%%%%%%%%%%%%%%%%%%%%%%%%%%%%%%%%%%%%%%%%%%%%%%%%%%%%%%%%%%%%%%%%%%%%%%%%%%%%%%
%%%%%%%%%%%%%%%%%%%%%%%%%%%%%%%%%%%%%%%%%%%%%%%%%%%%%%%%%%%%%%%%%%%%%%%%%%%%%%%
\subsection[\emph{UcuBtU8ByteCardNibpLut[\protect\mbox{\protect$\cdot$}]}]
           {\emph{UcuBtU8ByteCardNibpLut[\protect\mbox{\protect\boldmath $\cdot$}]}}
\label{cbsf0:sctb0:suec0}

\index{UcuBtU8ByteCardNibpLut[]@\emph{UcuBtU8ByteCardNibpLut[$\cdot$]}}%

\noindent\textbf{PROTOTYPE}
\begin {list}{}{\setlength{\leftmargin}{0.25in}\setlength{\topsep}{0.0in}}
\item
\begin{verbatim}
const UCU_UINT8 UcuBtU8ByteCardNibpLut[128]
\end{verbatim}
\end{list}
\vspace{2.8ex}

\noindent\textbf{SYNOPSIS}
\begin{list}{}{\setlength{\leftmargin}{0.25in}\setlength{\topsep}{0.0in}}
\item Contains the number of `1' bits (also called the bit cardinality of the byte)
      in each of the 256 possible bytes, packed
      into nibbles.  If the byte value $b$ is $\leq$127, the 
      bit cardinality is in the lower nibble of array element
      [$b$].  If the byte value $b$ is $\geq$128, the 
      bit cardinality is in the upper nibble of array
      element [$b-128$].
\item For example, the value of element [11] is \$43.  This indicates that that
      the number 11 (= \$0B = \%00001011) contains three `1' bits and that the number
      139 (= 128+11 = \$8B = \%10001011) contains four `1' bits.
\end{list}
\vspace{2.8ex}

\noindent\textbf{TABLE NAME MNEMONIC}
\begin{list}{}{\setlength{\leftmargin}{0.25in}\setlength{\topsep}{0.0in}}
\item \emph{U8}: each element is of type UCU\_UINT8.
      \emph{ByteCard}: cardinality of a byte.
      \emph{Nibp}: nibble packed.
      \emph{Lut}: lookup table.
\end{list}


%%%%%%%%%%%%%%%%%%%%%%%%%%%%%%%%%%%%%%%%%%%%%%%%%%%%%%%%%%%%%%%%%%%%%%%%%%%%%%%
%%%%%%%%%%%%%%%%%%%%%%%%%%%%%%%%%%%%%%%%%%%%%%%%%%%%%%%%%%%%%%%%%%%%%%%%%%%%%%%
%%%%%%%%%%%%%%%%%%%%%%%%%%%%%%%%%%%%%%%%%%%%%%%%%%%%%%%%%%%%%%%%%%%%%%%%%%%%%%%
\subsection[\emph{UcuBtU32RmaskLut[\protect\mbox{\protect$\cdot$}]}]
           {\emph{UcuBtU32RmaskLut[\protect\mbox{\protect\boldmath $\cdot$}]}}
\label{cbsf0:sctb0:srml0}

\index{UcuBtU32RmaskLut[]@\emph{UcuBtU32RmaskLut[$\cdot$]}}%

\noindent\textbf{PROTOTYPE}
\begin {list}{}{\setlength{\leftmargin}{0.25in}\setlength{\topsep}{0.0in}}
\item
\begin{verbatim}
const UCU_UINT32 UcuBtU32RmaskLut[33]
\end{verbatim}
\end{list}
\vspace{2.8ex}

\noindent\textbf{SYNOPSIS}
\begin{list}{}{\setlength{\leftmargin}{0.25in}\setlength{\topsep}{0.0in}}
\item Array element [$b$] contains a UCU\_UINT32 integer with the
      rightmost $b$ bits set to `1' and the others `0'.
\item For example, the value of element [5] is \%00000000 00000000 00000000 00011111.
\item Element [0] contains 0, and element [32] contains $2^{32}-1$.
\end{list}
\vspace{2.8ex}

\noindent\textbf{TABLE NAME MNEMONIC}
\begin{list}{}{\setlength{\leftmargin}{0.25in}\setlength{\topsep}{0.0in}}
\item \emph{U32}: each element is of type UCU\_UINT32.
      \emph{Rmask}: right mask.
      \emph{Lut}: lookup table.
\end{list}


%%%%%%%%%%%%%%%%%%%%%%%%%%%%%%%%%%%%%%%%%%%%%%%%%%%%%%%%%%%%%%%%%%%%%%%%%%%%%%%
%%%%%%%%%%%%%%%%%%%%%%%%%%%%%%%%%%%%%%%%%%%%%%%%%%%%%%%%%%%%%%%%%%%%%%%%%%%%%%%
%%%%%%%%%%%%%%%%%%%%%%%%%%%%%%%%%%%%%%%%%%%%%%%%%%%%%%%%%%%%%%%%%%%%%%%%%%%%%%%
\subsection[\emph{UcuBtU32BitByIndexLut[\protect\mbox{\protect$\cdot$}]}]
           {\emph{UcuBtU32BitByIndexLut[\protect\mbox{\protect\boldmath $\cdot$}]}}
\label{cbsf0:sctb0:sbbi0}

\index{UcuBtU32BitByIndexLut[]@\emph{UcuBtU32BitByIndexLut[$\cdot$]}}%

\noindent\textbf{PROTOTYPE}
\begin {list}{}{\setlength{\leftmargin}{0.25in}\setlength{\topsep}{0.0in}}
\item
\begin{verbatim}
const UCU_UINT32 UcuBtU32BitByIndexLut[32]
\end{verbatim}
\end{list}
\vspace{2.8ex}

\noindent\textbf{SYNOPSIS}
\begin{list}{}{\setlength{\leftmargin}{0.25in}\setlength{\topsep}{0.0in}}
\item Array element [$b$] contains a UCU\_UINT32 integer with
      bit $b$ bit set to `1' and the others `0'.  (Bits are numbered starting
      with 0 from the right.)
\item For example, the value of element [5] is \%00000000 00000000 00000000 00100000.
\item Element [0] contains 1, and element [31] contains $2^{31}$.
\end{list}
\vspace{2.8ex}

\noindent\textbf{TABLE NAME MNEMONIC}
\begin{list}{}{\setlength{\leftmargin}{0.25in}\setlength{\topsep}{0.0in}}
\item \emph{U32}: each element is of type UCU\_UINT32.
      \emph{BitByIndex}: indexing into the table by $b$ provides bit number $b$.
      \emph{Lut}: lookup table.
\end{list}


%%%%%%%%%%%%%%%%%%%%%%%%%%%%%%%%%%%%%%%%%%%%%%%%%%%%%%%%%%%%%%%%%%%%%%%%%%%%%%%
%%%%%%%%%%%%%%%%%%%%%%%%%%%%%%%%%%%%%%%%%%%%%%%%%%%%%%%%%%%%%%%%%%%%%%%%%%%%%%%
%%%%%%%%%%%%%%%%%%%%%%%%%%%%%%%%%%%%%%%%%%%%%%%%%%%%%%%%%%%%%%%%%%%%%%%%%%%%%%%
\section{Bit Cardinality Functions}
\label{cbsf0:sbcf0}


%%%%%%%%%%%%%%%%%%%%%%%%%%%%%%%%%%%%%%%%%%%%%%%%%%%%%%%%%%%%%%%%%%%%%%%%%%%%%%%
%%%%%%%%%%%%%%%%%%%%%%%%%%%%%%%%%%%%%%%%%%%%%%%%%%%%%%%%%%%%%%%%%%%%%%%%%%%%%%%
%%%%%%%%%%%%%%%%%%%%%%%%%%%%%%%%%%%%%%%%%%%%%%%%%%%%%%%%%%%%%%%%%%%%%%%%%%%%%%%
\subsection[\emph{UcuBtU8BitCardRxx(\protect\mbox{\protect$\cdot$})}]
           {\emph{UcuBtU8BitCardRxx(\protect\mbox{\protect\boldmath $\cdot$})}}
\label{cbsf0:sbcf0:sbce0}

\index{UcuBtU8BitCardRxx()@\emph{UcuBtU8BitCardRxx($\cdot$)}}%

\noindent\textbf{PROTOTYPE}
\begin {list}{}{\setlength{\leftmargin}{0.25in}\setlength{\topsep}{0.0in}}
\item
\begin{verbatim}
UCU_UINT8 UcuBtU8BitCardRxx(UCU_UINT8 arg)
\end{verbatim}
\end{list}
\vspace{2.8ex}

\noindent\textbf{SYNOPSIS}
\begin{list}{}{\setlength{\leftmargin}{0.25in}\setlength{\topsep}{0.0in}}
\item Calculates the number of bits set in a UCU\_UINT8.  The return value cannot
      exceed 8.
\end{list}
\vspace{2.8ex}

\noindent\textbf{INPUTS}
\begin{list}{}{\setlength{\leftmargin}{0.5in}\setlength{\itemindent}{-0.25in}\setlength{\topsep}{0.0in}\setlength{\partopsep}{0.0in}}
\item \emph{\textbf{arg}}\\
      The UCU\_UINT8 for which the bit cardinality is to be calculated.
\end{list}
\vspace{2.8ex}

\noindent\textbf{OUTPUT}
\begin{list}{}{\setlength{\leftmargin}{0.25in}\setlength{\topsep}{0.0in}}
\item  The number of bits set in $arg$.  The return value will not exceed 8.
\end{list}
\vspace{2.8ex}

\noindent\textbf{EXCEPTION CASES}
\begin{list}{}{\setlength{\leftmargin}{0.25in}\setlength{\topsep}{0.0in}}
\item None.
\end{list}
\vspace{2.8ex}

\noindent\textbf{INTERRUPT COMPATIBILITY}
\begin{list}{}{\setlength{\leftmargin}{0.25in}\setlength{\topsep}{0.0in}}
\item This function may be used from both non-ISR and ISR software.
\item This function is thread-safe.
\end{list}
\vspace{2.8ex}

\noindent\textbf{EXECUTION TIME}
\begin{list}{}{\setlength{\leftmargin}{0.25in}\setlength{\topsep}{0.0in}}
\item TBD.
\end{list}
\vspace{2.8ex}

\noindent\textbf{FUNCTION NAME MNEMONIC}
\begin{list}{}{\setlength{\leftmargin}{0.25in}\setlength{\topsep}{0.0in}}
\item \emph{U8}: operates on unsigned UCU\_UINT8 operands.
      \emph{BitCard}: bit cardinality.
\end{list}


%%%%%%%%%%%%%%%%%%%%%%%%%%%%%%%%%%%%%%%%%%%%%%%%%%%%%%%%%%%%%%%%%%%%%%%%%%%%%%%
%%%%%%%%%%%%%%%%%%%%%%%%%%%%%%%%%%%%%%%%%%%%%%%%%%%%%%%%%%%%%%%%%%%%%%%%%%%%%%%
%%%%%%%%%%%%%%%%%%%%%%%%%%%%%%%%%%%%%%%%%%%%%%%%%%%%%%%%%%%%%%%%%%%%%%%%%%%%%%%
\subsection[\emph{UcuBtU16BitCardRxx(\protect\mbox{\protect$\cdot$})}]
           {\emph{UcuBtU16BitCardRxx(\protect\mbox{\protect\boldmath $\cdot$})}}
\label{cbsf0:sbcf0:sbcs0}

\index{UcuBtU16BitCardRxx()@\emph{UcuBtU16BitCardRxx($\cdot$)}}%

\noindent\textbf{PROTOTYPE}
\begin {list}{}{\setlength{\leftmargin}{0.25in}\setlength{\topsep}{0.0in}}
\item
\begin{verbatim}
UCU_UINT8 UcuBtU16BitCardRxx(UCU_UINT16 arg)
\end{verbatim}
\end{list}
\vspace{2.8ex}

\noindent\textbf{SYNOPSIS}
\begin{list}{}{\setlength{\leftmargin}{0.25in}\setlength{\topsep}{0.0in}}
\item Calculates the number of bits set in a UCU\_UINT16.  The return value cannot
      exceed 16.
\end{list}
\vspace{2.8ex}

\noindent\textbf{INPUTS}
\begin{list}{}{\setlength{\leftmargin}{0.5in}\setlength{\itemindent}{-0.25in}\setlength{\topsep}{0.0in}\setlength{\partopsep}{0.0in}}
\item \emph{\textbf{arg}}\\
      The UCU\_UINT16 for which the bit cardinality is to be calculated.
\end{list}
\vspace{2.8ex}

\noindent\textbf{OUTPUT}
\begin{list}{}{\setlength{\leftmargin}{0.25in}\setlength{\topsep}{0.0in}}
\item  The number of bits set in $arg$.  The return value will not exceed 16.
\end{list}
\vspace{2.8ex}

\noindent\textbf{EXCEPTION CASES}
\begin{list}{}{\setlength{\leftmargin}{0.25in}\setlength{\topsep}{0.0in}}
\item None.
\end{list}
\vspace{2.8ex}

\noindent\textbf{INTERRUPT COMPATIBILITY}
\begin{list}{}{\setlength{\leftmargin}{0.25in}\setlength{\topsep}{0.0in}}
\item This function may be used from both non-ISR and ISR software.
\item This function is thread-safe.
\end{list}
\vspace{2.8ex}

\noindent\textbf{EXECUTION TIME}
\begin{list}{}{\setlength{\leftmargin}{0.25in}\setlength{\topsep}{0.0in}}
\item TBD.
\end{list}
\vspace{2.8ex}

\noindent\textbf{FUNCTION NAME MNEMONIC}
\begin{list}{}{\setlength{\leftmargin}{0.25in}\setlength{\topsep}{0.0in}}
\item \emph{U16}: operates on unsigned UCU\_UINT16 operands.
      \emph{BitCard}: bit cardinality.
\end{list}


%%%%%%%%%%%%%%%%%%%%%%%%%%%%%%%%%%%%%%%%%%%%%%%%%%%%%%%%%%%%%%%%%%%%%%%%%%%%%%%
%%%%%%%%%%%%%%%%%%%%%%%%%%%%%%%%%%%%%%%%%%%%%%%%%%%%%%%%%%%%%%%%%%%%%%%%%%%%%%%
%%%%%%%%%%%%%%%%%%%%%%%%%%%%%%%%%%%%%%%%%%%%%%%%%%%%%%%%%%%%%%%%%%%%%%%%%%%%%%%
\subsection[\emph{UcuBtU32BitCardRxx(\protect\mbox{\protect$\cdot$})}]
           {\emph{UcuBtU32BitCardRxx(\protect\mbox{\protect\boldmath $\cdot$})}}
\label{cbsf0:sbcf0:sbct0}

\index{UcuBtU32BitCardRxx()@\emph{UcuBtU32BitCardRxx($\cdot$)}}%

\noindent\textbf{PROTOTYPE}
\begin {list}{}{\setlength{\leftmargin}{0.25in}\setlength{\topsep}{0.0in}}
\item
\begin{verbatim}
UCU_UINT8 UcuBtU32BitCardRxx(UCU_UINT32 arg)
\end{verbatim}
\end{list}
\vspace{2.8ex}

\noindent\textbf{SYNOPSIS}
\begin{list}{}{\setlength{\leftmargin}{0.25in}\setlength{\topsep}{0.0in}}
\item Calculates the number of bits set in a UCU\_UINT32.  The return value cannot
      exceed 32.
\end{list}
\vspace{2.8ex}

\noindent\textbf{INPUTS}
\begin{list}{}{\setlength{\leftmargin}{0.5in}\setlength{\itemindent}{-0.25in}\setlength{\topsep}{0.0in}\setlength{\partopsep}{0.0in}}
\item \emph{\textbf{arg}}\\
      The UCU\_UINT32 for which the bit cardinality is to be calculated.
\end{list}
\vspace{2.8ex}

\noindent\textbf{OUTPUT}
\begin{list}{}{\setlength{\leftmargin}{0.25in}\setlength{\topsep}{0.0in}}
\item  The number of bits set in $arg$.  The return value will not exceed 32.
\end{list}
\vspace{2.8ex}

\noindent\textbf{EXCEPTION CASES}
\begin{list}{}{\setlength{\leftmargin}{0.25in}\setlength{\topsep}{0.0in}}
\item None.
\end{list}
\vspace{2.8ex}

\noindent\textbf{INTERRUPT COMPATIBILITY}
\begin{list}{}{\setlength{\leftmargin}{0.25in}\setlength{\topsep}{0.0in}}
\item This function may be used from both non-ISR and ISR software.
\item This function is thread-safe.
\end{list}
\vspace{2.8ex}

\noindent\textbf{EXECUTION TIME}
\begin{list}{}{\setlength{\leftmargin}{0.25in}\setlength{\topsep}{0.0in}}
\item TBD.
\end{list}
\vspace{2.8ex}

\noindent\textbf{FUNCTION NAME MNEMONIC}
\begin{list}{}{\setlength{\leftmargin}{0.25in}\setlength{\topsep}{0.0in}}
\item \emph{U32}: operates on unsigned UCU\_UINT32 operands.
      \emph{BitCard}: bit cardinality.
\end{list}


%%%%%%%%%%%%%%%%%%%%%%%%%%%%%%%%%%%%%%%%%%%%%%%%%%%%%%%%%%%%%%%%%%%%%%%%%%%%%%%
%%%%%%%%%%%%%%%%%%%%%%%%%%%%%%%%%%%%%%%%%%%%%%%%%%%%%%%%%%%%%%%%%%%%%%%%%%%%%%%
%%%%%%%%%%%%%%%%%%%%%%%%%%%%%%%%%%%%%%%%%%%%%%%%%%%%%%%%%%%%%%%%%%%%%%%%%%%%%%%
\subsection[\emph{UcuBtU32BitCardRnRxx(\protect\mbox{\protect$\cdot$})}]
           {\emph{UcuBtU32BitCardRnRxx(\protect\mbox{\protect\boldmath $\cdot$})}}
\label{cbsf0:sbcf0:sbcr0}

\index{UcuBtU32BitCardRnRxx()@\emph{UcuBtU32BitCardRnRxx($\cdot$)}}%

\noindent\textbf{PROTOTYPE}
\begin {list}{}{\setlength{\leftmargin}{0.25in}\setlength{\topsep}{0.0in}}
\item
\begin{verbatim}
UCU_UINT8 UcuBtU32BitCardRnRxx(UCU_UINT32 arg, UCU_UINT8 n)
\end{verbatim}
\end{list}
\vspace{2.8ex}

\noindent\textbf{SYNOPSIS}
\begin{list}{}{\setlength{\leftmargin}{0.25in}\setlength{\topsep}{0.0in}}
\item Calculates the number of bits set among the rightmost
      $n$ bits in a $arg$.  The return value cannot
      exceed 32.
\end{list}
\vspace{2.8ex}

\noindent\textbf{INPUTS}
\begin{list}{}{\setlength{\leftmargin}{0.5in}\setlength{\itemindent}{-0.25in}\setlength{\topsep}{0.0in}\setlength{\partopsep}{0.0in}}
\item \emph{\textbf{arg}}\\
      The UCU\_UINT32 for which the bit cardinality 
      of the rightmost $n$ bits is to be calculated.
\item \emph{\textbf{n}}\\
      The number of bits at the right for which to
      calculate the cardinality.  A value of 0 will 
      result in 0 returned from this function.  A
      value of 32 will result in behavior idential to the
      \emph{UcuBtU32BitCardRxx($\cdot$)} function.
\end{list}
\vspace{2.8ex}

\noindent\textbf{OUTPUT}
\begin{list}{}{\setlength{\leftmargin}{0.25in}\setlength{\topsep}{0.0in}}
\item  The number of bits set in $arg$ among the
       rightmost $n$ bits.  The return value will not exceed 32.
\end{list}
\vspace{2.8ex}

\noindent\textbf{EXCEPTION CASES}
\begin{list}{}{\setlength{\leftmargin}{0.25in}\setlength{\topsep}{0.0in}}
\item A value of $n$ greater than 32 will result in $n$ being treated
      as 32.
\end{list}
\vspace{2.8ex}

\noindent\textbf{INTERRUPT COMPATIBILITY}
\begin{list}{}{\setlength{\leftmargin}{0.25in}\setlength{\topsep}{0.0in}}
\item This function may be used from both non-ISR and ISR software.
\item This function is thread-safe.
\end{list}
\vspace{2.8ex}

\noindent\textbf{EXECUTION TIME}
\begin{list}{}{\setlength{\leftmargin}{0.25in}\setlength{\topsep}{0.0in}}
\item TBD.
\end{list}
\vspace{2.8ex}

\noindent\textbf{FUNCTION NAME MNEMONIC}
\begin{list}{}{\setlength{\leftmargin}{0.25in}\setlength{\topsep}{0.0in}}
\item \emph{U32}: operates on unsigned UCU\_UINT32 operands.
      \emph{BitCard}: bit cardinality.
      \emph{Rn}: a number of bits at the right.
\end{list}


%%%%%%%%%%%%%%%%%%%%%%%%%%%%%%%%%%%%%%%%%%%%%%%%%%%%%%%%%%%%%%%%%%%%%%%%%%%%%%%
%%%%%%%%%%%%%%%%%%%%%%%%%%%%%%%%%%%%%%%%%%%%%%%%%%%%%%%%%%%%%%%%%%%%%%%%%%%%%%%
%%%%%%%%%%%%%%%%%%%%%%%%%%%%%%%%%%%%%%%%%%%%%%%%%%%%%%%%%%%%%%%%%%%%%%%%%%%%%%%
\section{Rotation Functions}
\label{cbsf0:srof0}


%%%%%%%%%%%%%%%%%%%%%%%%%%%%%%%%%%%%%%%%%%%%%%%%%%%%%%%%%%%%%%%%%%%%%%%%%%%%%%%
%%%%%%%%%%%%%%%%%%%%%%%%%%%%%%%%%%%%%%%%%%%%%%%%%%%%%%%%%%%%%%%%%%%%%%%%%%%%%%%
%%%%%%%%%%%%%%%%%%%%%%%%%%%%%%%%%%%%%%%%%%%%%%%%%%%%%%%%%%%%%%%%%%%%%%%%%%%%%%%
\subsection[\emph{UcuBtU32RotLeftNInPlaceRxn(\protect\mbox{\protect$\cdot$})}]
           {\emph{UcuBtU32RotLeftNInPlaceRxn(\protect\mbox{\protect\boldmath $\cdot$})}}
\label{cbsf0:srof0:srle0}

\index{UcuBtU32RotLeftNInPlaceRxn()@\emph{UcuBtU32RotLeftNInPlaceRxn($\cdot$)}}%

\noindent\textbf{PROTOTYPE}
\begin {list}{}{\setlength{\leftmargin}{0.25in}\setlength{\topsep}{0.0in}}
\item
\begin{verbatim}
void UcuBtU32RotLeftNInPlaceRxn(UCU_UINT32 *tgt, UCU_UINT8 n)
\end{verbatim}
\end{list}
\vspace{2.8ex}

\noindent\textbf{SYNOPSIS}
\begin{list}{}{\setlength{\leftmargin}{0.25in}\setlength{\topsep}{0.0in}}
\item Rolls the target UCU\_UINT32 left by $n$ bits.  (By ``roll'' we mean
      that each bit shifted
      out of position 31 is copied into position 0.)
\end{list}
\vspace{2.8ex}

\noindent\textbf{INPUTS}
\begin{list}{}{\setlength{\leftmargin}{0.5in}\setlength{\itemindent}{-0.25in}\setlength{\topsep}{0.0in}\setlength{\partopsep}{0.0in}}
\item \emph{\textbf{tgt}}\\
      A pointer to the target UCU\_UINT32.  This pointer may not be
      NULL or otherwise invalid.  This function provides no guarantees
      about the order in which the bytes of the target will be accessed,
      how many times they will be accessed, or (if $n \bmod 32 = 0$) that
      they will be accessed at all.  The only guarantee provided is that
      at function exit the correct value will be stored in the target location.
\item \emph{\textbf{n}}\\
      The number of bits to roll the target left.
      All UCU\_UINT8 input values including 0
      are permitted and treated correctly.
\end{list}
\vspace{2.8ex}

\noindent\textbf{OUTPUT}
\begin{list}{}{\setlength{\leftmargin}{0.25in}\setlength{\topsep}{0.0in}}
\item  \emph{*tgt} will have been rolled left $n$ bits.
\end{list}
\vspace{2.8ex}

\noindent\textbf{EXCEPTION CASES}
\begin{list}{}{\setlength{\leftmargin}{0.25in}\setlength{\topsep}{0.0in}}
\item None.
\end{list}
\vspace{2.8ex}

\noindent\textbf{INTERRUPT COMPATIBILITY}
\begin{list}{}{\setlength{\leftmargin}{0.25in}\setlength{\topsep}{0.0in}}
\item This function may be used from both non-ISR and ISR software and
      is thread-safe,
      so long as the UCU\_UINT32 data items pointed
      to by \emph{*tgt} are unique.
\end{list}
\vspace{2.8ex}

\noindent\textbf{EXECUTION TIME}
\begin{list}{}{\setlength{\leftmargin}{0.25in}\setlength{\topsep}{0.0in}}
\item TBD.
\end{list}
\vspace{2.8ex}

\noindent\textbf{FUNCTION NAME MNEMONIC}
\begin{list}{}{\setlength{\leftmargin}{0.25in}\setlength{\topsep}{0.0in}}
\item \emph{U32}:     operates on unsigned UCU\_UINT32 operands.
      \emph{Rot}:     rotate.
      \emph{Left}:    left.
      \emph{N}:       by $n$ bits.
      \emph{InPlace}: operates on the operand in place (rather than returning the
                      modified operand).
\end{list}


%%%%%%%%%%%%%%%%%%%%%%%%%%%%%%%%%%%%%%%%%%%%%%%%%%%%%%%%%%%%%%%%%%%%%%%%%%
\noindent\begin{figure}[!b]
\noindent\rule[-0.25in]{\textwidth}{1pt}
\begin{tiny}
\begin{verbatim}
$RCSfile: c_bsf0.tex,v $
$Source: /home/dashley/cvsrep/uculib01/uculib01/doc/manual/c_bsf0/c_bsf0.tex,v $
$Revision: 1.9 $
$Author: dashley $
$Date: 2010/05/13 14:11:43 $
\end{verbatim}
\end{tiny}
\noindent\rule[0.25in]{\textwidth}{1pt}
\end{figure}

%%%%%%%%%%%%%%%%%%%%%%%%%%%%%%%%%%%%%%%%%%%%%%%%%%%%%%%%%%%%%%%%%%%%%%%%%%%%%%%
%$Log: c_bsf0.tex,v $
%Revision 1.9  2010/05/13 14:11:43  dashley
%Renaming of function UcuBtU32RotLeftNInPlaceRxx() to
%UcuBtU32RotLeftNInPlaceRxn().
%
%Revision 1.8  2010/05/12 22:24:57  dashley
%Addition of UcuBtU32RotLeftNInPlaceRxx() function.
%
%Revision 1.7  2010/02/11 16:57:28  dashley
%Addition of three functions.
%
%Revision 1.6  2010/02/10 20:04:37  dashley
%Edits.
%
%Revision 1.5  2010/02/10 19:55:27  dashley
%Edits.
%
%Revision 1.4  2010/02/10 16:46:56  dashley
%Edits.
%
%Revision 1.3  2010/01/28 21:18:32  dashley
%a)Chapter start quotes removed.
%b)Aesthetic comment line added at the bottom of most files.
%
%Revision 1.2  2010/01/24 05:37:27  dashley
%Addition and reorganization of content.
%
%Revision 1.1  2007/10/06 22:56:03  dtashley
%initial checkin.
%End of $RCSfile: c_bsf0.tex,v $.
%%%%%%%%%%%%%%%%%%%%%%%%%%%%%%%%%%%%%%%%%%%%%%%%%%%%%%%%%%%%%%%%%%%%%%%%%%%%%%%

