%$Header: /home/dashley/cvsrep/uculib01/uculib01/doc/manual/c_glo1/c_glo1.tex,v 1.2 2010/01/28 21:18:32 dashley Exp $

\chapter{Glossary Of Mathematical And Other Notation}
\markboth{GLOSSARY OF MATHEMATICAL NOTATION}{GLOSSARY OF MATHEMATICAL NOTATION}

\label{cglo1}

%%%%%%%%%%%%%%%%%%%%%%%%%%%%%%%%%%%%%%%%%%%%%%%%%%%%%%%%%%%%%%%%%%%%%%%%%%%%%%%
%%%%%%%%%%%%%%%%%%%%%%%%%%%%%%%%%%%%%%%%%%%%%%%%%%%%%%%%%%%%%%%%%%%%%%%%%%%%%%%
%%%%%%%%%%%%%%%%%%%%%%%%%%%%%%%%%%%%%%%%%%%%%%%%%%%%%%%%%%%%%%%%%%%%%%%%%%%%%%%

\section*{General Notation}

\begin{vworkmathtermglossaryenum}

\item \mbox{\boldmath $ \vworkdivides $}


      $a \vworkdivides b$, 
      \index{divides@divides ($\vworkdivides$)}
      \index{--@$\vworkdivides$ (divides)}
      read ``\emph{$a$ divides $b$}'', denotes that $b/a$ has no remainder.
      Equivalently, it may be stated that
      $(a \vworkdivides b) \Rightarrow (\exists c \in \vworkintset{}, b = ac)$.

\item \mbox{\boldmath $ \vworknotdivides $}

      $a \vworknotdivides b$, 
      \index{divides@divides ($\vworkdivides$)}
      \index{--@$\vworknotdivides$ (doesn't divide)}
      read ``\emph{$a$ does not divide $b$}'', denotes that $b/a$ has a reminder.
      Equivalently, it may be stated that
      $(a \vworknotdivides b) \Rightarrow (\nexists c \in \vworkintset{}, b = ac)$.

\item \mbox{\boldmath $ \lfloor \cdot \rfloor $}

      Used
      \index{floor function@floor function ($\lfloor\cdot\rfloor$)}
      \index{--@$\lfloor\cdot\rfloor$ (\emph{floor($\cdot$)} function)}
      to denote the \emph{floor($\cdot$)} function.  The
      \emph{floor($\cdot$)}
      function is the largest integer not larger than the
      argument.

\item \mbox{\boldmath $\lceil \cdot \rceil$ }

      Used
      \index{ceiling function@ceiling function ($\lceil\cdot\rceil$)}
      \index{--@$\lceil\cdot\rceil$ (\emph{ceiling($\cdot$)} function)}
      to denote the \emph{ceiling($\cdot$)} function.
      The \emph{ceiling($\cdot$)} function
      is the smallest integer not smaller than the
      argument.
\end{vworkmathtermglossaryenum}

%%%%%%%%%%%%%%%%%%%%%%%%%%%%%%%%%%%%%%%%%%%%%%%%%%%%%%%%%%%%%%%%%%%%%%%%%%%%%%%
%%%%%%%%%%%%%%%%%%%%%%%%%%%%%%%%%%%%%%%%%%%%%%%%%%%%%%%%%%%%%%%%%%%%%%%%%%%%%%%
%%%%%%%%%%%%%%%%%%%%%%%%%%%%%%%%%%%%%%%%%%%%%%%%%%%%%%%%%%%%%%%%%%%%%%%%%%%%%%%

\section*{Usage Of English And Greek Letters}

\begin{vworkmathtermglossaryenum}

\item \mbox {\boldmath $a/b$}

      An arbitrary \index{rational number}rational number.

\item \mbox {\boldmath $ F_N $}

      The \index{Farey series}Farey 
      series of order $N$.  The Farey series is the
      ordered set of irreducible rational numbers 
          in [0,1] with a
      denominator not larger than $N$.

\item \mbox {\boldmath $F_{k_{MAX}, \overline{h_{MAX}}}$}
      
          \index{FKMAXHMAX@$F_{k_{MAX}, \overline{h_{MAX}}}$}
          The ordered set of irreducible rational numbers
          $h/k$ subject to the constraints $0 \leq h \leq h_{MAX}$
          and $1 \leq k \leq h_{MAX}$.  
%         (See Section \cfryzeroxrefhyphen{}\ref{cfry0:schk0}.)


\item \mbox{\boldmath $H/K$}, \mbox{\boldmath $h/k$},
      \mbox{\boldmath $h'/k'$}, \mbox{\boldmath $h''/k''$},
      \mbox{\boldmath $h_i/k_i$}

      Terms in a Farey series of order $N$.

\item \mbox{\boldmath $r_A$}

      The rational number $h/k$ used to approximate
      an arbitrary real number $r_I$.

\item \mbox{\boldmath $r_I$}

      The real number, which may or may not be rational,
      which is to be approximated by a rational number
      $r_A = h/k$.

\item \textbf{reduced}

      See \emph{irreducible}.

\item \mbox{\boldmath $s_k = p_k/q_k$}

      The $k$th convergent of a continued fraction.

\item \mbox{\boldmath $x_{MAX}$}

      The largest element of the domain for which the
      behavior of an approximation must be guaranteed.
      In this paper, most derivations assume
      that $x \in [0, x_{MAX}]$, $x_{MAX} \in \vworkintsetpos{}$.
\end{vworkmathtermglossaryenum}

%%%%%%%%%%%%%%%%%%%%%%%%%%%%%%%%%%%%%%%%%%%%%%%%%%%%%%%%%%%%%%%%%%%%%%%%%%%%%%%
%%%%%%%%%%%%%%%%%%%%%%%%%%%%%%%%%%%%%%%%%%%%%%%%%%%%%%%%%%%%%%%%%%%%%%%%%%%%%%%
%%%%%%%%%%%%%%%%%%%%%%%%%%%%%%%%%%%%%%%%%%%%%%%%%%%%%%%%%%%%%%%%%%%%%%%%%%%%%%%

\section*{Bitfields And Portions Of Integers}

\begin{vworkmathtermglossaryenum}
\item \mbox{\boldmath $a_{b}$}

      The $b$th bit of the integer $a$.  Bits are numbered with the
      least significant bit ``0'', and consecutively through 
      ``$n-1$'', where $n$ is the total number of bits.

      In general, if $p$ is an $n$-bit unsigned integer,

      \begin{equation}
      \nonumber p = \sum_{i=0}^{n-1} 2^i p_i .
      \end{equation}

\item \mbox{\boldmath $a_{c:b}$}

      The integer consisting of the $b$th through the
      $c$th bits of the integer $a$.  Bits are numbered with the
      least significant bit ``0'', and consecutively through 
      ``$n-1$'', where $n$ is the total number of bits.

      For example, if $p$ is a 24-bit unsigned integer, then

      \begin{equation}
      \nonumber p = 2^{16}p_{23:16} + 2^{8}p_{15:8} + p_{7:0} .
      \end{equation}

\item \mbox{\boldmath $a_{[b]}$}

      The $b$th word of the integer $a$.  Words are numbered 
      with the
      least significant word ``0'', and consecutively through 
      ``$n-1$'', where $n$ is the total number of words.

      In general, if $p$ is an $n$-word unsigned integer 
      and $z$ is the wordsize in bits,

      \begin{equation}
      \nonumber p = \sum_{i=0}^{n-1} 2^{iz} p_i .
      \end{equation}

\item \mbox{\boldmath $a_{[c:b]}$}

      The integer consisting of the $b$th through the
      $c$th word of the integer $a$.  Words are numbered with the
      least significant word ``0'', and consecutively through 
      ``$n-1$'', where $n$ is the total number of words.

      For example, if $p$ is a 24-word unsigned integer and
      $z$ is the wordsize in bits, then

      \begin{equation}
      \nonumber p = 2^{16z}p_{[23:16]} + 2^{8z}p_{[15:8]} + p_{[7:0]} .
      \end{equation}

\end{vworkmathtermglossaryenum}

%%%%%%%%%%%%%%%%%%%%%%%%%%%%%%%%%%%%%%%%%%%%%%%%%%%%%%%%%%%%%%%%%%%%%%%%%%%%%%%
%%%%%%%%%%%%%%%%%%%%%%%%%%%%%%%%%%%%%%%%%%%%%%%%%%%%%%%%%%%%%%%%%%%%%%%%%%%%%%%
%%%%%%%%%%%%%%%%%%%%%%%%%%%%%%%%%%%%%%%%%%%%%%%%%%%%%%%%%%%%%%%%%%%%%%%%%%%%%%%

\section*{Matrices And Vectors}

\begin{vworkmathtermglossaryenum}

\item \mbox{\boldmath $0$}

      $\mathbf{0}$ (in bold face) is used to denote either a vector or matrix
      populated with all zeroes.  Optionally, in cases where the context is not clear
      or where there is cause to highlight the dimension, $\mathbf{0}$ may be subscripted
      to indicate the dimension, i.e. 
      
      \begin{equation}
      \nonumber
      \mathbf{0}_3 = \left[\begin{array}{c} 0 \\ 0 \\ 0 \end{array}\right]
      \end{equation}

      \begin{equation}
      \nonumber
      \mathbf{0}_{3 \times 2} = \left[\begin{array}{cc} 0&0 \\ 0&0 \\ 0&0 \end{array}\right]
      \end{equation}

\item \mbox{\boldmath $I$}

      $I$ is used to denote the square identity matrix (the matrix with all
      elements 0 except elements on the diagonal which are 1).
      Optionally, in cases where the context is not clear
      or where there is cause to highlight the dimension, $I$ may be subscripted
      to indicate the dimension, i.e. 
      
      \begin{equation}
      \nonumber
      I = I_3 = I_{3 \times 3} = \left[\begin{array}{ccc} 1&0&0 \\ 0&1&0 \\ 0&0&1 \end{array}\right]
      \end{equation}

\end{vworkmathtermglossaryenum}


%%%%%%%%%%%%%%%%%%%%%%%%%%%%%%%%%%%%%%%%%%%%%%%%%%%%%%%%%%%%%%%%%%%%%%%%%%%%%%%
%%%%%%%%%%%%%%%%%%%%%%%%%%%%%%%%%%%%%%%%%%%%%%%%%%%%%%%%%%%%%%%%%%%%%%%%%%%%%%%
%%%%%%%%%%%%%%%%%%%%%%%%%%%%%%%%%%%%%%%%%%%%%%%%%%%%%%%%%%%%%%%%%%%%%%%%%%%%%%%

\section*{Sets And Set Notation}

\begin{vworkmathtermglossaryenum}

\item \mbox{\boldmath $n(A)$}

      The \index{cardinality}cardinality of set $A$.  (The cardinality of a set is the
      number of elements in the set.)

\end{vworkmathtermglossaryenum}

%%%%%%%%%%%%%%%%%%%%%%%%%%%%%%%%%%%%%%%%%%%%%%%%%%%%%%%%%%%%%%%%%%%%%%%%%%%%%%%
%%%%%%%%%%%%%%%%%%%%%%%%%%%%%%%%%%%%%%%%%%%%%%%%%%%%%%%%%%%%%%%%%%%%%%%%%%%%%%%
%%%%%%%%%%%%%%%%%%%%%%%%%%%%%%%%%%%%%%%%%%%%%%%%%%%%%%%%%%%%%%%%%%%%%%%%%%%%%%%

\section*{Sets Of Numbers}

\begin{vworkmathtermglossaryenum}

\item \mbox{\boldmath $\vworkintsetpos$}

      The 
      \index{natural number}
      \index{N@$\vworkintsetpos$}
      set of positive integers (natural numbers).

\item \mbox{\boldmath $\vworkratset$}

      The 
      \index{rational number}
      \index{Q@$\vworkratset$}
      set of rational numbers.

\item \mbox{\boldmath $\vworkratsetnonneg$}

      The 
      \index{rational number}
      \index{Q+@$\vworkratsetnonneg$}
      set of non-negative rational numbers.

\item \mbox{\boldmath $\vworkrealset$}

      The 
      \index{real number}
      \index{R@$\vworkrealset$}
      set of real numbers.

\item \mbox{\boldmath $\vworkrealsetnonneg$}

      The 
      \index{real number}
      \index{R+@$\vworkrealsetnonneg$}
      set of non-negative real numbers.

\item \mbox{\boldmath $\vworkintset$}

      The 
      \index{integer}
      \index{Z@$\vworkintset$}
      set of integers.

\item \mbox{\boldmath $\vworkintsetnonneg$}

      The 
      \index{integer}
      \index{Z+@$\vworkintsetnonneg$}
      set of non-negative integers.

\end{vworkmathtermglossaryenum}


%%%%%%%%%%%%%%%%%%%%%%%%%%%%%%%%%%%%%%%%%%%%%%%%%%%%%%%%%%%%%%%%%%%%%%%%%%%%%%%
%%%%%%%%%%%%%%%%%%%%%%%%%%%%%%%%%%%%%%%%%%%%%%%%%%%%%%%%%%%%%%%%%%%%%%%%%%%%%%%
%%%%%%%%%%%%%%%%%%%%%%%%%%%%%%%%%%%%%%%%%%%%%%%%%%%%%%%%%%%%%%%%%%%%%%%%%%%%%%%

\noindent\begin{figure}[!b]
\noindent\rule[-0.25in]{\textwidth}{1pt}
\begin{tiny}
\begin{verbatim}
$RCSfile: c_glo1.tex,v $
$Source: /home/dashley/cvsrep/uculib01/uculib01/doc/manual/c_glo1/c_glo1.tex,v $
$Revision: 1.2 $
$Author: dashley $
$Date: 2010/01/28 21:18:32 $
\end{verbatim}
\end{tiny}
\noindent\rule[0.25in]{\textwidth}{1pt}
\end{figure}

%%%%%%%%%%%%%%%%%%%%%%%%%%%%%%%%%%%%%%%%%%%%%%%%%%%%%%%%%%%%%%%%%%%%%%%%%%%%%%%
%$Log: c_glo1.tex,v $
%Revision 1.2  2010/01/28 21:18:32  dashley
%a)Chapter start quotes removed.
%b)Aesthetic comment line added at the bottom of most files.
%
%Revision 1.1  2007/08/30 14:43:38  dtashley
%Initial checkin.
%
%End of file $RCSfile: c_glo1.tex,v $.
%%%%%%%%%%%%%%%%%%%%%%%%%%%%%%%%%%%%%%%%%%%%%%%%%%%%%%%%%%%%%%%%%%%%%%%%%%%%%%%

