%$Header: /home/dashley/cvsrep/uculib01/uculib01/doc/manual/c_iov0/c_iov0.tex,v 1.29 2010/06/11 15:42:50 dashley Exp $

\chapter{Introduction and Overview}

\label{ciov0}



%%%%%%%%%%%%%%%%%%%%%%%%%%%%%%%%%%%%%%%%%%%%%%%%%%%%%%%%%%%%%%%%%%%%%%%%%%%%%%%
%%%%%%%%%%%%%%%%%%%%%%%%%%%%%%%%%%%%%%%%%%%%%%%%%%%%%%%%%%%%%%%%%%%%%%%%%%%%%%%
%%%%%%%%%%%%%%%%%%%%%%%%%%%%%%%%%%%%%%%%%%%%%%%%%%%%%%%%%%%%%%%%%%%%%%%%%%%%%%%
\section{License}
\label{ciov0:slic0}

TBD.


%%%%%%%%%%%%%%%%%%%%%%%%%%%%%%%%%%%%%%%%%%%%%%%%%%%%%%%%%%%%%%%%%%%%%%%%%%%%%%%
%%%%%%%%%%%%%%%%%%%%%%%%%%%%%%%%%%%%%%%%%%%%%%%%%%%%%%%%%%%%%%%%%%%%%%%%%%%%%%%
%%%%%%%%%%%%%%%%%%%%%%%%%%%%%%%%%%%%%%%%%%%%%%%%%%%%%%%%%%%%%%%%%%%%%%%%%%%%%%%
\section{Overview of \emph{\productbasenameshort{}}}
%Section tag:  pov0
\label{ciov0:spov0}

The \emph{\productbasenamelong{} Version \productversion{}}
(or more compactly, 
\emph{\productbasenameshort{}-\productversion{}})
is an
open-source utility library for inexpensive 8-bit, 16-bit, and 32-bit 
microcontrollers and microprocessors.

The functions provided in \emph{\productbasenameshort{}} fall into these categories
(with some classification ambiguity and overlap).

\begin{itemize}
\item Arithmetic.
\item Bounded arithmetic.
\item Fixed-point arithmetic.
\item Large-operand and extended-precision arithmetic.
\item Block memory operations.
\item Bit-mapped set functions.
\item Searching.
\item Sorting.
\item Array manipulation.
\item Linear filters.
\item Non-linear filters.
\item Vertical counters.
\item Control system components.
\item CRC, checksums, and non-cryptographic hashes.
\item Cryptographic hashes.
\item Ciphers.
\item Miscellaneous functions.
\item Utility functions.
\item Speed-enhanced development tool replacement functions.
\end{itemize}

\emph{\productbasenameshort{}-\productversion{}}
is packaged as two libraries, each with
a different purpose:

\begin{itemize}
\item The \index{general library}``general library'' contains functions not normally provided
      with development tools.  The general library is described by
      every chapter in this document except Chapter \ref{csef0}.
\item The \index{replacement library}``replacement library'' (not yet implemented
      for any platform) contains
      speed-enhanced replacements
      for functions packaged with development tools, in those cases
      where some optimization could be performed or where an alternate
      and faster algorithm is known.  The replacement library is covered by 
      Chapter \ref{csef0}.  Functions in the replacement library
      generally are designed to boost speed at the expense of program
      memory and/or RAM\@.  Use of the replacement library is optional.  
      The replacement library may not exist for all platforms.
\end{itemize}


%%%%%%%%%%%%%%%%%%%%%%%%%%%%%%%%%%%%%%%%%%%%%%%%%%%%%%%%%%%%%%%%%%%%%%%%%%%%%%%
%%%%%%%%%%%%%%%%%%%%%%%%%%%%%%%%%%%%%%%%%%%%%%%%%%%%%%%%%%%%%%%%%%%%%%%%%%%%%%%
%%%%%%%%%%%%%%%%%%%%%%%%%%%%%%%%%%%%%%%%%%%%%%%%%%%%%%%%%%%%%%%%%%%%%%%%%%%%%%%
\section{Supported CPU Cores and CPU Core Variants}
%Section tag:  lcv0
\label{ciov0:sscv0}

The CPU core and the development tool chain are closely tied together in terms
of the usability of a library.  A library compiled for a different CPU core
usually won't function, but a library compiled using a different set of
development tools may also not function (due to differences in object file formats,
differences in calling conventions, etc.).  For this reason,
the CPU core and development tool chain are bundled together (see Table \ref{tbl:ciov0:sscv0:01})
and sometimes referred to only as
``CPU core''.

\begin{table}
\caption{Supported CPU Cores and Development Tool Chains}
\label{tbl:ciov0:sscv0:01}
\begin{center}
\begin{tabular}{|c|l|l|l|}
\hline
Integer & CPU Core          & Development            & Remarks                               \\
Code    &                   & Tool Chain             &                                       \\
\hline
\hline
   1    & ST STM8           & Cosmic                 & ST Microelectronics STM8 with         \\
        &                   &                        & Cosmic Software development tools.    \\
\hline
   2    & Freescale CPU08   & Cosmic                 & Freescale CPU08 with                  \\
        &                   &                        & Cosmic Software development tools.    \\
\hline
\end{tabular}
\end{center}
\end{table}

Table \ref{tbl:ciov0:sscv0:01} lists the CPU cores and development
tool chain combinations supported by
\emph{\productbasenameshort{}-\productversion{}}\@.  The
integer codes listed in Table \ref{tbl:ciov0:sscv0:01}
is the code returned by the
\index{UcuMfLibVerCpuRxx()@\emph{UcuMfLibVerCpuRxx($\cdot$)}}\emph{UcuMfLibVerCpuRxx($\cdot$)}
function (\S{}\ref{cmsc0:slvc0}, p. \pageref{cmsc0:slvc0}).

\begin{table}
\caption{CPU Core Variants for ST STM8 / Cosmic}
\label{tbl:ciov0:sscv0:02}
\begin{center}
\begin{tabular}{|c|l|}
\hline
Variant           & Description                                                              \\
Integer           &                                                                          \\
Code              &                                                                          \\
\hline
\hline
1                 & Base variant.                                                            \\
\hline
\end{tabular}
\end{center}
\end{table}

Within the STM8/Cosmic CPU core / development tool combination, 
Table \ref{tbl:ciov0:sscv0:02} provides the CPU core variant codes.
(\emph{Note:}
at the present time, only functions for the STM8 using the Cosmic tool chain
are implemented.)

\begin{table}
\caption{CPU Core Variants for Freescale CPU08 / Cosmic}
\label{tbl:ciov0:sscv0:03}
\begin{center}
\begin{tabular}{|c|l|}
\hline
Variant           & Description                                                              \\
Integer           &                                                                          \\
Code              &                                                                          \\
\hline
\hline
1                 & Base variant.                                                            \\
\hline
\end{tabular}
\end{center}
\end{table}

Within the CPU08 / Cosmic CPU core / development tool combination, 
Table \ref{tbl:ciov0:sscv0:03} provides the CPU core variant codes.
(\emph{Note:}  at the present time, no functions for the CPU08 are
implemented.)


%%%%%%%%%%%%%%%%%%%%%%%%%%%%%%%%%%%%%%%%%%%%%%%%%%%%%%%%%%%%%%%%%%%%%%%%%%%%%%%
%%%%%%%%%%%%%%%%%%%%%%%%%%%%%%%%%%%%%%%%%%%%%%%%%%%%%%%%%%%%%%%%%%%%%%%%%%%%%%%
%%%%%%%%%%%%%%%%%%%%%%%%%%%%%%%%%%%%%%%%%%%%%%%%%%%%%%%%%%%%%%%%%%%%%%%%%%%%%%%
\section{Library Conventions}
%Section tag:  lcv0
\label{ciov0:slcv0}


%%%%%%%%%%%%%%%%%%%%%%%%%%%%%%%%%%%%%%%%%%%%%%%%%%%%%%%%%%%%%%%%%%%%%%%%%%%%%%%
%%%%%%%%%%%%%%%%%%%%%%%%%%%%%%%%%%%%%%%%%%%%%%%%%%%%%%%%%%%%%%%%%%%%%%%%%%%%%%%
%%%%%%%%%%%%%%%%%%%%%%%%%%%%%%%%%%%%%%%%%%%%%%%%%%%%%%%%%%%%%%%%%%%%%%%%%%%%%%%
\subsection{Library Versioning}
%Section tag:  lvn0
\label{ciov0:slcv0:slvn0}

\index{versioning (library)}Version numbers of the library are of the
form \emph{j.m-} (example: \emph{2.5c}) where:

\begin{itemize}
\item \emph{j} is the major version number, a non-negative integer.
      This version number is returned by the
      \index{UcuMfLibVerMajRxx()@\emph{UcuMfLibVerMajRxx($\cdot$)}}\emph{UcuMfLibVerMajRxx($\cdot$)}
      function (\S{}\ref{cmsc0:slvf0}, p. \pageref{cmsc0:slvf0}).
\item \emph{m} is the minor version number, a non-negative integer.
      This version number is returned by the
      \index{UcuMfLibVerMinMicRxx()@\emph{UcuMfLibVerMinMicRxx($\cdot$)}}\emph{UcuMfLibVerMinMicRxx($\cdot$)}
      function (\S{}\ref{cmsc0:slvm0}, p. \pageref{cmsc0:slvm0}).
\item \emph{-} is the micro version number, a non-negative integer.  The micro version number
      is represented as a letter in documentation (for example, the `c' in `2.5c').
      This version number is returned by the
      \index{UcuMfLibVerMinMicRxx()@\emph{UcuMfLibVerMinMicRxx($\cdot$)}}\emph{UcuMfLibVerMinMicRxx($\cdot$)}
      function (\S{}\ref{cmsc0:slvm0}, p. \pageref{cmsc0:slvm0}) as an 8-bit integer.
      When represented as an integer, the correspondence is that the letter `a' is represented by 0,
      the letter `b' by 1, etc.
\end{itemize}


%%%%%%%%%%%%%%%%%%%%%%%%%%%%%%%%%%%%%%%%%%%%%%%%%%%%%%%%%%%%%%%%%%%%%%%%%%%%%%%
%%%%%%%%%%%%%%%%%%%%%%%%%%%%%%%%%%%%%%%%%%%%%%%%%%%%%%%%%%%%%%%%%%%%%%%%%%%%%%%
%%%%%%%%%%%%%%%%%%%%%%%%%%%%%%%%%%%%%%%%%%%%%%%%%%%%%%%%%%%%%%%%%%%%%%%%%%%%%%%
\subsection{Library Binary File Naming Conventions}
%Section tag:  lpn0
\label{ciov0:slcv0:slpn0}


\index{naming conventions (library)}%
A typical binary library file name is ``\emph{libucu\_g\_2\_5c\_1\_1.sm8}''.
In order, the binary library file name contains the following components:

\begin{itemize}
\item The characters \emph{libucu\_} (a constant string
      to identify the library).
\item \emph{g\_} or \emph{r\_} to indicate whether the library
      is the `general' or the `replacement' library.
\item The version number of the library followed by an underscore
      (example: `\emph{2\_5c\_}').
\item The CPU core integer identifier, followed by an underscore.  (This integer
      is selected from Table \ref{tbl:ciov0:sscv0:01}.)
\item The CPU core variant integer identifier.  (This integer is selected from the table 
      of variants specific to the CPU core:  for example, Table \ref{tbl:ciov0:sscv0:02} or
      Table \ref{tbl:ciov0:sscv0:03}.)
\item A period followed by a filename extension traditional for the
      development tools and that CPU core and CPU core variant.
\end{itemize}


%%%%%%%%%%%%%%%%%%%%%%%%%%%%%%%%%%%%%%%%%%%%%%%%%%%%%%%%%%%%%%%%%%%%%%%%%%%%%%%
%%%%%%%%%%%%%%%%%%%%%%%%%%%%%%%%%%%%%%%%%%%%%%%%%%%%%%%%%%%%%%%%%%%%%%%%%%%%%%%
%%%%%%%%%%%%%%%%%%%%%%%%%%%%%%%%%%%%%%%%%%%%%%%%%%%%%%%%%%%%%%%%%%%%%%%%%%%%%%%
\subsection{Function Naming Conventions}
%Section tag:  fnc0
\label{ciov0:slcv0:sfnc0}

A typical function name within the general library is
`\emph{UcuLfU16FiltAFiltRxx}'.  Function names within the general library
usually consist of the following components, with some components optional:

\begin{itemize}
\item The character string `\emph{Ucu}'.
\item Two characters representing the function family (selected from 
      Table \ref{tbl:ciov0:slcv0:sfnc0:03}).
\item The fundamental data type on which the function operates and/or returns.  (\emph{U16} means
      a 16-bit unsigned integer, \emph{U16a} means an array of unsigned integers, etc.)
\item A name for the function, indicating what it does.
\item Qualifiers for the function.  These may include input argument sizes, in the case that
      there are variants of the function that accept different input argument sizes.  These
      may also include different levels of sophistication in handling invalid or sentinel
      values of inputs.
\item The interrupt compatibility code (Table \ref{tbl:ciov0:slcv0:sfnc0:01}), capitalized.
\item The interrupt mask modification code (Table \ref{tbl:ciov0:slcv0:sfnc0:02}), in lower case.
\item The shared data modification code (Table \ref{tbl:ciov0:slcv0:sfnc0:02b}), in lower case.
\end{itemize}

\begin{table}
\caption{Function Family Code}
\label{tbl:ciov0:slcv0:sfnc0:03}
\begin{center}
\begin{tabular}{|l|l|}
\hline
Function       & Meaning                                                                     \\
Family         &                                                                             \\
Code           &                                                                             \\
\hline
\hline
\texttt{At}    & Arithmetic function (Chapter \ref{cafn0}).                                  \\
\hline
\texttt{Ba}    & Bounded arithmetic function (Chapter \ref{cbaf0}).                          \\
\hline
\texttt{Fp}    & Fixed-Point arithmetic function (Chapter \ref{cfpa0}).                      \\
\hline
\texttt{Lo}    & Large-operand/extended precision function (Chapter \ref{claf0}).            \\
\hline
\texttt{Bm}    & Block memory function (Chapter \ref{cbmf0}).                                \\
\hline
\texttt{Bt}    & Bit-mapped set function (Chapter \ref{cbsf0}).                              \\
\hline
\texttt{Se}    & Search function (Chapter \ref{csea0}).                                      \\
\hline
\texttt{So}    & Sort function (Chapter \ref{csol0}).                                        \\
\hline
\texttt{Am}    & Array manipulation function (Chapter \ref{cami0}).                          \\
\hline
\texttt{Lf}    & Linear filter function (Chapter \ref{clfi0}).                               \\
\hline
\texttt{Nf}    & Non-linear filter function (Chapter \ref{cnfi0}).                           \\
\hline
\texttt{Vc}    & Vertical counter function (Chapter \ref{cvco0}).                            \\
\hline
\texttt{Cs}    & Control system component function (Chapter \ref{ccso0}).                    \\
\hline
\texttt{Cc}    & CRC/checksum/non-cryptographic hash function                                \\
               & (Chapter \ref{ccrc0}).                                                      \\
\hline
\texttt{Ch}    & Cryptographic hash function (Chapter \ref{ccrh0}).                          \\
\hline
\texttt{Ci}    & Cipher function (Chapter \ref{ccip0}).                                      \\
\hline
\texttt{Mi}    & Miscellaneous function (Chapter \ref{cmsc0}).                               \\
\hline
\texttt{Ut}    & Utility function (Chapter \ref{cnef0}).                                     \\
\hline
\end{tabular}
\end{center}
\end{table}

\begin{table}
\caption{Interrupt Compatibility Codes}
\label{tbl:ciov0:slcv0:sfnc0:01}
\begin{center}
\begin{tabular}{|l|l|}
\hline
Interrupt     & Meaning                                                        \\
Compatibility &                                                                \\
Code          &                                                                \\
\hline
\hline
\texttt{n}    & \textbf{non-ISR Usage Only}                                    \\
              & The function may be called from non-interrupt service          \\
              & only, and may not be called from an interrupt service          \\
              & routine.                                                       \\
              &                                                                \\
              & (This code gives no information about whether the data         \\
              & operated on by a function may be shared with an                \\
              & ISR, and the function documentation must be consulted          \\
              & to determine this.)                                            \\
\hline
\texttt{i}    & \textbf{ISR Usage Only}                                        \\
              & The function may be called from interrupt service only,        \\
              & and may not be called from non-interrupt service.              \\
\hline
\texttt{r}    & \textbf{Re-Entrant}                                            \\
              & The function may be called from both non-interrupt service     \\
              & and interrupt service.                                         \\
\hline
\end{tabular}
\end{center}
\end{table}

\begin{table}
\caption{Interrupt Mask Modification Code}
\label{tbl:ciov0:slcv0:sfnc0:02}
\begin{center}
\begin{tabular}{|l|l|}
\hline
Interrupt     & Meaning                                                          \\
Compatibility &                                                                  \\
Code          &                                                                  \\
\hline
\hline
\texttt{x}    & \textbf{Not Modified}                                            \\
              & The global interrupt mask is not modified.  This often           \\
              & means that the function may be called with interrupts            \\
              & either disabled or enabled (consult the function                 \\
              & documentation and \S{}\ref{ctbg0:sici0}).                        \\
\hline
\texttt{c}    & \textbf{Classic Critical Sections}                               \\
              & \emph{Classic} critical sections (DI/EI) are used.  (See         \\
              & \S{}\ref{ctbg0:sici0}.)                                          \\
\hline
\texttt{r}    & \textbf{Restorative}                                             \\
              & The global interrupt mask is saved and restored.  This           \\
              & often means that the function can be safely called with          \\
              & interrupts disabled or enabled and/or that the function          \\
              & may be part of a larger critical section that (at the            \\
              & very beginning and end) uses the classis DI/EI protocol.         \\
              & (Consult the function documentation and \S{}\ref{ctbg0:sici0}.)  \\
\hline
\texttt{u}    & \textbf{Universally Restorative}                                 \\
              & Either ``x'' or ``r'' applies, but the documentation can't       \\
              & commit because it may vary from processor to processor           \\
              & (depending on the processor's instruction set).  This            \\
              & often means that the function can be safely called with          \\
              & interrupts disabled or enabled and/or that the function          \\
              & may be part of a larger critical section that (at the            \\
              & very beginning and end) uses the classis DI/EI protocol.         \\
              & (Consult the function documentation and \S{}\ref{ctbg0:sici0}.)  \\
\hline
\end{tabular}
\end{center}
\end{table}

\begin{table}
\caption{Shared Data Mutual Exclusion Code}
\label{tbl:ciov0:slcv0:sfnc0:02b}
\begin{center}
\begin{tabular}{|l|l|}
\hline
Interrupt     & Meaning                                                          \\
Compatibility &                                                                  \\
Code          &                                                                  \\
\hline
\hline
\texttt{n}    & \textbf{No Protection}                                           \\
              & Accessed data is not protected against concurrent access         \\
              & by another thread.                                               \\
\hline
\texttt{c}    & \textbf{Component Protection}                                    \\
              & Individual array elements are protected, but the array           \\
              & as a whole is not.                                               \\
\hline
\texttt{f}    & \textbf{Full Protection}                                         \\
              & The full data structure is protected.                            \\
\hline
\texttt{x}    & \textbf{Not Applicable}                                          \\
              & There is no shared data to protect.                              \\
\hline
\end{tabular}
\end{center}
\end{table}



%%%%%%%%%%%%%%%%%%%%%%%%%%%%%%%%%%%%%%%%%%%%%%%%%%%%%%%%%%%%%%%%%%%%%%%%%%%%%%%
%%%%%%%%%%%%%%%%%%%%%%%%%%%%%%%%%%%%%%%%%%%%%%%%%%%%%%%%%%%%%%%%%%%%%%%%%%%%%%%
%%%%%%%%%%%%%%%%%%%%%%%%%%%%%%%%%%%%%%%%%%%%%%%%%%%%%%%%%%%%%%%%%%%%%%%%%%%%%%%
\section{Library Revision History}
\label{ciov0:slrh0}

\noindent\textbf{0.1e} (June 11, 2010)
\begin{itemize}
\item Addition of \emph{UcuMfCpuCcrSetRxx($\cdot$)} function.
\end{itemize}

\noindent\textbf{0.1d} (May 18, 2010)
\begin{itemize}
\item Addition of \emph{UcuBaS8AbsBoundedRxx($\cdot$)} function.
\item Addition of \emph{UcuBaS16AbsBoundedRxx($\cdot$)} function.
\item Addition of \emph{UcuBaS16DiffBoundedRxx($\cdot$)} function.
\item Addition of \emph{UcuBtU32RotLeftNInPlaceRxx($\cdot$)} function.
\end{itemize}

\noindent\textbf{0.1c} (March, 2010)
\begin{itemize}
\item Initial release.
\end{itemize}


%%%%%%%%%%%%%%%%%%%%%%%%%%%%%%%%%%%%%%%%%%%%%%%%%%%%%%%%%%%%%%%%%%%%%%%%%%
\noindent\begin{figure}[!b]
\noindent\rule[-0.25in]{\textwidth}{1pt}
\begin{tiny}
\begin{verbatim}
$RCSfile: c_iov0.tex,v $
$Source: /home/dashley/cvsrep/uculib01/uculib01/doc/manual/c_iov0/c_iov0.tex,v $
$Revision: 1.29 $
$Author: dashley $
$Date: 2010/06/11 15:42:50 $
\end{verbatim}
\end{tiny}
\noindent\rule[0.25in]{\textwidth}{1pt}
\end{figure}

%%%%%%%%%%%%%%%%%%%%%%%%%%%%%%%%%%%%%%%%%%%%%%%%%%%%%%%%%%%%%%%%%%%%%%%%%%%%%%%
%$Log: c_iov0.tex,v $
%Revision 1.29  2010/06/11 15:42:50  dashley
%Version number bump to 0.1e.
%
%Revision 1.28  2010/05/18 22:25:12  dashley
%Release date for version 0.1d added.
%
%Revision 1.27  2010/05/12 22:24:57  dashley
%Addition of UcuBtU32RotLeftNInPlaceRxx() function.
%
%Revision 1.26  2010/05/12 21:21:02  dashley
%Unimplemented functions removed.
%
%Revision 1.25  2010/04/16 19:33:37  dashley
%Addition of UcuAtS32NegationRxx() function.
%
%Revision 1.24  2010/04/16 19:19:40  dashley
%Unused absolute value function removed from documentation.
%
%Revision 1.23  2010/04/15 19:34:39  dashley
%Addition of UcuAtU32S16v2MagSquaredRxx() function.
%
%Revision 1.22  2010/04/15 17:54:10  dashley
%Addition of UcuBaS32AbsBoundedRxx() function.
%
%Revision 1.21  2010/04/15 17:34:52  dashley
%Addition of UcuAtS32IsNegRxx() function.
%
%Revision 1.20  2010/04/15 17:09:38  dashley
%Addition of UcuLoU32U64U32DivQFRxn() function.
%
%Revision 1.19  2010/04/15 15:56:41  dashley
%Addition of UcuU64U32SquareInPlaceRxn() function.
%
%Revision 1.18  2010/04/15 14:56:28  dashley
%Addition of UcuAtU32SqrtFRxx() function.
%
%Revision 1.17  2010/03/18 00:01:10  dashley
%Functions added.
%
%Revision 1.16  2010/03/16 21:57:45  dashley
%Addition of license section.
%
%Revision 1.15  2010/03/16 21:56:02  dashley
%Edits and corrections.
%
%Revision 1.14  2010/01/31 02:00:29  dashley
%Minor change in two-character naming code for bit-mapped set functions.
%
%Revision 1.13  2010/01/28 21:18:32  dashley
%a)Chapter start quotes removed.
%b)Aesthetic comment line added at the bottom of most files.
%
%Revision 1.12  2010/01/27 21:52:15  dashley
%Edits.
%
%Revision 1.11  2010/01/26 22:05:21  dashley
%Edits.
%
%Revision 1.10  2010/01/26 21:56:25  dashley
%Edits.
%
%Revision 1.9  2010/01/26 16:16:18  dashley
%Edits.
%
%Revision 1.8  2007/10/10 18:28:33  dtashley
%Addition of UcuBmMemclrSmNu() function.
%
%Revision 1.7  2007/10/10 17:42:27  dtashley
%Edits.
%
%Revision 1.6  2007/10/08 19:51:59  dtashley
%Edits toward getting function documentation looking good.
%
%Revision 1.5  2007/10/08 18:16:33  dtashley
%Edits.
%End of $RCSfile: c_iov0.tex,v $.
%%%%%%%%%%%%%%%%%%%%%%%%%%%%%%%%%%%%%%%%%%%%%%%%%%%%%%%%%%%%%%%%%%%%%%%%%%%%%%%

