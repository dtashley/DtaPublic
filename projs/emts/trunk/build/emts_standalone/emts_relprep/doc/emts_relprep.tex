\documentclass[letterpaper,10pt,titlepage]{article}
%
%\pagestyle{headings}
%
\usepackage{amsmath}
\usepackage{amsfonts}
\usepackage{amssymb}
\usepackage[ansinew]{inputenc}
\usepackage[OT1]{fontenc}
\usepackage{graphicx}
%
\begin{document}
\title{\emph{emts\_relprep} Release Preparation Tool}
\author{\vspace{30mm}\\David T. Ashley\\\texttt{dashley@gmail.com}\\\vspace{1cm}}
\date{\vspace*{38mm}\small{\LaTeX{} Compilation Date: \today{}}}
\maketitle

%%%%%%%%%%%%%%%%%%%%%%%%%%%%%%%%%%%%%%%%%%%%%%%%%%%%%%%%%%%%%%%%%%%%%%%%%%%%%%%
%
\begin{abstract}
This document describes the \emph{emts\_relprep} (mnemonic:  
\emph{EMTS} \emph{rel}ease
\emph{prep}aration) tool.

\emph{emts\_relprep} performs many of the mundane tasks associated with
a release of \emph{The EMTS Tool Set}.
\end{abstract}

%%%%%%%%%%%%%%%%%%%%%%%%%%%%%%%%%%%%%%%%%%%%%%%%%%%%%%%%%%%%%%%%%%%%%%%%%%%%%%%

\section{Introduction and Overview}
\label{siov0}

This document describes the \emph{emts\_relprep}\footnote{Mnemonic:  
\emph{EMTS} \emph{rel}ease
\emph{prep}aration.} tool used for performing many of the 
mundane tasks associated with a release of \emph{The EMTS Tool Set}.
The tool can also be used during development, as the tasks it performs
are appropriate at times other than software release.

In large part, the tool performs tasks that are helpful due to \emph{git}'s
lack of keyword expansion capability.\footnote{As will be discussed later in
the document, the tool uses a method for assigning successive revision numbers
to files.}
  
When invoked, the tool uses a configuration file to determine what steps to
perform and in what order.  The format of the configuration file is
described in \S{}\ref{scff0}.

The tool buffers an entire file into memory, then performs the requested
operations on the memory buffer, then rewrites the potentially modified buffer
to disk.  A consequence of this strategy is that each file on which operations
are performed must be able to comfortably fit within the heap that can be
allocated to a program.  Since a modern computer typically has 16 gigabytes or
more of RAM and since a typical text file does not exceed a megabyte, problems
should be rare; but they could conceptually exist with large text files.
The program will terminate with a meaningful error message if heap is exhausted.

At the time of this writing, the program builds only for \emph{Windows}, but
this may change in the future.
  
%%%%%%%%%%%%%%%%%%%%%%%%%%%%%%%%%%%%%%%%%%%%%%%%%%%%%%%%%%%%%%%%%%%%%%%%%%%%%%%
\section{Invocation and Command-Line Arguments}
\label{sicl0}

TBD.

%%%%%%%%%%%%%%%%%%%%%%%%%%%%%%%%%%%%%%%%%%%%%%%%%%%%%%%%%%%%%%%%%%%%%%%%%%%%%%%
\section{Keywords and Keyword Expansions}
\label{skwe0}

TBD.

%%%%%%%%%%%%%%%%%%%%%%%%%%%%%%%%%%%%%%%%%%%%%%%%%%%%%%%%%%%%%%%%%%%%%%%%%%%%%%%
\section{How File Version Numbers Are Maintained}
\label{shfv0}

In order to synthetically (without the assistance of \emph{git}) maintain a
file version number, the following approach is used:

\begin{itemize}
\item If the \emph{\$SHA512DigestLineExclusive:\$} keyword is present within the file
      being operated on, AND
\item The \emph{\$FileVersionSynthetic:\$} keyword is also present within the
      file being operated on, AND
\item The \emph{sha512digestlineexclusive\_update} and \\
      \emph{fileversionsynthetic\_bump\_conditional} both appear as file
      commands in the same file operation, with the former before the latter, THEN
\item The \emph{\$FileVersionSynthetic:\$} keyword expansion will be incremented
      whenever the target file changes and the program is run.
\end{itemize}

When the \emph{sha512digestlineexclusive\_update} command runs and must update
\emph{\$SHA512DigestLineExclusive:\$} keyword expansion(s) because the file
has changed, a flag is set.  This flag is later tested by the
\emph{fileversionsynthetic\_bump\_conditional} command, and the
\emph{\$FileVersionSynthetic:\$} keyword expansions are incremented by one
only if the \emph{\$SHA512DigestLineExclusive:\$} changed (meaning that the
file contents have changed).

TODO:  Change the SHA-512 keyword to only include the keyword expansion itself,
and to possibly include other areas of the same line.

TODO:  Keywords need a parenthesized form with KEYWORD(i, j, k), where i
is the number of lines down in the file to substitute, j is the starting column, and k is the
space available.  This would allow a comment in the line above to modify a string a line (or 
a few lines) below, without the baggage of having the keyword itself as part of the string.
The k value would prevent overflowing a string literal.

%%%%%%%%%%%%%%%%%%%%%%%%%%%%%%%%%%%%%%%%%%%%%%%%%%%%%%%%%%%%%%%%%%%%%%%%%%%%%%%
\section{Configuration File Contents and Format}
\label{scff0}

%%%%%%%%%%%%%%%%%%%%%%%%%%%%%%%%%%%%%%%%%%%%%%%%%%%%%%%%%%%%%%%%%%%%%%%%%%%%%%%
\subsection{General Formatting}
\label{scff0:sgfm0}

The configuration file is an ASCII text file.

Comments begin with `\#', and then the rest of the line is ignored.

Lines are continued with `\textbackslash{}'.  The `\textbackslash{}' need not be the last character of the line.
Exception:  comments may not be continued in this way.

The configuration file is buffered in its entirety before the program
begins processing, so any modifications to the file won't take effect until the
program is run again.

All file paths are with the forward slash.  Double-dot may be used.  Single dot may not.

File and path names may contains spaces and may be double-quoted.

Wildcards may not be used.

Unicode not supported.

Format is \emph{command pars}.

Commands are processed in order, once.


%%%%%%%%%%%%%%%%%%%%%%%%%%%%%%%%%%%%%%%%%%%%%%%%%%%%%%%%%%%%%%%%%%%%%%%%%%%%%%%
\subsection{How the Configuration File is Processed}
\label{scff0:shcp0}

TBD.


%%%%%%%%%%%%%%%%%%%%%%%%%%%%%%%%%%%%%%%%%%%%%%%%%%%%%%%%%%%%%%%%%%%%%%%%%%%%%%%
\subsection{The \emph{SRV} Command}
\label{scff0:ssrv0}

Sets the release version.


%%%%%%%%%%%%%%%%%%%%%%%%%%%%%%%%%%%%%%%%%%%%%%%%%%%%%%%%%%%%%%%%%%%%%%%%%%%%%%%
\subsection{The \emph{CGPGUID} Command}
\label{scff0:cgp0}

Checks and optionally generates a project GUID.


%%%%%%%%%%%%%%%%%%%%%%%%%%%%%%%%%%%%%%%%%%%%%%%%%%%%%%%%%%%%%%%%%%%%%%%%%%%%%%%
\subsection{The \emph{LDPGUID} Command}
\label{scff0:lgp0}

Loads a project GUID.  Sticky until the next such command.


%%%%%%%%%%%%%%%%%%%%%%%%%%%%%%%%%%%%%%%%%%%%%%%%%%%%%%%%%%%%%%%%%%%%%%%%%%%%%%%
\subsection{The \emph{FMOD} Command}
\label{scff0:sfmd0}

TBD.

%%%%%%%%%%%%%%%%%%%%%%%%%%%%%%%%%%%%%%%%%%%%%%%%%%%%%%%%%%%%%%%%%%%%%%%%%%%%%%%
\subsection{The \emph{FCOPY} Command}
\label{scff0:sfcp0}

TBD.

%%%%%%%%%%%%%%%%%%%%%%%%%%%%%%%%%%%%%%%%%%%%%%%%%%%%%%%%%%%%%%%%%%%%%%%%%%%%%%%
\subsection{The \emph{FDEL} Command}
\label{scff0:sfde0}

TBD.  May accept wildcards here.


%%%%%%%%%%%%%%%%%%%%%%%%%%%%%%%%%%%%%%%%%%%%%%%%%%%%%%%%%%%%%%%%%%%%%%%%%%%%%%%
\subsection{The \emph{DDEL} Command}
\label{scff0:sdde0}

TBD.

%%%%%%%%%%%%%%%%%%%%%%%%%%%%%%%%%%%%%%%%%%%%%%%%%%%%%%%%%%%%%%%%%%%%%%%%%%%%%%%
\subsection{The \emph{EXEC} Command}
\label{scff0:sexc0}

TBD.  Must support EXE's, batch files, and scripts.


%%%%%%%%%%%%%%%%%%%%%%%%%%%%%%%%%%%%%%%%%%%%%%%%%%%%%%%%%%%%%%%%%%%%%%%%%%%%%%%
\subsection{Example Configuration File Listing}
\label{scff0:sexp0}

TBD.

%%%%%%%%%%%%%%%%%%%%%%%%%%%%%%%%%%%%%%%%%%%%%%%%%%%%%%%%%%%%%%%%%%%%%%%%%%%%%%%
\end{document}
%%%%%%%%%%%%%%%%%%%%%%%%%%%%%%%%%%%%%%%%%%%%%%%%%%%%%%%%%%%%%%%%%%%%%%%%%%%%%%%

