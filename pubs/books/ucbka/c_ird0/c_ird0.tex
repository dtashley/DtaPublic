\chapter{\cirdzerolongtitle{}}

\label{cird0}

\section{Introduction To \emph{Insektengericht} And Lessons Learned}

\emph{Insektengericht}\index{Insektengericht}
(German language:  \emph{insect court}) is a word
we learned from a page of a \emph{Beavis and Butthead}\index{Beavis and Butthead}
comic book purchased
in the M\"unchen Hauptbahnhoff around 1995.  In a page of this comic
book---titled \emph{Insektengericht}---Beavis and Butthead sit in judgement
of insects.  Without exception, the verdict is
\emph{schuldig} (guilty), and the sentence is harsh
(\emph{Tod durch explodieren}, or ``death by explosion'', is typical in the
Beavis and Butthead schema of insect justice).

Just like Beavis and Butthead, in this section we sit in judgement of software
defects (or bugs)---probably with a similar conviction rate and similar
sentencing guidelines.

The importance of analyzing software defects can't be overemphasized.
The analysis of defects supplies valuable information about what 
\emph{can} go wrong and what is \emph{most likely} to go wrong.
Because it isn't possible in any product development
process to eliminate \emph{all} types of
software defects, analysis of the relative frequency
of different types of software defects allows
one to concentrate on the ``hotspots'', or types of
software defects shown most likely to occur.

In this part of the work (\emph{Insektengericht}) we strive
to present each different \emph{type} of software defect we've
encountered, and we try to avoid presenting nearly
identical defects, as this is redundant and has no 
instructional value.  We haven't kept track of 
the relative frequency of different types of defects.

We begin in this chapter with a mother lode of intermittent software defects, 
interrupt service.


\section{Exercises}

%End of file c_ird0.tex

