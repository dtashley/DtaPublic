%$Header: svn://localhost/dtapublic/pubs/books/ucbka/trunk/volshare/workprfa.tex 274 2019-08-11 21:43:05Z dashley $
%
%------------------------------------------------------------------------------------------------------------------
%This file isn't used directly--instead, the Tcl script CP_SCRIPT.TCL inserts it into the the main .TEX volume.
%In this process, the <mv> and <sv> tags on some lines are processed.  <mv> lines go only to the multi-volume
%work, and <sv> lines to only to the single-volume work.
%
%For the multi-volume work, the symbolic patchup is near perfect.  But for the single-volume work, the parts
%are numbered manually, which is a process open to human error.  This preface must be checked when chapters
%are moved, added, or deleted.
%------------------------------------------------------------------------------------------------------------------
\chapter{Preface}

In 1998, I began work on the
interrupt latency compatibility problem.  As I searched
for prior work, I was
surprised to discover the paucity
of research available on practical embedded software
problems.  Textbooks were also not very helpful,
as I found that almost none
of the hard
problems that occur in practice
were addressed.

In 1999, I decided to compose my own book,
maintained on the Web,
to collect useful
research results, solutions to practical problems,
unsolved problems, and directions for future
research and refinement.
\LaTeX{} was chosen as the text-processing
tool, as it was 
recognized immediately
that size could become an issue and that the book
would have a substantial mathematical content.
The freedom from any practical length
limitation
means that I am
free to include as many topics
in as much detail as I'd like.\footnote{My readers will
be consuming \emph{their} toner cartridges, not mine.}

In 2001, this book was moved to 
\index{SourceForge}\emph{SourceForge}\footnote{\emph{SourceForge}
(\texttt{http://www.sourceforge.net}), to whom I am very grateful,
is an organization that hosts open-source projects.} and
combined with a tool
set for microcontroller software work 
(\emph{The Iju Tool Set}).

In 2002, the collective set of materials (the tool set and this book) 
was renamed \emph{The ESRG Tool Set}\footnote{\emph{ESRG}
is an acronym for \emph{E}mbedded \emph{S}ystems \emph{R}esearch \emph{G}roup.}, 
and placed under a different
project at \emph{SourceForge}.

\emph{The ESRG Tool Set}, which is fully open-source and 
available for free from \emph{SourceForge} at the URL
\texttt{http://esrg.sourceforge.net}, contains 
these materials: 

\begin{itemize} 
\item A \texttt{.PDF} copy of the most recent released
      version of this book.

\item The packaged \LaTeX{} source code and graphics
      for the most recent released
      version of this book.

\item The most recent (unreleased) source code for this book
      (available from the CVS version control archives).

\item The most recent released version of 
      \emph{The ESRG Tool Set}, packaged as an installation
      for Windows 95/98/ME/NT/2000/XP.  This installation also
	  contains web pages with helpful URLs, and supplemental
	  materials (such as spreadsheets to demonstrate techniques,
	  other documents and standards maintained as part of this
	  effort,
	  Java applets, etc.).

\item The packaged source code for the most recent released
      version of \emph{The ESRG Tool Set}.

\item The most recent (unreleased) source code of
      \emph{The ESRG Tool Set} (available from the CVS
      version control arvhives).

\item The most recent (unreleased) versions of the non-tool
      materials maintained with \emph{The ESRG Tool Set}, including
	  spreadsheets to demonstrate techniques, other
      documents and standards maintained as part of this
	  effort, Java applets, etc. (available from the CVS version control
	  archives).
\end{itemize} 


I have produced this set of works with the following four goals.
\marginpar{\textbf{Goals of This Work}}
\begin{itemize}
\item \textbf{To share results and coordinate work
   with researchers.}  I have
   found that there is a disconnect between practitioners
   (those who have exposure to difficult problems) and
   researchers (those who often have the ability to solve these
   difficult problems).  A large work is required to
   explain the design challenges present in practical embedded software
   work, and to place the most difficult unsolved problems in the hands
   of researchers.
\item \textbf{To share results and coordinate work with practitioners.}
   Many of the problems and solutions presented in this work apply
   to almost all applications of embedded software, and in many cases
   are inherent to \emph{all} software.  These problems and solutions
   will be useful to software practitioners.  I also expect
   that practitioners have strong solutions to contribute to this
   work.\footnote{With less jubilation, I also expect that
   practitioners have new hard problems to contribute.
   With any luck, I will obtain more new solutions than new problems.}
\item \textbf{To enrich the educational experience for students.}
   All of the material contained in this set of works is free\footnote{Subject
   to license restrictions.  The license under which this book may be used
   is included later in this preface.  Software tools are licensed under the GPL.} (which,
   from a student's point of view, is the best price).  
   For example, this book can be used by instructors as
   free supplemental
   material for a course.
\item \textbf{To guide tool vendors.}  The design of tools
   is very much like the design of software, in that one must have
   a nearly complete understanding of the requirements
   in order to produce a useful product.  I have often found that
   tool vendors are unaware of the actual needs of embedded
   software developers.  By clearly enumerating requirements and
   lessons learned, I strive to give tool vendors the information
   and insight
   necessary to better support embedded software developers.
\end{itemize}

In \marginpar{\textbf{\emph{Small} Defined}}
the title of this book, I've indicated that this book is
about \index{small microcontroller system}\emph{small} microcontroller work.  By
\emph{small microcontroller work}, I mean the development
and production of small
embedded
software/hardware
systems such as are found
in consumer electronics and the automobile industry.  Although
it is difficult to specify precisely what is and is not a
\emph{small} microcontroller system, the systems
I have in mind generally meet
the following criteria.

\begin{itemize}
\item They
involve a single-chip control solution (the address and data busses of
the microcontroller are not extended outside the package).
\item They
involve ROM and RAM sizes of under 1M.\footnote{But usually much less--256K or less of ROM and 24K
or less of RAM would be typical for my definition of \emph{small}.}
\item They frequently involve non-preemptive scheduling.  In the systems I would
classify as \emph{small}, pre-emptive scheduling and/or commercial
RTOSs are relatively uncommon.
\item They involve a software
program that cannot easily be changed (either because the program
is lithographically masked into the microcontroller or because special
equipment or special access to the microcontroller is required).\footnote{A consequence of the
software unchangeability
combined with high production volumes is
that software mistakes are very, very expensive.}
\item They exist in an unstable electrical environment,
characterized by supply voltage fluctuations,
electrical transients, and spurious resets.
\end{itemize}

I use the term \index{hard problem}\emph{hard problem} throughout this book.
I define a
\marginpar{\textbf{\emph{Hard Problem} Defined}}
\emph{hard} problem as a problem that
meets the following two criteria.
\begin{itemize}
   \item \textbf{The problem is \emph{hard}.}
         The time to solve the problem is measured in
         months or years, rather than hours or days; and furthermore,
         the problem may be beyond the analytical ability of
         most practicing embedded software engineers.  In practice, this means that
         an embedded software engineer cannot solve the problem in the time
         available during a product development cycle.
   \item \textbf{A solution to the problem is necessary for
         reliable software or for ease of software construction.}
         Generally, I would wish to exclude from consideration
         problems which are difficult but which have no practical implications
         for software reliability or ease of software construction
         if they are not solved.
\end{itemize}

Clearly, if an embedded software engineer is to be productive and
produce a reliable embedded product, the number of unsolved
\emph{hard} problems s/he encounters during a career must
be fairly small.  However, the reality is that a practicing
embedded software engineer encounters unsolved hard problems
on a daily basis.

In this work I take a special but not exclusive interest in
\emph{hard} problems---those that are relevant and necessary
but
so difficult and time-consuming that a software engineer cannot
solve them in the time available during the product development
cycle.  I also have some interest in paradigms (ways of
thinking about software systems and problems), and in
best and worst practices (practices that are likely to to
avert problems and practices that are nearly suicidal,
respectively).  It is noteworthy that I do not
have much interest in established
mathematical results that are available in
textbooks my readers probably already own.
For example, I have very limited interest in
reprinting the mathematical results surrounding
sampled control systems, but
I might be very interested in results predicting what
could and could not happen if such a
sampled control system is implemented on
a microcontroller using economical but lossy
multiplication and integration (to the best of my
knowledge, this insight does not exist anywhere).

<mv>\textbf{Volume \vconzeroroman{}: \vconzerotitle{}}
<sv>\textbf{Part I:  Concepts}
\marginpar{\textbf{Book Contents}}
contains an introduction to the challenges of small microcontroller work.
\begin{itemize}
\item \textbf{Chapter\;\ref{cint0}, \cintzerotitle{}} explains the scope of the book 
      and what is meant by \emph{small microcontroller work}.
\item \textbf{Chapter\;\ref{chgr0}, \chgrzerotitle{}} provides my personal 
      interpretation of the
      Holy Grail of embedded real-time software---that is, what is
      desirable and undesirable and what fundamental goals
      exist.  In this chapter, I deal primarily in paradigms of
      thought and
      unattainable goals.
\end{itemize}

<mv>\textbf{Volume \vmfrzeroroman{}: \vmfrzerotitle{}} 
<sv>\textbf{Part II:  Mathematical Frameworks}
contains key mathematical frameworks and
results that are divorced from any specific problem.
\begin{itemize}
\item \textbf{Chapter\;\ref{cpri0}, \cprizerotitle{}} presents results surrounding
      properties of integers and prime and composite numbers.
\item \textbf{Chapter\;\ref{cfry0}, \cfryzerotitle{}} presents results surrounding
      the \index{Farey series}Farey series.  The Farey series, which is the 
      ordered set of rational numbers with a certain maximum denominator,
      forms the basis for some techniques of rational approximation.
\item \textbf{Chapter\;\ref{ccfr0}, \ccfrzerotitle{}} presents results surrounding
      \index{continued fractions}continued fractions.  The apparatus of continued
      fractions provides the best algorithm for finding Farey neighbors and hence
      best rational approximations.
\item \textbf{Chapter\;\ref{cbal0}, \cbalzerotitle{}} presents results surrounding
      Boolean algebra.  Traditionally, Boolean algebra and the study of Boolean
      functions is associated with digital hardware design, but results and
      algorithms involving the simplification of Boolean functions also have
      application to the simplification of software control flow constructs.
\item \textbf{Chapter\;\ref{cqua0}, \cquazerotitle{}} presents results surrounding
      the analysis of calculation error introduced by \emph{quantization}, the
      unavoidable mapping from $\vworkrealset{}$ to $\vworkintset{}$ that must occur
      as signals from the external world, which are usually continuous in nature,
      are converted and processed by a microcontroller, which can inherently manipulate
      only integers.
\item \textbf{Chapter\;\ref{cmtn0}, \cmtnzerotitle{}} presents miscellaneous results
      from number theory which are useful or interesting, but awkward to
      categorize.
\end{itemize}


<mv>\textbf{Volume \vcswzeroroman{}:  \vcswzerotitle{}}
<sv>\textbf{Part III:  Construction Of Embedded Software}
details the way embedded systems are (or should be)
constructed in practice.
\begin{itemize}
\item \textbf{Chapter\;\ref{cpco0}, \cpcozerotitle{}} outlines many aspects of
      the way that embedded software is constructed (or should be constructed) 
      in practice.

\item \textbf{Chapter\;\ref{csoc0}, \csoczerotitle{}} treats issues specific to 
      the support of peripherals that are usually located on the same silicon
      die as the microcontroller, such as hardware watchdogs or EEPROM.

\item \textbf{Chapter\;\ref{csoc1}, \csoconetitle{}} treats issues specific to 
      the support of peripherals that are usually separate from the microcontroller,
      such as transducers and actuators.

\item \textbf{Chapter\;\ref{csnc0}, \csnczerotitle{}} discusses the support
      of communication protocols and networks.

\item \textbf{Chapter\;\ref{csfo0}, \csfozerotitle{}} discusses requirements
      that tend to occur in many different types of embedded software,
      such as the need to calibrate an embedded product or diagnose it
      via a communication link.

\item \textbf{Chapter\;\ref{crta0}, \crtazerotitle{}} discusses the real-time
      analysis of embedded software; that is, how embedded software can be
      analyzed to be sure that it will always meet its real-time constraints.
\end{itemize}


<mv>\textbf{Volume \valgzeroroman{}: \valgzerotitle{}} 
<sv>\textbf{Part IV:  Embedded System Algorithms And Techniques}
presents algorithms which are used in a target embedded system.

\begin{itemize}
\item \textbf{Chapter\;\ref{ccil0}, \ccilzerotitle{}} presents
      classical and simple\footnote{Here \emph{classical} is used
	  in the same sense as Knuth uses it in \cite[p. 265]{bibref:b:knuthclassic2ndedvol2}.} 
	  integer algorithms and techniques which 
      are used in embedded systems.  Because integer and fixed point algorithms
      are the dominant paradigm of arithmetic in small microcontroller
      systems, this chapter is the backbone of most arithmetic in these systems.

\item \textbf{Chapter\;\ref{crat0}, \cratzerotitle{}} develops techniques
      and results surrounding rational approximation; that is, arpproximating a real number
      $r_I$ by a rational number $h/k$.

\item \textbf{Chapter\;\ref{cdta0}, \cdtazerotitle{}} presents algorithms
      for use in discrete time (such as integration and differentiation),
      and also more sophisticated integer algorithms.

\item \textbf{Chapter\;\ref{cnnu0}, \cnnuzerotitle{}} presents miscellaneous
      non-numerical algorithms for use in target embedded systems.
\end{itemize}


<mv>\textbf{Volume \vpaczeroroman{}: \vpaczerotitle{}} 
<sv>\textbf{Part V:  Practical, Administrative, Incidental, And Miscellaneous Topics}
discusses topics that are non-technical, only peripherally related to embedded system
software development,
or difficult to categorize elsewhere.

\begin{itemize}
\item \textbf{Chapter\;\ref{cmpd0}, \cmpdzerotitle{}} discusses practical technqiues for
      the management of materials produced during the product development process.

\item \textbf{Chapter\;\ref{cpit0}, \cpitzerotitle{}} provides designs and some
      analysis of circuits that are useful in microcontroller software 
	  development---either in microcontroller products themselves or in
	  test fixtures.

\item \textbf{Chapter\;\ref{cbma0}, \cbmazerotitle{}} discusses
      bad management and unpleasant work situations, provides paradigms
      and strategies for dealing with unpleasant work situations, and provides
	  guidance in finding employment.

\item \textbf{Chapter\;\ref{cpxf0}, \cpxfzerotitle{}} describes software and
      hardware products which are exceptionally useful in embedded product development.

\item \textbf{Chapter\;\ref{corq0}, \corqzerotitle{}} describes open research
      questions, i.e. problems I am aware of but do not know how to solve.

\item \textbf{Chapter\;\ref{cisk0}, \ciskzerotitle{}} contains reverse analyses of 
      embedded product defects, and the lessons to be learned from these 
	  defects.\footnote{The word \emph{Insektengericht} (``insect court'', in German)
	  comes from a page in a \emph{Beavis and Butthead} comic book purchsed in 
	  the M\"unchen Hauptbahnhoff around 1996.  In this comic book, Beavis and 
	  Butthead sit in judgement of insects (bugs), always with the verdict 
	  \emph{schuldig} (guilty), and always with harsh sentences such as 
	  \emph{Tod durch explodieren} (death by explosion).  I do much the same 
	  thing with embedded software bugs and embedded product defects.}
	  Because it isn't possible in the product development process to address
	  \emph{every} possible source of defect, I feel it is important to
	  have a visceral feel for what is most likely to go wrong; and such
	  intuition can only come from actively collecting product defects.  I very much
	  hope that airline pilots study transcripts and reports from airplane crashes, and I
	  encourage embedded software developers to do the same.
\end{itemize}


<mv>\textbf{Volume \vijtzeroroman{}: \vijtzerotitle{}}
<sv>\textbf{Part VI:  ESRG Tool Set Reference Guide}
is a reference for the research tool set for microcontroller 
work (based on Tcl/Tk from Scriptics/Ajuba/Interwoven), which is packaged
with this book.

\begin{itemize}
\item \textbf{Chapter\;\ref{ctin0}, \ctinzerotitle{}} contains an overview
      of the tool set and its design goals and construction.

\item \textbf{Chapter\;\ref{ctcm0}, \ctcmzerotitle{}} provides a reference
      for the Tcl scripting language, around which the tool set is constructed.

\item \textbf{Chapter\;\ref{ctkm0}, \ctkmzerotitle{}} provides a reference
      for the Tk graphical extensions to Tcl.  These graphical extensions,
	  which are built into the tool set, are very useful for providing
	  attractive GUI interfaces.

\item \textbf{Chapter\;\ref{cfaq0}, \cfaqzerotitle{}} addresses frequently
      asked questions and ``gotchas'' in Tcl and Tk.

\item \textbf{Chapter\;\ref{cxtn0}, \cxtnzerotitle{}} documents the Tcl
      extensions which have been added to the script interpreters
	  (\emph{EsrgScripter} and \emph{EsrgConsole}).  These extensions form the
	  ``meat'' of \emph{The ESRG Tool Set}, and represent the new ideas or
	  new effort we are trying to convey.  These new ideas and
      new effort are embedded into  Tcl interpreters to grant
	  greater flexibility in their application.

\item \textbf{Chapter\;\ref{cdcm0}, \cdcmzerotitle{}} documents the DOS command-line
      utilities packaged with the tool set.  In many cases, these utilities
	  perform functions identical or similar to Tcl extensions, but there are advantages to
	  packaging functionality as a DOS command-line utility.
\end{itemize}


<mv>\textbf{Volume \vsmazeroroman{}: \vsmazerotitle{}}
<sv>\textbf{Part VIII:  Solutions To Selected Exercises}
contains solutions for selected exercises
found throughout the work.

Although \marginpar{\textbf{Features For Readers}} I
know that all of my readers will be involved in some way
with technical things---such as mathematics, computer science,
control theory, software quality, or the design and development
of embedded products---I realize that my readers will
vary in what they are trying to obtain from the work.  Some readers
will be very research oriented, and may be trying to solve or pose
a particular hard problem.
Other readers may be looking for a specific practical
solution.

For the more practical reader, I have tried to be complete--that
is, I have tried not to omit information that might be
necessary in understanding my work without additional
research effort.  In addition, a rich set of helpful materials is
packaged with \emph{The ESRG Tool Set} installation.

For the more research-oriented reader, I have tried to allow the
reader to rapidly find books, technical papers, Web sites,
technical experts,
and researchers.  For this reason, books, papers,
Web sites, and individuals
are all cited and referenced using the same
framework in the bibliography.  This means the
reader is directed not only to technical documents,
but also to the authors of the
documents.

Since \marginpar{\textbf{Version Identification}}
this book will be maintained on the Web, the issue arises of how
a reader would know one version of this book from another.
I have provided two mechanisms for this.\footnote{\TeX{}
tradition would demand that I pick a transcendental number
and with each successive version reveal another decimal place,
but that is too much work for me, given that
version control systems will handle the task automatically.}

\begin{itemize}

   \item The title page indicates when the work was compiled from
   \LaTeX{} source code.  Two copies of the work with the
   same compile date are almost certainly identical.

   \item The entire work is maintained under a version control
   system (CVS at \emph{SourceForge}),
   \index{CVS}\index{SourceForge} and I've included
   keyword expansions near the end of the \LaTeX{} source files so that version
   control information is included automatically in the document text (for example,
   a block of version control information is at the end of each chapter).
   Sections or chapters with the same version control information are almost certainly
   identical.

\end{itemize}

This \marginpar{\textbf{Book License}}\index{license} book is licensed under the 
\emph{Academic Free License v. 2.0}, reproduced below for convenience.\footnote{Please
note that \emph{The ESRG Tool Set} and certain other materials are licensed under the GPL.  At this time
only this book is licensed under the \emph{Academic Free License}.}

\begin{it}
This Academic Free License (the ``License'') applies to any original work of authorship 
(the ``Original Work'') whose owner (the ``Licensor'') has placed the following 
notice immediately following the copyright notice for the Original Work:\\
\\
Licensed under the Academic Free License version 2.0

\begin{enumerate}
\item \textbf{Grant of Copyright License.} Licensor hereby grants You a world-wide, royalty-free, 
      non-exclusive, perpetual, sublicenseable license to do the following:
	  
   \begin{enumerate}
   \item to reproduce the Original Work in copies; 
   \item to prepare derivative works ("Derivative Works") based upon the Original Work; 
   \item to distribute copies of the Original Work and Derivative Works to the public; 
   \item to perform the Original Work publicly; and 
   \item to display the Original Work publicly. 
   \end{enumerate}

\item \textbf{Grant of Patent License.} Licensor hereby grants You a world-wide, royalty-free, non-exclusive, 
      perpetual, sublicenseable license, under patent claims owned or controlled by the Licensor that are embodied 
	  in the Original Work as furnished by the Licensor, to make, use, sell and offer for sale the 
	  Original Work and Derivative Works. 
   	   
\item \textbf{Grant of Source Code License.}  The term ``Source Code'' means the preferred form of the 
      Original Work for making modifications to it and all available documentation describing how to modify the 
	  Original Work. Licensor hereby agrees to provide a machine-readable copy of the Source Code of the 
	  Original Work along with each copy of the Original Work that Licensor distributes. 
	  Licensor reserves the right to satisfy this obligation by placing a machine-readable copy of 
	  the Source Code in an information repository reasonably calculated to permit inexpensive and convenient 
	  access by You for as long as Licensor continues to distribute the Original Work, and by publishing the 
	  address of that information repository in a notice immediately following the copyright 
	  notice that applies to the Original Work. 

\item \textbf{Exclusions From License Grant.}  Neither the names of Licensor, nor the names of any 
      contributors to the Original Work, nor any of their trademarks or service marks, may 
	  be used to endorse or promote products derived from this Original Work without express prior 
	  written permission of the Licensor.  Nothing in this License shall be deemed to grant any 
	  rights to trademarks, copyrights, patents, trade secrets or any other intellectual property 
	  of Licensor except as expressly stated herein. No patent license is granted to make, use, 
	  sell or offer to sell embodiments of any patent claims other than the licensed claims 
	  defined in Section 2.  No right is granted to the trademarks of Licensor even if such marks 
	  are included in the Original Work. Nothing in this License shall be interpreted to prohibit 
	  Licensor from licensing under different terms from this License any Original Work that 
	  Licensor otherwise would have a right to license. 

\item This section intentionally omitted. 

\item \textbf{Attribution Rights.}  You must retain, in the Source Code of any Derivative Works that
      You create, all copyright, patent or trademark notices from the Source Code of the Original Work, 
	  as well as any notices of licensing and any descriptive text identified therein as an 
	  ``Attribution Notice.''  You must cause the Source Code for any Derivative Works that 
	  You create to carry a prominent Attribution Notice reasonably calculated to inform 
	  recipients that You have modified the Original Work. 

\item \textbf{Warranty of Provenance and Disclaimer of Warranty.}  Licensor warrants that the copyright in 
      and to the Original Work and the patent rights granted herein by Licensor are owned by the 
	  Licensor or are sublicensed to You under the terms of this License with the permission of 
	  the contributor(s) of those copyrights and patent rights.  Except as expressly stated in the 
	  immediately proceeding sentence, the Original Work is provided under this License on an 
	  ``AS IS'' BASIS and WITHOUT WARRANTY, either express or implied, including, without limitation, 
	  the warranties of NON-INFRINGEMENT, MERCHANTABILITY or FITNESS FOR A PARTICULAR PURPOSE. 
	  THE ENTIRE RISK AS TO THE QUALITY OF THE ORIGINAL WORK IS WITH YOU.  This DISCLAIMER OF WARRANTY 
	  constitutes an essential part of this License.  No license to Original Work is granted 
	  hereunder except under this disclaimer. 

\item \textbf{Limitation of Liability.}  Under no circumstances and under no legal theory, whether in 
      tort (including negligence), contract, or otherwise, shall the Licensor be liable to any person for 
	  any direct, indirect, special, incidental, or consequential damages of any character arising 
	  as a result of this License or the use of the Original Work including, without limitation, 
	  damages for loss of goodwill, work stoppage, computer failure or malfunction, or any and all 
	  other commercial damages or losses. This limitation of liability shall not apply to liability 
	  for death or personal injury resulting from Licensor's negligence to the extent applicable law 
	  prohibits such limitation. Some jurisdictions do not allow the exclusion or limitation of 
	  incidental or consequential damages, so this exclusion and limitation may not apply to You. 

\item \textbf{Acceptance and Termination.}  If You distribute copies of the Original Work or a 
      Derivative Work, You must make a reasonable effort under the circumstances to 
      obtain the express assent of recipients to the terms of this License. Nothing 
      else but this License (or another written agreement between Licensor and You) 
      grants You permission to create Derivative Works based upon the Original Work 
      or to exercise any of the rights granted in Section 1 herein, and any attempt 
      to do so except under the terms of this License (or another written agreement 
      between Licensor and You) is expressly prohibited by U.S. copyright law, the 
      equivalent laws of other countries, and by international treaty. Therefore, 
      by exercising any of the rights granted to You in Section 1 herein, You indicate 
      Your acceptance of this License and all of its terms and conditions. 

\item \textbf{Termination for Patent Action.}  This License shall terminate automatically 
      and You may no longer exercise any of the rights granted to You by this License 
      as of the date You commence an action, including a cross-claim or counterclaim, 
      for patent infringement (i) against Licensor with respect to a patent applicable 
      to software or (ii) against any entity with respect to a patent applicable to 
      the Original Work (but excluding combinations of the Original Work with other 
      software or hardware). 

\item \textbf{Jurisdiction, Venue and Governing Law.}  Any action or suit relating to this 
      License may be brought only in the courts of a jurisdiction wherein the Licensor 
      resides or in which Licensor conducts its primary business, and under the laws
      of that jurisdiction excluding its conflict-of-law provisions. The application 
      of the United Nations Convention on Contracts for the International Sale of Goods 
      is expressly excluded. Any use of the Original Work outside the scope of this 
      License or after its termination shall be subject to the requirements and 
      penalties of the U.S. Copyright Act, 17 U.S.C. 101 et seq., the equivalent 
      laws of other countries, and international treaty. This section shall 
      survive the termination of this License. 

\item \textbf{Attorneys Fees.}  In any action to enforce the terms of this License or 
      seeking damages relating thereto, the prevailing party shall be entitled 
      to recover its costs and expenses, including, without limitation, reasonable 
      attorneys' fees and costs incurred in connection with such action, including
      any appeal of such action. This section shall survive the termination of 
      this License. 

\item \textbf{Miscellaneous.} This License represents the complete agreement 
      concerning the subject matter hereof. If any provision of this License 
      is held to be unenforceable, such provision shall be reformed only to 
      the extent necessary to make it enforceable. 

\item \textbf{Definition of ``You'' in This License.} ``You'' throughout this License, 
      whether in upper or lower case, means an individual or a legal entity 
      exercising rights under, and complying with all of the terms of, this 
      License. For legal entities, ``You'' includes any entity that controls, 
      is controlled by, or is under common control with you. For purposes of 
      this definition, ``control'' means (i) the power, direct or indirect, 
      to cause the direction or management of such entity, whether by 
      contract or otherwise, or (ii) ownership of fifty percent (50%) 
      or more of the outstanding shares, or (iii) beneficial ownership 
      of such entity. 

\item \textbf{Right to Use.}  You may use the Original Work in all ways not 
      otherwise restricted or conditioned by this License or by law, 
      and Licensor promises not to interfere with or be responsible 
      for such uses by You. 
\end{enumerate}
\end{it}


I \marginpar{\textbf{Intellectual Property}} believe very
strongly in the notion of intellectual property.  I have endeavored
in this book to cite all references and to identify all work and
all ideas that are not my own.\footnote{I've actually gone a step
further---I've tried to provide enough information to allow a
motivated reader to get in touch with the originator of the work
or ideas.}  I would request the same courtesy in return (but will
not enforce this courtesy, except by exposure of parties responsible
for the theft of ideas without credit).  Plagiarism and theft of intellectual property is a 
regrettable but common occurrence, and I can
reliably predict that some student somewhere in the Webbed world
will borrow my ideas or work without citation
for a semester project---such events would earn my scorn 
but be relatively insignificant
on my radar screen.  However, if any of my work appears without
credit in academic journals, I would take the matter less
jovially and would very likely report the plagiarism to the journal 
involved.

I \marginpar{\textbf{Contacting the Author}}\index{contacting the author}
can be contacted at \texttt{dtashley@aol.com}.  I invite readers
to contact me about
any material in this book, in \emph{The ESRG Tool Set}, or at
the web site \texttt{http://esrg.sourceforge.net}.

%%%%%%%%%%%%%%%%%%%%%%%%%%%%%%%%%%%%%%%%%%%%%%%%%%%%%%%%%%%%%%%%%%%%%%%%%%

\noindent\begin{figure}[!b]
\noindent\rule[-0.25in]{\textwidth}{1pt}
\begin{tiny}
\begin{verbatim}
$HeadURL: svn://localhost/dtapublic/pubs/books/ucbka/trunk/volshare/workprfa.tex $
$Revision: 274 $
$Date: 2019-08-11 17:43:05 -0400 (Sun, 11 Aug 2019) $
$Author: dashley $
\end{verbatim}
\end{tiny}
\noindent\rule[0.25in]{\textwidth}{1pt}
\end{figure}
%
%End of file WORKPRFA.TEX
