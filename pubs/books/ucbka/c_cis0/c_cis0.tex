%$Header: svn://localhost/dtapublic/pubs/books/ucbka/trunk/c_cis0/c_cis0.tex 277 2019-08-13 02:35:39Z dashley $

\chapter[Solutions: \ccilzeroxrefcomma{}Chapter \ref{ccil0}]
        {Solutions: \ccilzeroxrefcomma{}Chapter \ref{ccil0}, \ccilzerolongtitle{}}

\label{ccis0}

\vworkexercisechapterheader{}
\begin{vworkexercisesolution}{\ref{exe:ccil0:sexe0:01}}
We can show this result in two ways.  The first way, based on bit patterns, is to note
that adding an $m$-bit number, $u$, to its one's complement will result in a bit pattern
containing all 1's, i.e. $\forall i$, $u_{[i]} = 1$.  Adding 1 to this bit pattern will
always produce $\forall i$, $u_{[i]} = 0$ with a carry out.  Since the order of addition
does not matter, this establishes that adding $u$ to $\sim{}u+1$ will produce 0, thus showing
that $u$ and $\sim{}u+1$ are additive inverses.  This method, although valid, does not 
establish that $u$ and $\sim{}u+1$ actually represent additive inverses.  For example, if
$u=-2^{m-1}$, $u=\sim{}u+1$, and clearly a non-zero number cannot be an additive inverse of
itself.  Thus, it would be more comforting to show this result in a way that demonstrates the
actual values of the integers represented.

We present a second method now.  Assume that $u \neq -2^{m-1}$, since 
$-2^{m-1}$ cannot have an additive inverse in an $m$-bit signed integer.
If $u=0$, $\sim{}u+1=0$, so the relationship is clearly met.  If $u<0$, then
$u_{[m-1]}=1$, and by
(\ccilzeroxrefhyphen\ref{eq:ccil0:sroi0:sros0:00}),


\end{vworkexercisesolution}
\vworkexercisechapterfooter

%%%%%%%%%%%%%%%%%%%%%%%%%%%%%%%%%%%%%%%%%%%%%%%%%%%%%%%%%%%%%%%%%%%%%%%%%%
\vfill
\noindent\begin{figure}[!b]
\noindent\rule[-0.25in]{\textwidth}{1pt}
\begin{tiny}
\begin{verbatim}
$HeadURL: svn://localhost/dtapublic/pubs/books/ucbka/trunk/c_cis0/c_cis0.tex $
$Revision: 277 $
$Date: 2019-08-12 22:35:39 -0400 (Mon, 12 Aug 2019) $
$Author: dashley $
\end{verbatim}
\end{tiny}
\noindent\rule[0.25in]{\textwidth}{1pt}
\end{figure}
%%%%%%%%%%%%%%%%%%%%%%%%%%%%%%%%%%%%%%%%%%%%%%%%%%%%%%%%%%%%%%%%%%%%%%%%%%
%
%End of file C_CIS0.TEX
