\chapter{\cbmazerolongtitle{}}

\label{cbma0}

\beginchapterquote{``Being responsible sometimes means pissing people
                     off.  Good leadership involves responsibility to the
					 welfare of the group, which means that some people
					 will get angry at your actions and decisons.
					 It's inevitable, if you're honorable.  Trying to get
					 everyone to like you is a sign of mediocrity:
					 you'll avoid the tough decisions, you'll avoid confronting
					 the people who need to be confronted, and you'll avoid
					 offering differential rewards based on differential
					 performance because some people might get upset.
					 Ironically, by procrastinating on the difficult choices,
					 by trying not to get anyone mad, and by treating everyone
					 equally `nicely' regardless of their contributions,
					 you'll simply ensure that the only people you'll wind
					 up angering are the most creative and productive
					 people in the 
					 \index{Powell, Colin}organization.''\footnote{General Powell's
					 presentation (\cite{bibref:d:powellleadershipprimer})
					 is an absolute goldmine of tremendous quotes.  There were
					 many equally striking contenders for this spot (the opening quote
					 of the chapter about bad management).}}{General Colin Powell (Retired)
					 \cite{bibref:d:powellleadershipprimer}}

\section{Introduction}
%Section Tag: INT
With the comic strip
\index{Dilbert@\emph{Dilbert}}\emph{Dilbert}, 
and several books, \index{Adams, Scott}Scott Adams made his fortune
anecdotally characterizing bad management.  Certainly, in any
country, \index{bad management}bad management 
is an abundant natural resource
and a shortage of bad management is not
on the horizon.

We are less concerned with the humorous aspects of
bad management and more concerned with the practical 
aspects.  In this chapter, we offer opinion on the
following topics:

\begin{itemize}
\item What \emph{is} bad management (i.e. what do we mean by
      \emph{bad management}
	  and what characterizes bad management)?

\item What do bad managers do?

\item Which employees are most sensitive to bad management?

\item In practical situations, how should one deal with bad
      management?

\item What are the best strategies for escaping unrewarding
      work situations?
\end{itemize}

%%%%%%%%%%%%%%%%%%%%%%%%%%%%%%%%%%%%%%%%%%%%%%%%%%%%%%%%%%%%%%%%%%%%%%%%%%%%%%%

\section{Characteristics Of Bad Management}

TBD.


%%%%%%%%%%%%%%%%%%%%%%%%%%%%%%%%%%%%%%%%%%%%%%%%%%%%%%%%%%%%%%%%%%%%%%%%%%%%%%%

\section{How To Detect Bad Management During The Interview Process}

The interview process is naturally an opportunity for a prospective
employer to form impressions of a prospective employee; but it is
also an opportunity for the prospective employee to form impressions
of the prospective employer.  In this section, we supply some suggestions
about what to look for during an interview.

\subsection{The Automobile Taillight Analogy}

One of us (\index{Ashley, David T.}Dave Ashley, \cite{bibref:i:daveashley})
has an acquaintance who has described his method of evaluating
a used car (for purchase) as checking every electric light in the
vehicle to be sure that it works.  The stated rationale is that if
all of the light bulbs in the vehicle are maintained, the probability
is high that other [major] vehicle maintenance has also been performed.
Similar reasoning \emph{may} (or may not!) apply to evaluating
a work environment.

Stated more formally, it may be advantageous to find easily observable
indicators which correlate well with the quality of the work environment
at a company.

We are not sure precisely what indicators should be used,\footnote{We welcome
suggestions here \ldots please e-mail us \ldots{}} but the two 
strongest indicators that immediately come to mind are coding standards
and lessons learned.

\begin{itemize}
\item \textbf{Coding Standards.}  
      During the interview process, it may be a good idea to inquire about
	  what coding standards are in place within the organization, to 
	  ask to examine the standards, and also to inquire how the coding
	  standards are enforced (in some cases, tools such as QAC or
	  PC-LINT may automate this
	  process).  The rationale for inquiring about coding standards
	  is that maintaining order in the primary workproduct of
	  software development---the code---is a fundamental goal.
	  An organization that has no coding standards in place probably
	  has other serious problems.
\item \textbf{Collection Of Lessons Learned.}
      In any organization that produces embedded products, product failures
	  of one kind or another have probably occured.  These may be cases
	  where a software defect has made its way into production, or even
	  software product build process failures where a software defect
	  was due to the build process or where a software load was not
	  reproducible from version control archives.  A mature organization
	  would document and collect these failures, in order to feed them
	  back into the training (so that software developers don't make a
	  similar mistake again), into the process (if any changes in the process
	  would decisively prevent recurrence), and the tools (if the defect
	  is automatically detectable).  During the interview process, 
	  it may be prudent to inquire if product problems are documented
	  and fed back to prevent recurrence, and to inspect documentation
	  of past product problems.  An organization that does not collect
	  product problems and try to prevent recurrence may have other
	  serious problems.
\end{itemize}

\subsection{The \emph{What You See Is What You Get} Rule}

During the interview process, any prospective employer will have a tendency
to misrepresent chronic problems as acute problems.  As a general rule,
\textbf{problems of any type observed during the interview process are
\emph{chronic} in nature, no matter what claims are made by the employer.}

An analogy involving overweight people may help to explain this point.\footnote{We
mean no disrespect or insensitivity towards people struggling to maintain a
healthy weight.  However, the analogy is rather good, and for this reason
we would like to use it.}  It is very common to meet an overweight person
who describes themselves as ``trying to lose weight'', i.e. being actively
on a diet.  However, a study of overweight people who are ``trying to 
lose weight'' would probably reveal that nearly all of them are struggling
with a chronic problem---nearly all were probably overweight five years
in the past and will be overweight five years in the future.  For this reason,
when one meets an overweight person, it is a safe guess statistically that
the condition is chronic.

Organizations are very similar to individuals in that patterns of behavior
are slow to change.  Diets do not usually work.  Individuals find ways---even
while on a diet---to consume ice cream and hamburgers.  Organizations are
similar in that self-reform measures rarely succeed.  Organizations, like
individuals, find ways to sabotage their own stated objectives.

For this reason, any anomalies observed during the interview process are
almost certainly chronic rather than acute; no matter what claims
are made by interviewers.


%%%%%%%%%%%%%%%%%%%%%%%%%%%%%%%%%%%%%%%%%%%%%%%%%%%%%%%%%%%%%%%%%%%%%%%%%%%%%%%

\section{The Employment ``Dating Game''}

TBD.

%%%%%%%%%%%%%%%%%%%%%%%%%%%%%%%%%%%%%%%%%%%%%%%%%%%%%%%%%%%%%%%%%%%%%%%%%%%%%%%

\section{Authors And Acknowledgements}

TBD.

%End of file c_bma0.tex

