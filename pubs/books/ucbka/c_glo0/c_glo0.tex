%$Header: svn://localhost/dtapublic/pubs/books/ucbka/trunk/c_glo0/c_glo0.tex 278 2019-08-14 23:10:36Z dashley $

\chapter*{Glossary Of Terms}
\markboth{GLOSSARY OF TERMS}{GLOSSARY OF TERMS}

\label{cglo0}

\begin{vworktermglossaryenum}

\item \textbf{axiom}\index{axiom}

      A statement used in the premises of arguments and assumed to be true
	  without proof.  In some cases axioms are held to be self-evident, as in 
	  Euclidian geometry, while in others they are assumptions put forward for
	  the sake of argument.
      (Taken verbatim from \cite{bibref:b:penguindictionaryofmathematics:2ded}.)

\item \textbf{cardinality}\index{cardinality}

      The cardinality of a set is the
      number of elements in the set.  In this work, the cardinality
      of a set is denoted $n()$.  For example, 
      $n(\{12,29,327\}) = 3$.

\item \textbf{coprime}\index{coprime}

      Two integers that share no prime factors are \emph{coprime}.
      \emph{Example:}
      6 and 7 are coprime, whereas 6 and 8 are not.

\item \textbf{GMP}\index{GMP}

      The \emph{G}NU \emph{M}ultiple \emph{P}recision library.
      The GMP is an arbitrary-precision integer, rational number,
      and floating-point library that places no restrictions on
      size of integers or number of significant digits in floating-point
      numbers.  This 
      library is famous because it is the fastest of its
      kind, and generally uses asymptotically superior algorithms.

\item \textbf{greatest common divisor (g.c.d.)}

      The greatest common divisor of two integers is the largest
      integer which divides both integers without a remainder.
      \emph{Example:} the g.c.d. of 30 and 42 is 6.

\item \textbf{integer}\index{integer}\index{sets of integers}\index{Z@$\vworkintset$}%
      \index{integer!Z@$\vworkintset$}\index{integer!sets of}

      (Nearly verbatim from \cite{bibref:w:wwwwhatiscom}) An \emph{integer}
      (pronounced \emph{IN-tuh-jer}) is a whole number
      (not a fractional number) that can be positive, negative, or zero. 

      Examples of integers are: -5, 1, 5, 8, 97, and 3,043. 

      Examples of numbers that are not integers are: -1.43, 1 3/4, 3.14, 
      0.09, and 5,643.1. 

      The set of integers, denoted $\vworkintset{}$, is formally defined as:

      \begin{equation}
      \vworkintset{} = \{\ldots{}, -3, -2, -1, 0, 1, 2, 3, \ldots{} \}
      \end{equation}

      In mathematical equations, unknown or unspecified integers are 
      represented by lowercase, italicized letters from the 
      ``late middle'' of the alphabet.  The most common 
      are $p$, $q$, $r$, and $s$.

\item \textbf{irreducible}

      A rational number $p/q$ where $p$ and $q$ are coprime
      is said to be \emph{irreducible}.
      Equivalently, it may be stated that $p$ and $q$ share no prime factors
      or that the greatest common divisor of
      $p$ and $q$ is 1.

\item \textbf{KPH}

      Kilometers per hour.

\item \textbf{limb}\index{limb}

      An integer of a size which a machine can manipulate natively
      that is arranged in an array to create a larger
      integer which the machine cannot manipulate natively and must be
      manipulated through arithmetic subroutines.

\item \textbf{limbsize}\index{limbsize}

      The size, in bits, of a limb.  The limbsize usually represents
      the size of integer that a machine can manipulate directly
      through machine instructions.  For an inexpensive microcontroller,
      8 or 16 is a typical limbsize.  For a personal computer or 
      workstation, 32 or 64 is a typical limbsize.

\item \textbf{MPH}

      Miles per hour.

\item \textbf{mediant}\index{mediant}

      The mediant of two fractions $m/n$ and $m'/n'$ is the fraction 
	  $\frac{m+m'}{n+n'}$ (see Definition 
	  \cfryzeroxrefhyphen{}\ref{def:cfry0:spfs:02}).  Note that the
	  mediant of two fractions with non-negative integer components
	  is always between them, but not usually exactly at the 
	  midpoint (see Lemma \cfryzeroxrefhyphen{}\ref{lem:cfry0:spfs:02c}).

\item \textbf{natural number}\index{natural number}\index{integer!natural number}%
      \index{sets of integers}\index{N@$\vworkintsetpos$}%
      \index{integer!N@$\vworkintsetpos$}\index{integer!sets of}
         
      (Nearly verbatim from \cite{bibref:w:wwwwhatiscom})
      A \emph{natural number}
      is a number that occurs commonly and obviously in nature.  
      As such, it is a whole, non-negative number.  
      The set of natural numbers, denoted $\vworkintsetpos{}$, 
      can be defined in either of two ways:

      \begin{equation}
      \label{cglo0:eq0001}
      \vworkintsetpos{} = \{ 0, 1, 2, 3, \ldots{} \}
      \end{equation}

      \begin{equation}
      \label{cglo0:eq0002}
      \vworkintsetpos{} = \{ 1, 2, 3, 4, \ldots{} \}
      \end{equation}
      
      In mathematical equations, unknown or unspecified natural numbers 
      are represented by lowercase, italicized letters from the 
      middle of the alphabet.  The most common is $n$, followed by 
      $m$, $p$, and $q$.  
      In subscripts, the lowercase $i$ is sometimes used to represent 
      a non-specific natural number when denoting the elements in a 
      sequence or series.  However, $i$ is more often used to represent 
      the positive square root of -1, the unit imaginary number.

      \textbf{Important Note:}  The definition above is reproduced nearly
      verbatim from \cite{bibref:w:wwwwhatiscom}, and (\ref{cglo0:eq0001})
      is supplied only for perspective.  In this work, a natural
      number is defined by (\ref{cglo0:eq0002}) rather than (\ref{cglo0:eq0001}).
      In this work, the set of non-negative integers is denoted by
      $\vworkintsetnonneg{}$ rather than $\vworkintsetpos{}$.\index{Z+@$\vworkintsetnonneg$}%
      \index{integer!Z+@$\vworkintsetnonneg$}\index{integer!non-negative}

\item \textbf{postulate}\index{postulate!definition}

      An axiom (see \emph{axiom} earlier in this glossary).  The term is usually
	  used in certain contexts, e.g. Euclid's postulates or Peano's postulates.
	  (Taken verbatim from \cite{bibref:b:penguindictionaryofmathematics:2ded}.)

\item \textbf{prime number}\index{prime number!definition}

      (Nearly verbatim from \cite{bibref:w:wwwwhatiscom}) A \emph{prime number}
      is a whole number greater than 1, whose only two whole-number 
      factors are 1 and itself.  The first few prime numbers are 
      2, 3, 5, 7, 11, 13, 17, 19, 23, and 29.  As we proceed in the set of 
      natural numbers $\vworkintsetpos{} = \{ 1, 2, 3, \ldots{} \} $, the 
      primes become less and less frequent in general.  
      However, there is no largest prime number.  
      For every prime number $p$, there exists a prime number $p'$ such that 
      $p'$ is greater than $p$.  This was demonstrated in ancient times by the 
      Greek mathematician \index{Euclid}Euclid.\index{prime number!no largest prime number}%
      \index{Euclid!Second Theorem}

      Suppose $n$ is a whole number, and we want to test it to see if it is prime.   
      First, we take the square root (or the 1/2 power) of $n$; then we round this 
      number up to the next highest whole number.  Call the result $m$.  
      We must find all of the following quotients:

      \begin{equation}
      \begin{array}{rcl}
         q_m     & =        & n / m              \\
         q_{m-1} & =        & n / (m-1)          \\
         q_{m-2} & =        & n / (m-2)          \\
         q_{m-3} & =        & n / (m-3)          \\
                 & \ldots{} &                    \\
         q_3     & =        & n / 3              \\
         q_2     & =        & n / 2              \\
      \end{array}
      \end{equation}

      The number $n$ is prime if and only if none of the $q$'s, as 
      derived above, are whole numbers.

      A computer can be used to test extremely large numbers to see if they are prime.  
      But, because there is no limit to how large a natural number can be, 
      there is always a point where testing in this manner becomes too great 
      a task even for the most powerful supercomputers.  
      Various algorithms have been formulated in an attempt to generate 
      ever-larger prime numbers.  These schemes all have limitations.

\end{vworktermglossaryenum}

%%%%%%%%%%%%%%%%%%%%%%%%%%%%%%%%%%%%%%%%%%%%%%%%%%%%%%%%%%%%%%%%%%%%%%%%%%

\noindent\begin{figure}[!b]
\noindent\rule[-0.25in]{\textwidth}{1pt}
\begin{tiny}
\begin{verbatim}
$HeadURL: svn://localhost/dtapublic/pubs/books/ucbka/trunk/c_glo0/c_glo0.tex $
$Revision: 278 $
$Date: 2019-08-14 19:10:36 -0400 (Wed, 14 Aug 2019) $
$Author: dashley $
\end{verbatim}
\end{tiny}
\noindent\rule[0.25in]{\textwidth}{1pt}
\end{figure}

%%%%%%%%%%%%%%%%%%%%%%%%%%%%%%%%%%%%%%%%%%%%%%%%%%%%%%%%%%%%%%%%%%%%%%%%%%
%
%End of file C_GLO0.TEX
