\chapter[Solutions: Chapter \ref{crat0}]
        {Solutions: Chapter \ref{crat0}, \cratzerolongtitle{}}

\label{cras0}

\vworkexercisechapterheader{}
\begin{vworkexercisesolution}{\ref{exe:crat0:sexe0:01}}
The value of the \texttt{state} variable when 
evaluating the \emph{if()} clause on the
$n+1$'th invocation is

\begin{equation}
\label{eq:cras0:exe01:01}
K_1 - nK_4 + (n+1) K_2 .
\end{equation}

We require on the $n+1$'th invocation, in order for the 
test of the \emph{if()} clause to fail (i.e. that the function
``\texttt{A()}'' has been run on the first $n$ invocations of
the base subroutine but is not run on the $n+1$'th invocation), 
that:

\begin{equation}
\label{eq:cras0:exe01:01b}
K_1 - nK_4 + (n+1) K_2 < K_3.
\end{equation}

Solving this inequality for the smallest integral
value of $n$ yields:

\begin{equation}
\label{eq:cras0:exe01:01c}
n = \left\lceil
\frac{K_1 + K_2 - K_3 + 1}{K_4 - K_2}
\right\rceil .
\end{equation}

It can be verified using an example that 
(\ref{eq:cras0:exe01:01c}).  For example, with
$K_1 = 10$, $K_2 = 3$, $K_3 = 7$, and $K_4 = 5$, 
(\ref{eq:cras0:exe01:01}) predicts that on the first
$\lceil 7/2 \rceil = 4$ invocations of the base subroutine
the subroutine ``\texttt{A()}'' will be run but on the 5th
invocation it will not.  Tracing the algorithm with the
parameters specified reveals that at the
test in the \emph{if()} statement 
on the first invocation of the
subroutine, \texttt{state}=13 (``\texttt{A()}'' executed);
on the second invocation of the
subroutine, \texttt{state}=11 (``\texttt{A()}'' executed);
on the third invocation of the
subroutine, \texttt{state}=9 (``\texttt{A()}'' executed);
on the fourth invocation of the
subroutine, \texttt{state}=7 (``\texttt{A()}'' executed);
and on the fifth invocation of the
subroutine, \texttt{state}=5 (``\texttt{A()}'' not executed).
This is in agreement with
(\ref{eq:cras0:exe01:01}). 
\end{vworkexercisesolution}
%\vworkexerciseseparator
%\begin{vworkexercisesolution}{\ref{exe:cfry0:sexe0:02}}
%Placeholder.
%\end{vworkexercisesolution}
\vworkexercisechapterfooter


%%%%%%%%%%%%%%%%%%%%%%%%%%%%%%%%%%%%%%%%%%%%%%%%%%%%%%%%%%%%%%%%%%%%%%%%%%
%End of file c_ras0.tex

